\documentclass[floatfix,nofootinbib,superscriptaddress,fleqn]{revtex4-2} 
%\documentclass[aps,epsfig,tightlines,fleqn]{revtex4}
\usepackage[utf]{kotex}
\usepackage[HWP]{dhucs-interword}
\usepackage[dvips]{color}
\usepackage{graphicx}
\usepackage{bm}
%\usepackage{fancyhdr}
%\usepackage{dcolumn}
\usepackage{defcolor}
\usepackage{amsmath}
\usepackage{amsfonts}
\usepackage{amssymb}
\usepackage{amscd}
\usepackage{amsthm}
\usepackage[utf8]{inputenc}
 \usepackage{setspace}
 \usepackage{tikz}
%\pagestyle{fancy}

\begin{document}

\title{\Large 2021년 1학기 물리학 I: Quiz 9}
\author{김현철\footnote{Office: 5S-436D (면담시간 매주
    화요일-16:00$\sim$20:00)}} 
\email{hchkim@inha.ac.kr}
\author{Lee Hui-Jae} 
\email{hjlee6674@inha.edu}
\affiliation{Hadron Theory Group, Department of Physics,
Inha University, Incheon 22212, Republic of Korea }
\date{Spring semester, 2021}

\maketitle

\noindent {\bf 문제 1. (30pt)} 

\noindent {\bf 풀이 : } 
\begin{itemize}
  \item[(가)] 깡통 바닥의 중심을 원점으로 하자. $A$를 깡통 옆면의 넓이라고 하면 
  깡통 옆면의 질량 중심의 높이는,
  \begin{align}
    z_{cm} =\frac{1}{M}\int z\,dM=\frac{1}{A}\int z\,dA,\,\,\, A =2\pi R z,
  \end{align}
  를 통해 계산할 수 있다. $R$은 깡통 바닥의 반지름이다.
   $V$를 콜라의 부피라고 하면 콜라의 질량 중심의 높이는,
  \begin{align}
    z_{cm} =\frac{1}{m}\int z\,dm=\frac{1}{V}\int z\,dV,\,\,\,V=\pi R^2 z,
  \end{align}
  를 통해 계산할 수 있다. 따라서 $M$을 깡통의 질량, $m$을 콜라의 질량이라 하면 
  전체 질량 중심의 높이 $h$는,
  \begin{align}
    \begin{split}
      h &= \frac{1}{M+m}\left(\frac{M}{A}\int z\,dA
      +\frac{m}{V}\int z\,dV\right) \\
%      m = 0.140\,\mathrm{kg} = \left(1.40\times 10^{3}\,\mathrm{g}\right)
%      H = \left(12\,\mathrm{cm}\right) 
%      M = \left(0.210\,\mathrm{g}\right)
       &= \frac{1}{M+m}\left(\frac{M}{2\pi R H}\int_0^H 2\pi R z\,dz
      +\frac{m}{\pi R^2 H}\int_0^H\pi R^2 z\,dz\right)  \\
      &= \frac{1}{M+m}\left(\frac{M}{H}\int_0^H z\,dz
      +\frac{m}{H}\int_0^H z\,dz\right) \\
      &= \frac{1}{M+m}\left(
        \frac{M}{H}\frac{1}{2}H^2
        +\frac{m}{H}\frac{1}{2}H^2
      \right)  \\
      &=\frac{1}{2}H,
    \end{split}
  \end{align}
  이다. $H=12$ cm이므로 전체 질량 중심의 높이는 6 cm 이다.
  \item[(나)] 콜라가 모두 빠져 나갔으므로 $h$는,
  \begin{align}
    \begin{split}
      h&=\frac{1}{V}\int z\,dV 
      = \frac{1}{2\pi R H}\int_0^H 2\pi R z\,dz \\
      &=\frac{1}{H}\int_0^H z\,dz \\
      &=\frac{1}{2}H,
    \end{split}
  \end{align}
  이다. 따라서 콜라가 모두 빠져나가도 전체 질량 중심의 높이는 6 cm 이다.
  \item[(다)] 콜라가 빠져나갈 때 콜라의 높이를 $x$라고 하자. 콜라의 높이는 
  더 이상 $H$가 아니고 깡통 안에 남은 콜라의 질량 $m$도 상수가 아니다. 콜라가 
  가득 차 있을 때의 질량을 $m_0$,
  콜라의 밀도를 $\rho$라고 하면,
  \begin{align}
    \begin{split}
      \rho = \frac{m_0}{\pi R^2 H}
    \end{split}
  \end{align}
  이고 깡통 안에 남은 콜라의 질량 $m$은,
  \begin{align}\label{eq:1-1}
    m= \rho\pi R^2 x = \frac{m_0x}{H}
  \end{align}
  이다.
  전체 질량 중심의 높이 $h$는,
  \begin{align}
    \begin{split}
      h &= \frac{1}{M+m}\left(\frac{M}{A}\int z\,dA
      +\frac{M}{V}\int z\,dV\right) \\
%      m = 0.140\,\mathrm{kg} = \left(1.40\times 10^{3}\,\mathrm{g}\right)
%      H = \left(12\,\mathrm{cm}\right) 
%      M = \left(0.210\,\mathrm{g}\right)
      &= \frac{1}{M+m}\left(\frac{M}{2\pi R H}\int_0^H 2\pi R z\,dz
      +\frac{m}{\pi R^2 x}\int_0^x\pi R^2 z\,dz\right)  \\
      &= \frac{1}{M+m}\left(\frac{M}{H}\int_0^H z\,dz
      +\frac{m}{x}\int_0^x z\,dz\right) \\
      &= \frac{1}{M+m}\left(\frac{MH}{2}
      +\frac{mx}{2}\right) = \frac{MH+mx}{2(M+m)}
    \end{split}
  \end{align}
  식 (\ref{eq:1-1})에 의해,
  \begin{align}\label{eq:1-2}
    \begin{split}
      h = \frac{MH+mx}{2(M+m)}=\frac{MH^2+m_0x^2}{2(MH+m_0x)}.
    \end{split}
  \end{align}
  이다. $x=H$일 때와 $x=0$일 때 $h=\frac{1}{2}H$임을 확인할 수 있다.
  $M=1.40\times 10^{3}\,\mathrm{g}$, $H=12\,\mathrm{cm}$,
  $m_0=2.10\times 10^3\,\mathrm{g}$일 때의 그래프는 다음과 같다.
  \begin{figure}[htbp]
    \centering
    \includegraphics{.pdf}
    \caption{$x$에 따른 $h$의 그래프}
    \label{<label>}
  \end{figure}
  \item[(라)] 질량 중심이 가장 낮은 점에 있을 때를 찾기 위해 식(\ref{eq:1-2})을 
  $x$에 대해 미분하여 0이 되도록 하는 $x$를 찾자. 즉,
  \begin{align}
    \frac{dh}{dx} = \frac{MH^2m_0-2MHm_0x-m_0^2x^2}{2(MH+m_0x)^2} = 0
  \end{align}
  을 만족하는 $x$를 구해야 한다. 이는 $x$에 대한 2차 방정식인,
  \begin{align}
    m_0^2x^2 +2MHm_0x -MH^2m_0 = 0 
  \end{align}
  을 푸는 것과 같다. 근의 공식을 이용하면,
  \begin{align}
    x = \frac{-MHm_0\pm\sqrt{M^2H^2m_0^2+MH^2m_0^3}}{m_0^2}
    = \frac{-MH\pm H\sqrt{M(M+m_0)}}{m_0}.
  \end{align}
  $M=1.40\times 10^{3}\,\mathrm{g}$, $H=12\,\mathrm{cm}$,
  $m_0=2.10\times 10^3\,\mathrm{g}$ 이므로,
  \begin{align}
    \begin{split}
      x_1 =& \frac{-(1.40\times 10^{3}\,\mathrm{g})(12\,\mathrm{cm}) 
      - (12\,\mathrm{cm})\sqrt{(1.40\times 10^{3}\,\mathrm{g})
      ((1.40\times 10^{3}\,\mathrm{g})
      +2.10\times 10^3\,\mathrm{g})}}{2.10\times 10^3\,\mathrm{g}}
      =-21\,\mathrm{cm}  \\
      x_2 =& \frac{-(1.40\times 10^{3}\,\mathrm{g})(12\,\mathrm{cm}) 
      + (12\,\mathrm{cm})\sqrt{(1.40\times 10^{3}\,\mathrm{g})
      ((1.40\times 10^{3}\,\mathrm{g})
      +2.10\times 10^3\,\mathrm{g})}}{2.10\times 10^3\,\mathrm{g}}
      =4.6\,\mathrm{cm}
    \end{split}
  \end{align}
  여기서 $x_1<0$은 우리가 원하는 길이가 아니다. 따라서 답은,
  \begin{align}
    x = 4.6\,\mathrm{cm}
  \end{align}
  이다.
\end{itemize}

\vspace{1cm}

\noindent {\bf 문제 2. (50 pt)} 

\noindent {\bf 풀이 : } 

\vspace{1cm}

\noindent {\bf 문제 3. (20 pt)}

\noindent {\bf 풀이 : } 

\vspace{1cm}

\noindent {\bf 문제 4. (20 pt)}

\noindent {\bf 풀이 : } 
\end{document}