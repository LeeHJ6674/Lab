\documentclass[floatfix,nofootinbib,superscriptaddress,fleqn,preprint]{revtex4-2}  
%\documentclass[aps,epsfig,tightlines,fleqn]{revtex4}
\usepackage[utf]{kotex}
\usepackage[HWP]{dhucs-interword}
\usepackage[dvips]{color}
\usepackage{graphicx}
\usepackage{bm}
%\usepackage{fancyhdr}
%\usepackage{dcolumn}
\usepackage{defcolor}
\usepackage{amsmath}
\usepackage{amsfonts}
\usepackage{amssymb}
\usepackage{amscd}
\usepackage{amsthm}
\usepackage[utf8]{inputenc}
 \usepackage{setspace}
%\pagestyle{fancy}

\begin{document}

\title{\Large 2022년 1학기 물리학 I: Quiz 19}
\author{김현철\footnote{Office: 5S-436D (면담시간 매주
    화요일-16:00$\sim$20:00)}} 
\email{hchkim@inha.ac.kr}
\affiliation{Hadron Theory Group, Department of Physics,
Inha University, Incheon 22212, Republic of Korea }
\date{Spring semester, 2022}


\vspace{1.cm}
\begin{abstract}
\noindent \textbf{ {\color{red}주의}: \color{blue} 단 한 번의 부정행위도 절대
  용납하지 않습니다. 적발 시, 학점은 F를 받게 됨은 물론이고,
  징계위원회에 회부합니다. One strike out임을 명심하세요.}\\
\\
문제는 다음 쪽부터 나옵니다.  \\ \\
{\bf Date:} 2022년 5월 16일 (월) 15:30-16:15
\\
{\bf 학번:} \hspace{4cm}
{\bf 이름:} 

\end{abstract}
\maketitle

\newpage
\noindent {\bf 문제 1. (60 pt)} 
아래의 식과 같이 파동이 주어져 있다. 
\begin{align}
  \label{eq:2}
\psi(x,t) = 0.00327 \sin(72.1x-2.72 t).  
\end{align}
\begin{itemize}
\item[(가)] 파동의 진폭을 구하여라.
\item[(나)] 파동의 파장, 주기, 진동수를 구하여라.
\item[(다)] 파동의 속력은 얼마인가?
\end{itemize}

 \newpage

{\color{gray} [문제 풀이 쪽]}

\newpage

\noindent {\bf 문제 2. (150 pt) ({\color{red} 난이도 상})}
각진동수 1200 rad/s와 진폭이 3.00 mm인 파동을 선밀도가 2.00 g/m이고,
장력이 1200 N인 줄에 보냈다.
\begin{itemize}
\item[(가)] 줄의 반대끝으로 파동이 전달하는 에너지의 평균 전달율은
  얼마인가?
\item[(나)] 만약에 똑같은 다른 파동이 인접한 같은 종류의 줄을 따라
  동시에 진행한다면 전달되는 총 에너지의 평균 전달율은 얼마인가? 만약
  하나의 같은 줄에 두 파동을 동시에 보낸다면 위상차가
\item[(다)] 0
\item[(라)] 0.4 rad
\item[(마)] $\pi$ rad일 때, 파동이 전달하는 평균에너지 전달률은 각각
  얼마인가?   
\end{itemize}
\newpage
{\color{gray} [문제 풀이 쪽]}

\newpage

\noindent {\bf 문제 3. (60pt)  }
한쪽 끝은 $x=0$, 다른 끝은 $x=10.0$ m에 매어져 있는 질량 100 g의 줄에
250 N의 장력이 작용한다. $t=0$일 때 $x=10.0$ m인 끝점에서 펄스 1을
줄을 따라 보내고, $t=30.0$ ms일 때 $x=0$인 끝점에서 펄스 2를
보냈다. 어떤 점 $x$에서 두 펄스가 만나겠는가? 
\newpage
{\color{gray} [문제 풀이 쪽]}

\newpage

\noindent {\bf 문제 4. (80pt)}
아래 식으로 주어진 파동의 속력을 구하여라.
\begin{align}
  \label{eq:4}
\psi(x,t) = (2.00\,\mathrm{mm}) \sqrt{(20\,\mathrm{m}^{-1}) x - (4.0
  \,\mathrm{s}^{-1})t} .
\end{align}
\newpage
{\color{gray} [문제 풀이 쪽]}

\newpage
\end{document}