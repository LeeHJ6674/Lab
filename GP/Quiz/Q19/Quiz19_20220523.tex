\documentclass[floatfix,nofootinbib,superscriptaddress,fleqn]{revtex4-2}  
%\documentclass[aps,epsfig,tightlines,fleqn]{revtex4}
\usepackage[utf]{kotex}
\usepackage[HWP]{dhucs-interword}
\usepackage[dvips]{color}
\usepackage{graphicx}
\usepackage{bm}
%\usepackage{fancyhdr}
%\usepackage{dcolumn}
\usepackage{defcolor}
\usepackage{amsmath}
\usepackage{amsfonts}
\usepackage{amssymb}
\usepackage{amscd}
\usepackage{amsthm}
\usepackage[utf8]{inputenc}
 \usepackage{setspace}
%\pagestyle{fancy}

\begin{document}

\title{\Large 2022년 1학기 물리학 I: Quiz 19}
\author{김현철\footnote{Office: 5S-436D (면담시간 매주
    화요일-16:00$\sim$20:00)}} 
\email{hchkim@inha.ac.kr}
\author{Lee Hui-Jae} 
\email{hjlee6674@inha.edu}
\affiliation{Hadron Theory Group, Department of Physics,
Inha University, Incheon 22212, Republic of Korea }
\date{Spring semester, 2022}


\vspace{1.cm}

\maketitle

\vspace{1.cm}
\noindent {\bf 문제 1. (60 pt)} 
아래의 식과 같이 파동이 주어져 있다. 
\begin{align}
  \label{eq:2}
\psi(x,t) = 0.00327 \sin(72.1x-2.72 t).  
\end{align}
\begin{itemize}
\item[(가)] 파동의 진폭을 구하여라.
\item[(나)] 파동의 파장, 주기, 진동수를 구하여라.
\item[(다)] 파동의 속력은 얼마인가?
\end{itemize}
\vspace{1.cm}
\noindent {\bf 풀이 :   }
파동의 진폭을 $A$, 파수를 $k$, 각속도를 $\omega$라고 한다면 파동함수를
\begin{align}
  \psi(x,t) = A \sin(k(x-v t))   = A \sin(kx-\omega t)   
\end{align}
와 같이 나타낼 수 있다.
\begin{itemize}
  \item[(가)] 
  파동의 진폭 $A$는 $ 0.00327$이다.
  \item[(나)]
  파장은 파수로부터, 주기는 각속도로부터, 진동수는 주기로부터 구할 수 있다.
  파동의 파장 $\lambda$는 
  \begin{align}
    \begin{split}
      \lambda &= \frac{2\pi}{k} = \frac{2\pi}{72.1} 
      =0.0871
    \end{split}
  \end{align}
  이고 파동의 주기 $T$는 
  \begin{align}
    \begin{split}
      T = \frac{2\pi}{vk} = \frac{2\pi}{\omega}
      = \frac{2\pi}{2.72} = 2.31
    \end{split}
  \end{align}
  이다. 주기 $T$를 이용해 진동수 $f$를 구할 수 있다.
  \begin{align}
    f = \frac{1}{T} = 0.433.
  \end{align}
  \item[(다)]
  파동의 속력 $v$는
  \begin{align}
    v = \frac{\omega}{k} = \frac{2.72}{72.1} = 0.0377
  \end{align}
  이다.
\end{itemize}

\vspace{1.cm}

\noindent {\bf 문제 2. (150 pt) ({\color{red} 난이도 상})}
각진동수 1200 rad/s와 진폭이 3.00 mm인 파동을 선밀도가 2.00 g/m이고,
장력이 1200 N인 줄에 보냈다.
\begin{itemize}
\item[(가)] 줄의 반대끝으로 파동이 전달하는 에너지의 평균 전달률은
  얼마인가?
\item[(나)] 만약에 똑같은 다른 파동이 인접한 같은 종류의 줄을 따라
  동시에 진행한다면 전달되는 총 에너지의 평균 전달률은 얼마인가? 만약
  하나의 같은 줄에 두 파동을 동시에 보낸다면 위상차가
\item[(다)] 0
\item[(라)] 0.4 rad
\item[(마)] $\pi$ rad일 때, 파동이 전달하는 평균에너지 전달률은 각각
  얼마인가?   
\end{itemize}
\vspace{1.cm}
\noindent {\bf 풀이 :   }
\begin{itemize}
  \item[(가)]
  에너지의 평균 전달률 $P$는 다음과 같이 주어진다.
  \begin{align}
    P = \frac{1}{2}\mu v \omega^2y_m^2.
  \end{align}
  $\mu$는 선밀도, $v$는 파동의 속력, $\omega$는 각진동수이고 $y_m$은 진폭이다.
  줄의 장력을 $F$라 하면 파동의 속력 $v$는
  \begin{align}
    v = \sqrt{\frac{F}{\mu}}
  \end{align}
  이므로 $P$는
  \begin{align}
    \begin{split}
      P &= \frac{1}{2}\sqrt{\mu F}\omega^2y^2_m
      = \frac{1}{2}\sqrt{(2.00\times 10^{-3}\,\mathrm{kg/m})(1\,200\,\mathrm{N})}
      (1\,200\,\mathrm{rad/s})^2(3.00\times10^{-3}\,\mathrm{m})^2 \\
      &=10.0\,\mathrm{W}
    \end{split}
  \end{align}
  이다.
  \item[(나)]
  인접한 같은 종류의 줄을 따라 동시에 파동이 진행한다면 전달되는 총 에너지의 평균 전달률은 파동 하나의 평균
  전달률의 2배이다. 따라서 이 경우 평균 전달률은 20 W이다.
  \item[(다)]
  같은 줄을 통해 보낸 두 파동의 위상차가 0 이면 두 파동은 보강간섭을 일으킨다. 따라서 전달되는 파동의 진폭이 2배가 되고
  에너지의 평균 전달률을 4배가 된다. 즉, 위상차가 0인 경우 평균에너지 전달률은 10 W이다.
  \item[(라)]
  두 파동의 위상차를 $\phi$라 하자. 진폭이 같은 두 파동의 중첩은 다음과 같이 표현할 수 있다.
  \begin{align}\label{eq:2-1}
    \begin{split}
      y_1(x,t)+y_2(x,t) &= A\sin{(kx-\omega t)}+A\sin{(kx-\omega t+\phi)}  \\
      &=2A\cos{\frac{1}{2}\phi}\sin{\left(kx-\omega t+\frac{1}{2}\phi\right)}.
    \end{split}
  \end{align}
  이를 통해 중첩된 파동의 진폭은 $2A\cos{\frac{1}{2}\phi}$임을 알 수 있다. $\phi = $ 0.4 rad이면,
  새로운 진폭 $y'_m$은
  \begin{align}
    y'_m = 2y_m\cos(0.2\,\mathrm{rad})
  \end{align}
  이고 새로운 진폭 $y'_m$에 대한 전달률 $P$는
  \begin{align}
    \begin{split}
      P &= \frac{1}{2}\sqrt{\mu F}\omega^2y'^2_m
      = \frac{1}{2}\sqrt{(2.00\times 10^{-3}\,\mathrm{kg/m})(1\,200\,\mathrm{N})}
      (1\,200\,\mathrm{rad/s})^2
      4(3.00\times10^{-3}\,\mathrm{m})^2
      (\cos(0.2\,\mathrm{rad}))^2 \\
      &=39\,\mathrm{W}
    \end{split}
  \end{align}
이다.

  \item[(마)]
  위상차가 $\pi$만큼 난다면 식~\eqref{eq:2-1}에 의해
  \begin{align}
    y'_m = 2y_m\cos\frac{1}{2}\pi = 0
  \end{align}
  이다. 따라서 전달률 또한 0이다.
\end{itemize}
\vspace{1.cm}

\noindent {\bf 문제 3. (60pt)  }
한쪽 끝은 $x=0$, 다른 끝은 $x=10.0$ m에 매어져 있는 질량 100 g의 줄에
250 N의 장력이 작용한다. $t=0$일 때 $x=10.0$ m인 끝점에서 펄스 1을
줄을 따라 보내고, $t=30.0$ ms일 때 $x=0$인 끝점에서 펄스 2를
보냈다. 어떤 점 $x$에서 두 펄스가 만나겠는가? 

\vspace{1.cm}
\noindent {\bf 풀이 :   }
파동의 속력 $v$는 다음과 같이 주어진다.
\begin{align}\label{eq:3-1}
  v = \sqrt{\frac{F}{\mu}}=\sqrt{\frac{Fl}{m}}.
\end{align}
$l$은 줄의 길이이고 $m$은 줄의 질량이다.
만나는 지점을 $d$라 하면 펄스 1이 움직인 거리는 $l-d$, 펄스 2가 움직인 거리는 $d$이다.
같은 줄에서 움직이기 때문에 두 펄스의 속력은 같고 펄스 1이 움직인 시간이 펄스 2보다 30.0 ms만큼 
더 기므로
\begin{align}
  \frac{l-d}{v}-\frac{d}{v} = 3.00\times 10^{-2}\,\mathrm{s}
  \Longrightarrow d=\frac{1}{2}(l-(3.00\times 10^{-2}\,\mathrm{s})v) 
\end{align}
이다. 식~\eqref{eq:3-1}을 대입하여 $d$를 구할 수 있다.
\begin{align}
  \begin{split}
    d &=\frac{1}{2}\left(l-(3.00\times 10^{-2}\,\mathrm{s})\sqrt{\frac{Fl}{m}}\right) 
    =\frac{1}{2}\left((10.0\,\mathrm{m})-(3.00\times 10^{-2}\,\mathrm{s})
    \sqrt{\frac{(250\,\mathrm{N})(10.0\,\mathrm{m})}{(0.100\,\mathrm{kg})}}\right)  \\
    &= 2.63\,\mathrm{m}.
  \end{split}
\end{align}
펄스가 만나는 지점은 $x=0$인 점으로부터 2.63 m만큼 떨어진 곳이다.

\vspace{1.cm}

\noindent {\bf 문제 4. (80pt)}
아래 식으로 주어진 파동의 속력을 구하여라.
\begin{align}
  \label{eq:4}
\psi(x,t) = (2.00\,\mathrm{mm}) \sqrt{(20\,\mathrm{m}^{-1}) x - (4.0
  \,\mathrm{s}^{-1})t} .
\end{align}

\vspace{1.cm}
\noindent {\bf 풀이 :   }
파동방정식은 다음과 같이 주어진다.
\begin{align}
  \frac{\partial^2 \psi}{\partial x^2} =\frac{1}{v^2}\frac{\partial^2 \psi}{\partial t^2}.
\end{align}
이를 이용해서 파동의 속력을 구해보자. 
$\psi = A\sqrt{kx-\omega t}$라 하고
$\psi$를 $x$와 $t$에 대해 각각 두번씩 편미분히면,
\begin{align}
  \frac{\partial^2 \psi}{\partial x^2}= -\frac{1}{4}Ak^2(kx-\omega t)^{-3/2},\,\,\,
  \frac{\partial^2 \psi}{\partial t^2}= -\frac{1}{4}A\omega^2(kx-\omega t)^{-3/2}
\end{align}
이다. 따라서 속력 $v$는
\begin{align}
  v = \frac{\omega}{k} = \frac{4.0\,\mathrm{s}^{-1}}{20\,\mathrm{m}^{-1}}
  = 0.2\,\mathrm{m/s}
\end{align}
이다.
\vspace{1.cm}
\end{document}