\documentclass[floatfix,nofootinbib,superscriptaddress,fleqn]{revtex4-2}  
%\documentclass[aps,epsfig,tightlines,fleqn]{revtex4}
\usepackage[utf]{kotex}
\usepackage[HWP]{dhucs-interword}
\usepackage[dvips]{color}
\usepackage{graphicx}
\usepackage{bm}
%\usepackage{fancyhdr}
%\usepackage{dcolumn}
\usepackage{defcolor}
\usepackage{amsmath}
\usepackage{amsfonts}
\usepackage{amssymb}
\usepackage{amscd}
\usepackage{amsthm}
\usepackage[utf8]{inputenc}
 \usepackage{setspace}
%\pagestyle{fancy}

\begin{document}

\title{\Large 2022년 1학기 물리학 I: Quiz 19}
\author{김현철\footnote{Office: 5S-436D (면담시간 매주
    화요일-16:00$\sim$20:00)}} 
\email{hchkim@inha.ac.kr}
\author{Lee Hui-Jae} 
\email{hjlee6674@inha.edu}
\affiliation{Hadron Theory Group, Department of Physics,
Inha University, Incheon 22212, Republic of Korea }
\date{Spring semester, 2022}


\vspace{1.cm}

\maketitle


%%%%%%%%%%%%%%%%%%%%%%%%%%%%%%%%%%%%%%%%%%%%%%%%%%%%%%%%%%%%%%%%%%%%%%%%%%%%%%5
\noindent {\bf 풀이 :   }
온도가 $20\,\mathrm{^\circ C}$일 때 강철자의 길이를 $L_{st}$,
물체의 길이를 $L'$이라 하자.
선팽창계수를 고려하면 강철자가 늘어난 길이 $\Delta L_{st}$와
물체가 늘어난 길이 $\Delta L'$은
\begin{align}\label{eq:1-1}
  \Delta L_{st} =L_{st}\alpha_{st}\Delta T,\,\,\,
  \Delta L' =L'\alpha'\Delta T
\end{align}
이다. $\alpha_{st}$와 $\alpha'$은 각각 강철과 물체의 선팽창계수이다. 
강철자의 길이가 $20\,\mathrm{^\circ C}$에서 1 cm라고 생각해보자. 
오븐의 넣기 전 강철자로 잰 막대의 길이가 20.05 cm이므로
\begin{align}\label{eq:1-2}
  \frac{L'}{L_{st}} = \frac{20.05\,\mathrm{cm}}{1\,\mathrm{cm}}
  =20.05 \Longrightarrow L' = 20.05 L_{st}
\end{align}
이고 오븐에 넣고서 강철자로 잰 막대의 길이가 20.11 cm이므로
\begin{align}\label{eq:1-3}
  \frac{L'+\Delta L'}{L_{st}+\Delta L_{st}} = 20.11
  \Longrightarrow 
  L'+\Delta L' = (20.11)(L_{st}+\Delta L_{st})
\end{align}
이다. 식~\eqref{eq:1-1}과~\eqref{eq:1-2}를 식~\eqref{eq:1-3}에 대입하면
다음과 같다.
\begin{align}
  \begin{split}
    &L'(1+\alpha'\Delta T) = L_{st}(20.05)(1+\alpha'\Delta T)
    = L_{st}(20.11)(1+\alpha_{st}\Delta T)  \\
    &\Longrightarrow \alpha' = \frac{1}{\Delta T}\left(\frac{20.11}{20.05}
    (1+\alpha_{st}\Delta T)-1\right).
  \end{split}
\end{align}
따라서 물체의 열팽창계수 $\alpha'$는
\begin{align}
  \begin{split}
    \alpha' &= \frac{1}{250\,\mathrm{^\circ C}}\left(\frac{20.11}{20.05}
    \left(1+(11\times 10^{-6}\,\mathrm{^\circ C^{-1}})
    (250\,\mathrm{^\circ C})\right)-1\right)  \\
    &=2.3\times 10^{-5}\,\mathrm{^\circ C^{-1}}
  \end{split}
\end{align}
이다.

\vspace{1.cm}

%%%%%%%%%%%%%%%%%%%%%%%%%%%%%%%%%%%%%%%%%%%%%%%%%%%%%%%%%%%%%%%%%%%%%%%%%%%%%%5
\noindent {\bf 풀이 :   }
비열이 $c$이고 질량이 $m$인 물질의 온도를 아주 작은 $dT$만큼 올리는데 필요한 에너지 $dQ$는
\begin{align}
 dQ = cm\,d T
\end{align}
이다. 이 물질의 온도를 $T_1$에서 $T_2$까지 가열한다고 할 때 필요한 에너지 $Q$는
\begin{align}
  \begin{split}
    Q &= m\int^{T_2}_{T_1}c\,dT
    = m \int^{T_2}_{T_1}0.20 +0.14T+0.023T^2\,dT  \\
    &= m\left( 0.20\left(T_2-T_1\right)+0.07\left(T_2^2-T_1^2\right)
    +\frac{0.023}{3}\left(T_2^3-T_1^3\right) \right)
  \end{split}
\end{align}
이다. $m$=2.0 g, $T_1$=5.0 $\mathrm{^\circ C}$이고 $T_2$=15 $\mathrm{^\circ C}$이므로
에너지 $Q$를 다음과 같이 얻는다.
\begin{align}
  \begin{split}
    Q &= (2.0\,\mathrm{g})\left(  
      (0.20)(10\,\mathrm{^\circ C})+(0.07)\left(
        (15\,\mathrm{^\circ C})^2-  
      (5.0\,\mathrm{^\circ C})^2\right)
      +\frac{0.023}{3}\left((15\,\mathrm{^\circ C})^3-  
      (5.0\,\mathrm{^\circ C})^3\right)
      \right) \\
      &= 82\,\mathrm{cal}.
    \end{split}
\end{align}
\vspace{1.cm}

%%%%%%%%%%%%%%%%%%%%%%%%%%%%%%%%%%%%%%%%%%%%%%%%%%%%%%%%%%%%%%%%%%%%%%%%%%%%%%5
\noindent {\bf 풀이 :   }
물과 얼음의 처음 온도를 $T_w$, $T_i$라 하고
물이 얼음에 전달한 열을 $Q_{w}$, 얼음이 흡수한 열을 $Q_{i}$라고 하자.
$Q_{w}$과 $Q_{i}$는
\begin{align}
  \begin{split}
    Q_{w}  = c_{w}m_{w}\Delta T_w ,\,\,\,Q_{i} = c_{i}m_{i}\Delta T_i
  \end{split}
\end{align}
이다. 열평형일 때 온도를 $T_e$라고 하면




\vspace{1.cm}
%%%%%%%%%%%%%%%%%%%%%%%%%%%%%%%%%%%%%%%%%%%%%%%%%%%%%%%%%%%%%%%%%%%%%%%%%%%%%%5
\noindent {\bf 풀이 :   }
열역학 제 1법칙은 다음과 같다.
\begin{align}\label{eq:4-1}
  dU = dQ -PdV.
\end{align}
\begin{itemize}
  \item[(가)]
  경로 $ab$는 등압과정, 경로 $ca$는 등적과정이다. 경로 $ab$를 거치면서 내부에너지가 
+3.0 J만큼 변하였고 외부에 5.0 J만큼 일을 해주었다. 식~\eqref{eq:4-1}에 의해 경로 $ab$를
다라가는 동안 기체에 전달된 열의 크기 $Q$는
\begin{align}
  Q = 3.0\,\mathrm{J} + 5.0\,\mathrm{J} 
  =8.0\,\mathrm{J}
\end{align}
이다. 
  \item[(나)] 전체 과정을 한번 순환하면 기체의 내부에너지의 변화는 없으므로  
  순환경로 $abca$를 한번 따라가면서 기체에 전달된 열에너지는 모두 기체가 외부에
  한 일로 변환된다. 기체에 전달된 열에너지의 합은
  \begin{align}
    Q_{ab}+Q_{bc}+Q_{ca} = 8.0\,\mathrm{J} +Q_{bc}+2.5\,\mathrm{J}
  \end{align}
  이고 기체가 외부에 한 일의 합은
  \begin{align}
    W_{ab}+W_{bc}+W_{ca} = 1.2\,\mathrm{J}
  \end{align}
  이다. 따라서 경로 $bc$를 따라갈 때 기체에 전달된 열에너지 $Q_{bc}$는
  \begin{align}
    Q_{bc} = 1.2\,\mathrm{J} - (8.0\,\mathrm{J} +2.5\,\mathrm{J})
    =-9.3\,\mathrm{J}
  \end{align}
  이다. $-$부호는 기체로부터 열에너지가 빠져나갔음을 의미한다.
\end{itemize}

\end{document}