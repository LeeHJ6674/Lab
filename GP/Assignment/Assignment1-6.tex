%\documentclass[preprint,tightenlines,showpacs,showkeys,floatfix,
%nofootinbib,superscriptaddress,fleqn]{revtex4} 
\documentclass[floatfix,nofootinbib,superscriptaddress,fleqn,preprint]{revtex4} 
%\documentclass[aps,epsfig,tightlines,fleqn]{revtex4}
\usepackage[utf]{kotex}
\usepackage[HWP]{dhucs-interword}
\usepackage[dvips]{color}
\usepackage{graphicx}
\usepackage{bm}
%\usepackage{fancyhdr}
%\usepackage{dcolumn}
\usepackage{defcolor}
\usepackage{amsmath}
\usepackage{amsfonts}
\usepackage{amssymb}
\usepackage{amscd}
\usepackage{amsthm}
\usepackage[utf8]{inputenc}
 \usepackage{setspace}
%\pagestyle{fancy}

\begin{document}

% \title{\Large 2021년 1학기 고전역학 I: Quiz 5}
% \author{김현철\footnote{Office: 5S-436D (면담시간 매주
%     화요일-16:00$\sim$20:00)}} 
% \email{hchkim@inha.ac.kr}
% \affiliation{Hadron Theory Group, Department of Physics, Inha
%   University, Incheon 402-751, Republic of Korea }
% \date{Spring semester, 2021}


% \vspace{1.cm}
% \begin{abstract}
% \noindent \textbf{ {\color{red}주의}: \color{blue} 단 한 번의 부정행위도 절대
%   용납하지 않습니다. 적발 시, 학점은 F를 받게 됨은 물론이고,
%   징계위원회에 회부합니다. One strike out임을 명심하세요.}\\
% \\
% 문제는 다음 쪽부터 나옵니다.  \\ \\
% {\bf Date:} 2021년 3월 16일 (화) 12:25-13:15 
% \\
% {\bf 학번:} \hspace{4cm}
% {\bf 이름:} 

% \end{abstract}
% \maketitle

\noindent {\bf 문제 6.} 

\vspace{1 cm}
\noindent{\bf 답:} 
\begin{itemize}
  \item[(a)] 벡터 $\vec{a}$ 와 $x$ 축이 이루는 각도를 $\phi$ 라고 하면 각 성분은 다음과 같다.
  \begin{align}
    a_x = |a|\cos{\phi},\,\,\,a_y = |a|\sin{\phi}
  \end{align}
  $|a|=3.00\,\mathrm{m}$ 이고 $\phi = 0^\circ$ 이므로,
  \begin{align}
    \begin{split}
    &a_x = (3.00\,\mathrm{m})(\cos{0^\circ}) =3.00\,\mathrm{m}   \\
    &a_y = (3.00\,\mathrm{m})(\sin{0^\circ})=0\,\mathrm{m}.
    \end{split}
  \end{align}
  \item[(b)] 벡터 $\vec{b}$ 의 각 성분은 다음과 같다.
  \begin{align}
    b_x = |b|\cos{\theta},\,\,\,b_y = |b|\sin{\theta}
  \end{align}
  $|b|=4.00\,\mathrm{m}$ 이고 $\theta = 30^\circ$ 이므로,
  \begin{align}
    \begin{split}
      &b_x = (4.00\,\mathrm{m})(\cos{30^\circ}) 
      =(4.00\,\mathrm{m})\left(\frac{\sqrt{3}}{2}\right)
      =(2.00\times\sqrt{3})\,\mathrm{m}\\
      &b_y = (4.00\,\mathrm{m})(\sin{30^\circ})
      =(4.00\,\mathrm{m})\left(\frac{1}{2}\right)
      =2.00\,\mathrm{m}.
    \end{split}
  \end{align}
  \item[(c)] 벡터 $\vec{c}$ 와 $x$ 축이 이루는 각도를 $\psi$ 라고 하면,
  \begin{align}
    \psi = \theta + 90^{\circ}.
  \end{align}
  따라서, 벡터 $\vec{c}$ 의 각 성분은 다음과 같다.
  \begin{align}
    c_x = |c|\cos{(\theta + 90^{\circ})},\,\,\,c_y = |c|\sin{(\theta + 90^{\circ})}
  \end{align}
  $|b|=4.00\,\mathrm{m}$ 이고 $\theta = 30^\circ$ 이므로,
  \item[(d)] 
\end{itemize}

\end{document}  