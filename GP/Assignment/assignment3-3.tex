\documentclass[fleqn,10pt]{article}%
\usepackage[hangul,nonfrench,finemath]{kotex}
\usepackage[HWP]{dhucs-interword}
\usehangulfontspec{ut}
\usepackage[hangul]{dhucs-setspace}
\usepackage{ifpdf}
\ifpdf
\usepackage[unicode,pdftex,colorlinks]{hyperref}
\input glyphtounicode\pdfgentounicode=1
\else
\usepackage[unicode,dvipdfm,colorlinks]{hyperref}
%\pdfmapfile{+unttf-pdftex-dhucs.map}
\fi 

\usepackage[utf8]{inputenc}
\usepackage[dvips]{color}
\usepackage{defcolor}
\usepackage{bm}
\usepackage{amsmath}
\usepackage{amsfonts}
\usepackage{amssymb}
\usepackage{graphicx}%
\usepackage{tikz}
\setlength{\topmargin}{-1.0in}
\setlength{\textheight}{9.25in}
\setlength{\oddsidemargin}{0.0in}
\setlength{\evensidemargin}{0.0in}
\setlength{\textwidth}{6.5in}
\pagestyle{plain}
\setcounter{secnumdepth}{0}
\parindent=0pt
\begin{document}

\begin{center}
\textbf{\Large 물리학 I 2022년 1학기}
\vspace{0.5cm}

\textbf{숙제 3: 2차원 운동, 뉴턴의 운동법칙 (담당교수: 김현철)}
\end{center}
\vspace{0.8cm}

\textbf{\color{red}제출기한: 2022년 3월 23일 수요일 오후 15:00시까지}
\vspace{1.0cm}


%----------------------------------------------------------
%----------------------------------------------------------

\noindent {\bf 문제 3. [20pt]} 질량이 $m_1$인 공과 질량이 $m_2$인 상자가
질량을 무시해도 될만큼 가벼운 줄로 마찰이 없는 도르래에 걸린 채
연결되어 있다(그림~\ref{fig:3}). 질량 $m_2$인 상자는 평면과
$\theta$각을 이루고 있는 비탈면에 놓여있다 (비탈면과 $m_2$ 상자
사이에는 마찰이 없다고 가정하라).
\begin{itemize}
\item[(1)] 이 두 물체의 가속도의 크기를 각각 구하고 줄에 걸리는 장력을
  구하라.    
\item[(2)] 빗면의 길이를 $l$이라고 하자. 물체 $m_2$가 최초에 빗면의 맨
  위 끝에 놓여 있었다면, 빗면을 다 내려오는 데 걸린 시간은 얼마인가? 
\end{itemize}
 
\begin{figure}[h]
  \centering
%\includegraphics[scale=0.6]{hw4_1.pdf}
  \caption{문제 3}
\label{fig:3}  
\end{figure}

\noindent {\bf 풀이 : }
\begin{itemize}
  \item[(1)] 두 물체의 자유 물체 다이어 그램을 그리고 각 물체에 대한 운동 방정식을 세우자.
  \begin{figure}[h]
    \centering
    \begin{tikzpicture}
      \draw node at (-2,3) {물체 $m_1$ : }  ;
      \draw (-2,0) -- (2,0) ;
      \draw (0,-2) -- (0,2) ;
      \draw [red,very thick,-latex] (0,0.1) -- (0,1.5) 
      node [left,black] {$T$};

      \draw [blue,very thick,-latex] (0,-0.1) -- (0,-1) 
      node [right,black] {$F_{g1}$};
    \end{tikzpicture}
    \hspace{1cm}
    \begin{tikzpicture}%\usepackage{tikz}
      \draw node at (-2,3) {물체 $m_2$ : }  ;
      \draw[rotate=-32] (-2.5,0) -- (2.5,0) ;
      \draw[rotate=-32] (0,-2) -- (0,2) ;
      \draw [rotate=-32,red,very thick,-latex] (-0.1,0) -- (-1.5,0) 
      node [below,black] {$T$};
      \draw [rotate=-32,red,very thick,-latex] (0,0.1) -- (0,1.5) 
      node [left,black] {$N$};
      \draw[black] (238:1) arc(232:270:0.9)
      node [below=4,left,black] {$\theta$} ;
      \draw [blue,very thick,-latex] (0,-0.1) -- (0,-2) 
      node [right,black] {$F_{g2}$};
    \end{tikzpicture}

 \caption{자유 물체 다이어 그램}
  \end{figure}

  물체 $m_1$ 에 대한 운동 방정식은,
  \begin{align}
    \begin{split}
      \sum F_x &= 0  \\
      \sum F_y &= T-F_{g1} = m_1 a,
    \end{split}
  \end{align}
  이다. 따라서 물체 $m_1$ 의 가속도의 크기는,
  \begin{align}\label{eq:3-1}
  a=\frac{T-F_{g1}}{m_1}  
  \end{align} 
  이다.
  물체 $m_2$ 에 대한 운동 방정식은,
  \begin{align}
    \begin{split}
      \sum F_x &= F_{g2}\sin{\theta}-T=m_2a  \\
      \sum F_y &= N - F_{g2}\cos{\theta} = 0,
    \end{split}
  \end{align}
  이다. 따라서 물체 $m_2$ 의 가속도의 크기는,
  \begin{align}
    a=\frac{F_{g2}\sin{\theta}-T}{m_2}  
  \end{align}
  이다. 두 물체는 줄에 의해 연결되어 있으므로 두 물체의 가속도는 같다는 사실을 이용하여 
  장력의 크기를 구할 수 있다.
  \begin{align}
    \frac{F_{g2}\sin{\theta}-T}{m_2} =\frac{T-F_{g1}}{m_1},\,\,\,
    T = \frac{m_2 F_{g1}+m_1 F_{g2}\sin{\theta}}{m_2+m_1}.
  \end{align}
  $F_{g1}=m_1g$, $F_{g2}=m_2g$ 이므로 줄에 걸리는 장력은,
  \begin{align}\label{eq:3-2}
    T = \frac{m_2 m_1g+m_1 m_2g\sin{\theta}}{m_2+m_1} 
    =\frac{m_1 m_2 g(1+\sin{\theta})}{m_2+m_1}.
  \end{align}
  \item[(2)] 물체 $m_2$ 에 걸리는 가속도의 크기는 물체가 빗면을 끝까지 내려올 때 까지 
  일정하다. 식 (\ref{eq:3-1}) 과 식 (\ref{eq:3-2}) 에 의해,
  \begin{align}
  a=   \frac{m_2g(1+\sin{\theta})}{m_1+m_2}-g
  =\frac{(m_2\sin{\theta}-m_1)g}{m_1+m_2}.  
  \end{align}
  물체는 초기 속력이 0 인 등가속도 운동을 하므로 거리 $l$ 만큼 움직이는데 걸린 시간은 
  다음과 같다.
  \begin{align}
    l=\frac{1}{2}at^2,\,\,\,t= \sqrt{\frac{2l}{a}}
  \end{align}
  따라서 빗면을 다 내려오는데 걸린 시간은 다음과 같다.
  \begin{align}
    \begin{split}
      t&=\sqrt{\frac{2l}{a}}
      =\sqrt{\frac{2l(m_1+m_2)}{(m_2\sin{\theta}-m_1)g}} 
    \end{split}
  \end{align}
\end{itemize}
 
\newpage

%----------------------------------------------------------
%----------------------------------------------------------

\end{document}
