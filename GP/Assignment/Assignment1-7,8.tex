\documentclass[floatfix,nofootinbib,superscriptaddress,fleqn]{revtex4-2} 
%\documentclass[aps,epsfig,tightlines,fleqn]{revtex4}
\usepackage[utf]{kotex}
\usepackage[HWP]{dhucs-interword}
\usepackage[dvips]{color}
\usepackage{graphicx}
\usepackage{bm}
%\usepackage{fancyhdr}
%\usepackage{dcolumn}
\usepackage{defcolor}
\usepackage{amsmath}
\usepackage{amsfonts}
\usepackage{amssymb}
\usepackage{amscd}
\usepackage{amsthm}
\usepackage{subfig}
\usepackage[utf8]{inputenc}
 \usepackage{setspace}
%\pagestyle{fancy}

\begin{document}

\title{\Large 물리학 I (3, 2022) \\
 Assignment 1}

\maketitle

\noindent {\bf 문제 7 [10pt]:} 평면 위에 놓여있는 용수철의 왼쪽 끝은 벽에 고정되어 있고, 오른쪽 끝에는 질량이 m인 물체가 달려있다. 이 용수철의 탄성계수(또는 용수철 상수)는 $k$이다. 이 용수철은 후크의 법칙에 따라 $F=-kx$ 만큼의 힘을 받는다. 여기서 $x$는 용수철이 평형점에서부터 늘어난 변위를 뜻한다. 차원분석을 이용해서 이 용수철의 주기가 무엇에 비례하는지 보여라. 여기서 힘의 단위는 kg$\cdot$m/s$^2$


\vspace{0.5 cm}

\noindent{\bf 풀이 :} \\

\noindent 힘의 단위가 kg$\cdot$m/s$^2$이므로 용수철 상수의 단위는 $k=-F/x$임을 이용하면
\begin{align}
[k]  =\frac{[F]}{[x]}=\frac{\rm kg\cdot m/s^2}{\rm m}=\rm kg/s^2
\end{align}
임을 알 수 있으므로 따라서 주기 $T$는
\begin{align}
  [T]\propto \frac{[m^{1/2}]}{[k^{1/2}]}=\frac{\rm kg^{1/2}}{\rm kg^{1/2}/s}=\rm s.
\end{align}
따라서 주기는 
\begin{align}
  T\propto\sqrt{\frac{m}{k}}
\end{align}
와 같이 $\sqrt{m/k}$에 비례함을 알 수 있다.
(이 문제에서 [A]는 A의 단위를 의미한다.)\\

\vspace{0.5 cm}

\noindent {\bf 문제 8 [10pt]:} 다음과 같이 세 개의 벡터가 주어져 있다.

\begin{align}
  \vec{d}_1=-3.0\hat{\boldsymbol{i}}+3.0\hat{\boldsymbol{j}}+2.0\hat{\boldsymbol{k}}\cr
  \vec{d}_2=-2.0\hat{\boldsymbol{i}}-4.0\hat{\boldsymbol{j}}+2.0\hat{\boldsymbol{k}}\cr
  \vec{d}_3=2.0\hat{\boldsymbol{i}}+3.0\hat{\boldsymbol{j}}+1.0\hat{\boldsymbol{k}}\cr
\end{align}

다음을 계산하라.
\begin{itemize}
  \item[(가)] $\vec{d}_1\cdot(\vec{d}_2+\vec{d}_3)$
  \item[(나)] $\vec{d}_1\cdot(\vec{d}_2\times\vec{d}_3)$
  \item[(다)] $\vec{d}_1\times(\vec{d}_2+\vec{d}_3)$
\end{itemize}
\vspace{0.5 cm}

\noindent{\bf 풀이 :} \\
\begin{itemize}
  \item[(가)] $\vec{d}_2+\vec{d}_3=-1.0\hat{\boldsymbol{j}}+3.0\hat{\boldsymbol{k}}$\\
   $\vec{d}_1\cdot(\vec{d}_2+\vec{d}_3)=(-3.0)(0)+(3.0)(-1.0)+(2.0)(3.0)=3.0$.
  \item[(나)] \begin{align}
     \vec{d}_2\times \vec{d}_3&=\left|\begin{array}{ccc}
      \hat{\boldsymbol{i}}&\hat{\boldsymbol{j}}&\hat{\boldsymbol{k}}\\
      -2.0 & -4.0 & 2.0\\
      2.0 & 3.0 & 1.0
     \end{array}\right|\cr
     &=\{(-4.0)(1.0)-(2.0)(3.0)\}\hat{\boldsymbol{i}}-\{(-2.0)(1.0)-(2.0)(2.0)\}\hat{\boldsymbol{j}}+\{(-2.0)(3.0)-(-4.0)(2.0)\}\hat{\boldsymbol{k}}\cr
     &=-10.0\hat{\boldsymbol{i}}+6.0\hat{\boldsymbol{j}}+2.0\hat{\boldsymbol{k}}
   \end{align}
   따라서 
   \begin{align}
     \vec{d}_1\cdot(\vec{d}_2\times\vec{d}_3)=(-3.0)(-10.0)+(3.0)(6.0)+(2.0)(2.0)=52.0.
   \end{align}

  \item[(다)] \begin{align}
     \vec{d}_1\times(\vec{d}_2+\vec{d}_3)&=\left|\begin{array}{ccc}
      \hat{\boldsymbol{i}}&\hat{\boldsymbol{j}}&\hat{\boldsymbol{k}}\\
      -3.0&3.0&2.0\\
      0.0&-1.0&3.0
     \end{array}\right|\cr
     &=\{(3.0)(3.0)-(2.0)(-1.0)\}\hat{\boldsymbol{i}}-\{(-3.0)(3.0)-(2.0)(0.0)\}\hat{\boldsymbol{j}}+\{(-3.0)(-1.0)-(3.0)(0.0)\}\hat{\boldsymbol{k}}\cr
     &=11.0\hat{\boldsymbol{i}}+9.0\hat{\boldsymbol{j}}+3.0\hat{\boldsymbol{k}}
   \end{align}
\end{itemize}
\noindent 
\end{document}