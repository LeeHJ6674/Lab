%\documentclass[preprint,tightenlines,showpacs,showkeys,floatfix,
%nofootinbib,superscriptaddress,fleqn]{revtex4} 
\documentclass[APS,floatfix,nofootinbib,superscriptaddress,fleqn,preprint]{revtex4} 
%\documentclass[aps,epsfig,tightlines,fleqn]{revtex4}
\usepackage[utf]{kotex}
\usepackage[HWP]{dhucs-interword}
\usepackage[dvips]{color}
\usepackage{graphicx}
\usepackage{bm}
%\usepackage{fancyhdr}
%\usepackage{dcolumn}
\usepackage{defcolor}
\usepackage{amsmath}
\usepackage{amsfonts}
\usepackage{amssymb}
\usepackage{amscd}
\usepackage{amsthm}
\usepackage[utf8]{inputenc}
 \usepackage{setspace}
%\pagestyle{fancy}

\begin{document}

\title{\Large 2022년 1학기 물리학 I: Quiz 2}
\author{Hui-Jae Lee} 
\email{hjlee6674@inha.edu}
\author{김현철\footnote{Office: 5S-436D (면담시간 매주s
    화요일-16:00$\sim$20:00)}} 
\email{hchkim@inha.ac.kr}
\affiliation{Hadron Theory Group, Department of Physics, Inha University,
Incheon 22212, Republic of Korea }
\date{Spring semester, 2022}


\vspace{1.cm}
\begin{abstract}
\noindent \textbf{ {\color{red}주의}: \color{blue} 단 한 번의 부정행위도 절대
  용납하지 않습니다. 적발 시, 학점은 F를 받게 됨은 물론이고,
  징계위원회에 회부합니다. One strike out임을 명심하세요.}\\
\\
문제는 다음 쪽부터 나옵니다.  \\ \\
{\bf Date:} 2022년 3월 7일 (월) 15:30-16:15 
\\
{\bf 학번:} \hspace{4cm}
{\bf 이름:} 

\end{abstract}
\maketitle
-----------------------------------------------------------------------------------------------------------------
\noindent {\bf 문제 1 [20pt]}


  \vspace{0.5cm}
  
\end{itemize} 
\noindent {\bf 문제 2 [10pt]}

\noindent {\bf 풀이} 

\vspace{0.5cm}

\noindent {\bf 문제 3 [10pt]} 

\noindent {\bf 풀이}

\vspace{0.5cm}

\noindent {\bf 문제 4 [20pt]}

\noindent {\bf 풀이}

\end{document}
