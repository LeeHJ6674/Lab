%\documentclass[preprint,tightenlines,showpacs,showkeys,floatfix,
%nofootinbib,superscriptaddress,fleqn]{revtex4} 
\documentclass[APS,floatfix,nofootinbib,superscriptaddress,fleqn,preprint]{revtex4} 
%\documentclass[aps,epsfig,tightlines,fleqn]{revtex4}
\usepackage[utf]{kotex}
\usepackage[HWP]{dhucs-interword}
\usepackage[dvips]{color}
\usepackage{graphicx}
\usepackage{bm}
%\usepackage{fancyhdr}
%\usepackage{dcolumn}
\usepackage{defcolor}
\usepackage{amsmath}
\usepackage{amsfonts}
\usepackage{amssymb}
\usepackage{amscd}
\usepackage{amsthm}
\usepackage[utf8]{inputenc}
 \usepackage{setspace}
%\pagestyle{fancy}

\begin{document}

\title{\Large 2022년 1학기 물리학 I: Quiz 2}
\author{Hui-Jae Lee} 
\email{hjlee6674@inha.edu}
\author{김현철\footnote{Office: 5S-436D (면담시간 매주s
    화요일-16:00$\sim$20:00)}} 
\email{hchkim@inha.ac.kr}
\affiliation{Hadron Theory Group, Department of Physics, Inha University,
Incheon 22212, Republic of Korea }
\date{Spring semester, 2022}


\vspace{1.cm}
\begin{abstract}
\noindent \textbf{ {\color{red}주의}: \color{blue} 단 한 번의 부정행위도 절대
  용납하지 않습니다. 적발 시, 학점은 F를 받게 됨은 물론이고,
  징계위원회에 회부합니다. One strike out임을 명심하세요.}\\
\\
문제는 다음 쪽부터 나옵니다.  \\ \\
{\bf Date:} 2022년 3월 7일 (월) 15:30-16:15 
\\
{\bf 학번:} \hspace{4cm}
{\bf 이름:} 

\end{abstract}
\maketitle
-----------------------------------------------------------------------------------------------------------------
\noindent {\bf 문제 1 [20pt]}

\noindent {\bf 해답}
\begin{itemize}
  \item[(가)]공은 중력에 의한 포물선 운동을 하므로, 수직 방향 운동과
  수평 방향 운동을 따로 생각할 수 있다. 수직 방향으로는 중력에 의한 등가속도 운동을 하게 된다.
  이 공의 초기 수직 방향 속력은 다음과 같다.
  \begin{align}
    v_{x,0}=v_i\sin{\theta_i}=(20.0\,\mathrm{m/s})\sin{30^\circ}
    =(20.0\,\mathrm{m/s})\left(\frac{1}{2}\right)
  \end{align}
  초기 속력의 방향과 중력 가속도의 방향이 반대라는 사실에 유의하면, 
  시간 $t$ 일 때 이 공의 위치는 다음과 같다.
  \begin{align}
    x=x_0+v_0t-\frac{1}{2}gt^2
  \end{align}
  $v_0=v_{x,0}$ 이고, 초기 위치는 45 m, 나중 위치는 0 m 이므로,
  \begin{align}
    \begin{split} 
      0\,\mathrm{m}&=45\,\mathrm{m}+(20.0\,\mathrm{m/s})
      \left(\frac{1}{2}\right)\times t
      -\left(\frac{1}{2}\right)
      (9.80\,\mathrm{m/s^2})\times t^2 \\
      &=45\,\mathrm{m}+(10.0\,\mathrm{m/s})\times t
      -(4.90\,\mathrm{m/s^2})\times t^2
    \end{split}
  \end{align}
  이는 $t$ 에 대한 2차 방정식이다. 해는 $t=-2.18\,\mathrm{s} $ 
  그리고 $t=4.22\,\mathrm{s}$ 이다. 
  따라서, 걸린 시간은 $4.22\,\mathrm{s}$ 이다.
  \item[(나)] 수직 방향 속력은 중력 가속도의 영향을 받아 변하지만, 
  수평 방향으로는 가속도가 존재하지 않기 때문에 수평 방향 속력은 
  변하지 않는다. 수평 방향 속력은 다음과 같다.
  \begin{align}
    v_y=v_i\cos{\theta_i}=(20.0\,\mathrm{m/s})\cos{30^\circ}
    =(20.0\,\mathrm{m/s})\left(\frac{\sqrt{3}}{2}\right)
  \end{align}
  수직 방향 속력은 중력 가속도를 받아 지면에 닿을 때 까지 일정하게 변한다. 
  지면에 닿을 때 수직 방향 속력은 다음과 같다.
  \begin{align}
    v_x=v_{x,0}-gt=v_i\sin{\theta_i}-gt
    =(20.0\,\mathrm{m/s})\left(\frac{1}{2}\right)
    -(9.80\,\mathrm{m/s^2})(4.22\,\mathrm{s})
  \end{align}
  이 공의 전체 속력은 다음과 같다.
  \begin{align}
    \begin{split}
      v&=\sqrt{v_x^2+v_y^2}=\sqrt{{(v_i\sin{\theta_i}-gt)}^2+{(v_i\cos{\theta_i})}^2} \\
      &=\sqrt{{\left((20.0\,\mathrm{m/s})\left(\frac{1}{2}\right)
      -(9.80\,\mathrm{m/s^2})(4.22\,\mathrm{s})\right)}^2
      +{\left((20.0\,\mathrm{m/s})\left(\frac{\sqrt{3}}{2}\right)\right)}^2}  \\
      &=\sqrt{{\left((10.0\,\mathrm{m/s})
      -(41.4\,\mathrm{m/s})\right)}^2
      +{\left((10.0\,\sqrt{3}\,\mathrm{m/s})\right)}^2} \\
      &=\sqrt{{\left(-31.4\,\mathrm{m/s}\right)}^2
      +300\,\mathrm{{(m/s)}^2}} \\
      &=\sqrt{986\,\mathrm{{(m/s)}^2}
      +300\,\mathrm{{(m/s)}^2}} \\
      &=35.86\,\mathrm{m/s}
    \end{split}
  \end{align}
  공이 지면에 닿을 때 속력은 $35.86\,\mathrm{m/s}$ 이다.
  
  \vspace{0.5cm}
  
\end{itemize} 
\noindent {\bf 문제 2 [10pt]}

\noindent {\bf 풀이} 

\vspace{0.5cm}

\noindent {\bf 문제 3 [10pt]} 

\noindent {\bf 풀이}

\vspace{0.5cm}

\noindent {\bf 문제 4 [20pt]}

\noindent {\bf 풀이}

\end{document}
