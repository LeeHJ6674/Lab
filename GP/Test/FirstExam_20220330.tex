\documentclass[floatfix,nofootinbib,superscriptaddress,fleqn]{revtex4-2} 
%\documentclass[aps,epsfig,tightlines,fleqn]{revtex4}
\usepackage[utf]{kotex}
\usepackage[HWP]{dhucs-interword}
\usepackage[dvips]{color}
\usepackage{graphicx}
\usepackage{bm}
%\usepackage{fancyhdr}
%\usepackage{dcolumn}
\usepackage{defcolor}
\usepackage{amsmath}
\usepackage{amsfonts}
\usepackage{amssymb}
\usepackage{amscd}
\usepackage{amsthm}
\usepackage[utf8]{inputenc}
 \usepackage{setspace}
%\pagestyle{fancy}
\usepackage{tikz}

\begin{document}

\title{\Large 2022년 1학기 물리학 I: 제1차 시험}
\author{김현철\footnote{Office: 5S-436D (면담시간 매주
    화요일-16:00$\sim$20:00)}} 
\email{hchkim@inha.ac.kr}
\author{Lee Hui-Jae} 
\email{hjlee6674@inha.edu}
\affiliation{Hadron Theory Group, Department of Physics,
Inha University, Incheon 22212, Republic of Korea }
\date{Spring semester, 2022}


\maketitle

\noindent {\bf 문제 (300pt)} 
그림~\ref{fig:1}처럼 스키 점프대를 만들었다. 비탈면은 평지에서부터
$32^\circ$의 각으로 기울어져 있다. 그리고 비탈면의 길이는 $d_1=100$
m이고, 비탈면이 끝나고 지면과 나란한 지점부터 스키 선수가 점프하는
지점까지 거리는 $d_2=2$ m이다. 
\begin{figure}[ht]
  \centering
\includegraphics[scale=0.5]{Qfig9-20220330.png}  
  \caption{스키 점프대}
  \label{fig:1}
\end{figure}
우선 지면과 스키 사이에 쓸림이 없고, 스키 선수가 정지상태에 있다가 미끄러져
내려오고 있다고 하자. 스키를 포함하여 스키 선수의 무게는 700 N이다. 
\begin{itemize}
\item[(1)] 스키가 점프하는 위치에서 스키 점프대를 떠날 때, 속력을
  구하여라.
\item[(2)] 스키 선수가 도착하는 지점인 $d_3$를 구하여라.
\item[(3)] $d_3$ 지점에서 스키 선수의 속력을 구하여라.   
\item[(4)] 에너지 보존 법칙을 이용해서 구한 속력과 (1)에서 구한 결과와
  비교하여라. 
\end{itemize}

이제 비탈면의 눈과 스키 사이의 운동마찰계수를 $\mu_k=0.1$이라고 하자.
\begin{itemize}
\item[(5)]  표면과 스키 사이의 쓸림힘을 고려하여 스키가 점프하는
  위치에서 스키 점프대를 떠날 때, 속력을 구하여라.
\item[(6)] 스키 선수가 도착하는 지점인 $d_3$를 구하여라.
\item[(7)]  마찰력에 의해 잃는 에너지를 구하여라.
\item[(8)] 이 경우에 운동에너지 보존은 어떻게 되는가? 
\end{itemize}

이제 스키 선수가 점프했을 때부터 공기저항 때문에 속도가 줄어든다고
하자. 이때 공기저항에 의한 힘을 $\vec{F}_d=100\,\hat{\bm{i}}$ N이라고
하고, 이 힘의 방향은 지면과 나란하고 방향은 스키 선수의 속도의 수평
성분과 반대 방향이라고 하자. 
\begin{itemize}
\item[(9)]  스키 선수가 도착하는 지점인 $d_3$를 구하여라.
\item[(10)] $d_3$ 지점에서 스키 선수의 속력을 구하여라.   
\end{itemize}

\noindent {\bf 풀이}
\begin{itemize}
  \item[(1)] 우선 스키 선수의 자유 물체 다이어 그램을 그리자.
  \begin{figure}[h]
    \begin{tikzpicture}
      \draw node at (-2,3) {구간 $d_1$ : }  ;
      \draw[rotate=-32] (-2.5,0) -- (2.5,0) ;
      \draw[rotate=-32] (0,-2.5) -- (0,2.5) ;
      \draw [rotate=-32,red,very thick,-latex] (0,0.1) -- (0,1.5) 
      node [left,black] {$N_1$};
      \draw[black] (238:1) arc(232:270:0.9)
      node [below=4,left,black] {$32^\circ$} ;
      \draw [blue,very thick,-latex] (0,-0.1) -- (0,-2) 
      node [right,black] {$F_g$};
    \end{tikzpicture}
    \hspace{1cm}
    \begin{tikzpicture}
      \draw node at (-2,3) {구간 $d_2$ : }  ;
      \draw (-2,0) -- (2,0) ;
      \draw (0,-2) -- (0,2) ;
      \draw [red,very thick,-latex] (0,0.1) -- (0,1.5) 
      node [left,black] {$N_2$};

      \draw [blue,very thick,-latex] (0,-0.1) -- (0,-2) 
      node [right,black] {$F_g$};
    \end{tikzpicture} \caption{자유 물체 다이어 그램}
  \end{figure}
  중력의 방향과 스키 점프 선수의 운동 방향 사이 각을 $\phi$ 라고 하면 
  스키 점프 선수에게 해준 일은 다음과 같다.
  \begin{align}
  W = F_g\cos{\phi}  .
  \end{align}
  구간 $d_1$ 과 $d_2$ 에서 중력이 해준 일을 각각 $W_1$, $W_2$ 라고 하면,
  \begin{align}
    \begin{split}
      W &= W_1 + W_2 
      = F_g d_1\cos{\left(\frac{\pi}{2}-\theta\right)} + F_gd_2\cos{90^\circ}  \\
      &=F_g d_1\sin{\theta}.
     \end{split}
  \end{align}
  구간 $d_2$ 에서 스키 점프 선수의 속력을 $v_2$ 라고 하면 
  스키 점프 선수의 운동에너지 변화량과 중력이 해준 일의 양이 같으므로,
  \begin{align}
    W = F_g d_1\sin{\theta}=mgd_1\sin{\theta}=\frac{1}{2}mv_2^2-0,\,\,\, 
    v_2 = \sqrt{2gd_1\sin{\theta}}.
  \end{align}
  따라서 구간 $d_2$ 를 지나 스키 점프 선수가 스키 점프대를 떠날 때의 속력은 다음과 같다.
  \begin{align}
    \begin{split}
      v_2 &= \sqrt{2(9.8\,\mathrm{m/s^2})(100\,\mathrm{m})\sin{32^\circ}}  \\
      &= 32\,\mathrm{m/s}
    \end{split}
  \end{align}
  \item[(2)] 스키 점프 선수가 스키 점프대를 떠날 때의 속력을 초기 속력이라고 하자. 
  각 방향으로의 초기 속력은 $v_{xi}=32\,\mathrm{m/s}$, $v_{yi}=0$ 이다. 
  스키 점프 선수의 착지 위치의 좌표를 $x_f$, $y_f$ 라고 하면,
  \begin{align}
    x_f = d_3\cos{\theta},\,\,\, y_f = -d_3\sin{\theta}
  \end{align}
  $y_f$ 가 음수인 이유는 점프하는 순간의 위치를 원점으로 잡았기 때문이다. 
  $x$ 방향으로는  $y$
  \item[(3)] 
  \item[(4)] 
  \item[(5)] 
  \item[(6)] 
  \item[(7)] 
  \item[(8)] 
  \item[(9)] 
  \item[(10)] 

\end{itemize} 
\end{document}