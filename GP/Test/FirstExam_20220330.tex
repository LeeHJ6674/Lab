\documentclass[floatfix,nofootinbib,superscriptaddress,fleqn]{revtex4-2} 
%\documentclass[aps,epsfig,tightlines,fleqn]{revtex4}
\usepackage[utf]{kotex}
\usepackage[HWP]{dhucs-interword}
\usepackage[dvips]{color}
\usepackage{graphicx}
\usepackage{bm}
%\usepackage{fancyhdr}
%\usepackage{dcolumn}
\usepackage{defcolor}
\usepackage{amsmath}
\usepackage{amsfonts}
\usepackage{amssymb}
\usepackage{amscd}
\usepackage{amsthm}
\usepackage[utf8]{inputenc}
 \usepackage{setspace}
%\pagestyle{fancy}
\usepackage{tikz}

\begin{document}

\title{\Large 2022년 1학기 물리학 I: 제1차 시험}
\author{김현철\footnote{Office: 5S-436D (면담시간 매주
    화요일-16:00$\sim$20:00)}} 
\email{hchkim@inha.ac.kr}
\author{Lee Hui-Jae} 
\email{hjlee6674@inha.edu}
\affiliation{Hadron Theory Group, Department of Physics,
Inha University, Incheon 22212, Republic of Korea }
\date{Spring semester, 2022}


\maketitle

\noindent {\bf 문제 (300pt)} 
그림~\ref{fig:1}처럼 스키 점프대를 만들었다. 비탈면은 평지에서부터
$32^\circ$의 각으로 기울어져 있다. 그리고 비탈면의 길이는 $d_1=100$
m이고, 비탈면이 끝나고 지면과 나란한 지점부터 스키 선수가 점프하는
지점까지 거리는 $d_2=2$ m이다. 
\begin{figure}[ht]
  \centering
\includegraphics[scale=0.5]{Qfig9-20220330.png}  
  \caption{스키 점프대}
  \label{fig:1}
\end{figure}
우선 지면과 스키 사이에 쓸림이 없고, 스키 선수가 정지상태에 있다가 미끄러져
내려오고 있다고 하자. 스키를 포함하여 스키 선수의 무게는 700 N이다. 
\begin{itemize}
\item[(1)] 스키가 점프하는 위치에서 스키 점프대를 떠날 때, 속력을
  구하여라.
\item[(2)] 스키 선수가 도착하는 지점인 $d_3$를 구하여라.
\item[(3)] $d_3$ 지점에서 스키 선수의 속력을 구하여라.   
\item[(4)] 에너지 보존 법칙을 이용해서 구한 속력과 (1)에서 구한 결과와
  비교하여라. 
\end{itemize}

이제 비탈면의 눈과 스키 사이의 운동마찰계수를 $\mu_k=0.1$이라고 하자.
\begin{itemize}
\item[(5)]  표면과 스키 사이의 쓸림힘을 고려하여 스키가 점프하는
  위치에서 스키 점프대를 떠날 때, 속력을 구하여라.
\item[(6)] 스키 선수가 도착하는 지점인 $d_3$를 구하여라.
\item[(7)]  마찰력에 의해 잃는 에너지를 구하여라.
\item[(8)] 이 경우에 운동에너지 보존은 어떻게 되는가? 
\end{itemize}

이제 스키 선수가 점프했을 때부터 공기저항 때문에 속도가 줄어든다고
하자. 이때 공기저항에 의한 힘을 $\vec{F}_d=100\,\hat{\bm{i}}$ N이라고
하고, 이 힘의 방향은 지면과 나란하고 방향은 스키 선수의 속도의 수평
성분과 반대 방향이라고 하자. 
\begin{itemize}
\item[(9)]  스키 선수가 도착하는 지점인 $d_3$를 구하여라.
\item[(10)] $d_3$ 지점에서 스키 선수의 속력을 구하여라.   
\end{itemize}

\noindent {\bf 풀이}
\begin{itemize}
  \item[(1)] 우선 스키 선수의 자유 물체 다이어 그램을 그리자.
  \begin{figure}[h]
    \begin{tikzpicture}
      \draw node at (-2,3) {구간 $d_1$ : }  ;
      \draw[rotate=-32] (-2.5,0) -- (2.5,0) ;
      \draw[rotate=-32] (0,-2.5) -- (0,2.5) ;
      \draw [rotate=-32,red,very thick,-latex] (0,0.1) -- (0,1.5) 
      node [left,black] {$N_1$};
      \draw[black] (238:1) arc(232:270:0.9)
      node [below=4,left,black] {$32^\circ$} ;
      \draw [blue,very thick,-latex] (0,-0.1) -- (0,-2) 
      node [right,black] {$F_g$};
    \end{tikzpicture}
    \hspace{1cm}
    \begin{tikzpicture}
      \draw node at (-2,3) {구간 $d_2$ : }  ;
      \draw (-2,0) -- (2,0) ;
      \draw (0,-2) -- (0,2) ;
      \draw [red,very thick,-latex] (0,0.1) -- (0,1.5) 
      node [left,black] {$N_2$};

      \draw [blue,very thick,-latex] (0,-0.1) -- (0,-2) 
      node [right,black] {$F_g$};
    \end{tikzpicture} \caption{자유 물체 다이어 그램}
  \end{figure}
  중력의 방향과 스키 점프 선수의 운동 방향 사이 각을 $\phi$ 라고 하면 
  스키 점프 선수에게 해준 일은 다음과 같다.
  \begin{align}
  W = F_g\cos{\phi}  .
  \end{align}
  구간 $d_1$ 과 $d_2$ 에서 중력이 해준 일을 각각 $W_1$, $W_2$ 라고 하면,
  \begin{align}
    \begin{split}
      W &= W_1 + W_2 
      = F_g d_1\cos{\left(\frac{\pi}{2}-\theta\right)} + F_gd_2\cos{90^\circ}  \\
      &=F_g d_1\sin{\theta}.
     \end{split}
  \end{align}
  구간 $d_2$ 에서 스키 점프 선수의 속력을 $v_2$ 라고 하면 
  스키 점프 선수의 운동에너지 변화량과 중력이 해준 일의 양이 같으므로,
  \begin{align}
    W = F_g d_1\sin{\theta}=mgd_1\sin{\theta}=\frac{1}{2}mv_2^2-0,\,\,\, 
    v_2 = \sqrt{2gd_1\sin{\theta}}.
  \end{align}
  따라서 구간 $d_2$ 를 지나 스키 점프 선수가 스키 점프대를 떠날 때의 속력은 다음과 같다.
  \begin{align}
    \begin{split}
      v_2 &= \sqrt{2(9.8\,\mathrm{m/s^2})(100\,\mathrm{m})\sin{32^\circ}}  \\
      &= 32\,\mathrm{m/s}
    \end{split}
  \end{align}
  \item[(2)] 스키 점프 선수가 스키 점프대를 떠날 때의 속력을 초기 속력이라고 하자. 
  각 방향으로의 초기 속력은 $v_{xi}=v_2=32\,\mathrm{m/s}$, $v_{yi}=0$ 이다. 
  스키 점프 선수의 착지 위치의 좌표를 $x_f$, $y_f$ 라고 하면,
  \begin{align}\label{eq:2-1}
    x_f = d_3\cos{\theta},\,\,\, y_f = -d_3\sin{\theta},
  \end{align}
  $y_f$ 가 음수인 이유는 점프하는 순간의 위치를 원점으로 잡았기 때문이다. 
  $x$ 방향으로는 등속 운동, $y$ 방향으로는 중력에 의한 등가속도 운동을 하므로,
  \begin{align}
    x_f = v_{xi}t= d_3\cos{\theta},\,\,\,
    y_f= -d_3\sin{\theta}=-\frac{1}{2}gt^2.
  \end{align}
  $t$ 를 소거하면 $d_3$ 는 다음과 같다.
  \begin{align}\label{eq:2-2}
    d_3 = \frac{2v^2_{xi}\sin{\theta}}{g\cos^2{\theta}}
    =\frac{2(32\,\mathrm{m/s})^2\sin{32^\circ}}
    {(9.8\,\mathrm{m/s^2})\cos^2{32^\circ}}
    =154\,\mathrm{m}
  \end{align}
  $d_3$ 는 $154\,\mathrm{m}$ 이고 식 (\ref{eq:2-1}) 에 의해
   착지 위치는 다음과 같다.
\begin{align}
  \begin{split}\label{eq:2-3}
    x_f &= (154\,\mathrm{m})\cos{32^\circ}=131\,\mathrm{m} \\
    y_f &= -(154\,\mathrm{m})\sin{32^\circ}=-82\,\mathrm{m}.
  \end{split}
\end{align}
  \item[(3)] 중력은 $y$ 방향으로만 일을 해주므로 중력이 해준 일은 모두 $y$ 방향의 
  운동 에너지 변화량이다. $y$ 방향으로의 움직임 만을 생각하자. 
  중력이 해준 일은,
  \begin{align}
    \begin{split}
      W = \vec{F_g}\cdot\vec{d_3}=mgd_3\sin{\theta}
    \end{split}
  \end{align}
  이고 이 일의 양이 운동에너지 변화량과 같고 $y$ 방향의 초기 속력이 0 이므로
  착지할 때 $y$ 방향의 속력을 $v_y$ 라고 하면,
  \begin{align}
    \frac{1}{2}mv_y^2=\vec{F_g}\cdot\vec{d_3}=mgd_3\sin{\theta},\,\,\,
    v_y = \sqrt{2gd_3\sin{\theta}}.
  \end{align}
  따라서 $v_y$ 는 다음과 같다.
  \begin{align}
    v_y = \sqrt{2(9.8\,\mathrm{m/s^2})(154\,\mathrm{m})\sin{32^\circ}}
      =40\,\mathrm{m/s}.
  \end{align}
  $x$ 방향의 속력은 일정하므로 착지할 때 $x$ 방향의 속력을 $v_x$ 라고 하면
   $v_x=v_{xi}=32\,\mathrm{m/s}$ 이다. 그러므로 착지할 때의 속력은 다음과 같다.
   \begin{align}
    \begin{split}
      v &= \sqrt{v_x^2+v_y^2} 
      = \sqrt{(32\,\mathrm{m/s})^2+(37\,\mathrm{m/s})^2}  \\
      &=49\,\mathrm{m/s}
    \end{split}
   \end{align}
  \item[(4)] 초기 역학적 에너지를 $E_i$, 
  $d_2$ 지점에서의 역학적 에너지를 $E_2$ 라고 하자. $E_i$ 는,
  \begin{align}
    E_i = mgh =mg\left(d_1+d_3\right)\sin{\theta},
  \end{align}
  이고 $d_2$ 지점에서의 속력을 $v_2$ 라고 하면 $E_2$ 는,
  \begin{align}
    E_2 = mgh + \frac{1}{2}mv^2
    = mgd_3\sin{\theta} + \frac{1}{2}mv_2^2.
  \end{align}
  역학적 에너지는 보존되어야 하므로 $E_i=E_2$ 이다. 따라서,
  \begin{align}
    mg\left(d_1+d_3\right)\sin{\theta}
    =mgd_3\sin{\theta} + \frac{1}{2}mv_2^2,\,\,\,
    v_2 = \sqrt{2gd_1\sin{\theta}}.
  \end{align}
  $v_2$ 는 다음과 같다.
  \begin{align}
    v_2 = \sqrt{2(9.8\,\mathrm{m/s^2})(100\,\mathrm{m})\sin{32^\circ}}
    = 32\,\mathrm{m/s}.
  \end{align} 
 (1) 의 결과와 동일함을 확인할 수 있다.
  \item[(5)] 
  마찰력이 작용하는 순간의 자유 물체 다이어 그램을 그려보자. 
  \begin{figure}[h]
    \begin{tikzpicture}
      \draw node at (-2,3) {구간 $d_1$ : }  ;
      \draw[rotate=-32] (-2.5,0) -- (2.5,0) ;
      \draw[rotate=-32] (0,-2.5) -- (0,2.5) ;
      \draw [rotate=-32,red,very thick,-latex] (0,0.1) -- (0,1.5) 
      node [left,black] {$N_1$};
      \draw[black] (238:1) arc(232:270:0.9)
      node [below=4,left,black] {$32^\circ$} ;
      \draw [violet,very thick,-latex] (0,-0.1) -- (0,-2) 
      node [right,black] {$F_g$};
      \draw [rotate=-32,blue,very thick,-latex] (-0.1,0) -- (-1.2,0) 
      node [left,black] {$f_{k1}$};
    \end{tikzpicture}
    \hspace{1cm}
    \begin{tikzpicture}
      \draw node at (-2,3) {구간 $d_2$ : }  ;
      \draw (-2,0) -- (2,0) ;
      \draw (0,-2) -- (0,2) ;
      \draw [red,very thick,-latex] (0,0.1) -- (0,1.5) 
      node [left,black] {$N_2$};
      \draw [blue,very thick,-latex] (-0.1,0) -- (-1.1,0) 
      node [above,black] {$f_{k2}$};

      \draw [violet,very thick,-latex] (0,-0.1) -- (0,-2) 
      node [right,black] {$F_g$};
    \end{tikzpicture} \caption{자유 물체 다이어 그램}
  \end{figure}
  각 구간에 작용하는 마찰력을 고려한다면 각 구간에서 스키 점프 선수가 받는
  알짜힘을 구할 수 있다. 구간 $d_1$ 에서 작용하는 마찰력을 $f_{k1}$ 이라 하고 
  구간 $d_2$ 에 작용하는 마찰력을 $f_{k2}$ 라 하면 각 구간에서 받는  알짜힘 
  $W_1$, $W_2$ 는,
  \begin{align}
    W = W_1+W_2 = F_g d_1\cos{\left(\frac{\pi}{2}-\theta\right)}
    -f_{k1}d_1-f_{k2}d_2,
  \end{align} 
  $f_{k1}=\mu_kN_1$, $f_{k2}=\mu_kN_2$ 이다.
  각 구간에서 스키 점프 선수에게 작용하는 수직항력
  $N_1$, $N_2$ 은 수직항력의 방향과 중력의 방향을 고려했을 때,
  \begin{align}\label{eq:5-1}
    N_1 = F_g\cos{\theta},\,\,\,N_2=F_g,\,\,\,F_g=mg
  \end{align}
  이므로,
  \begin{align}
    \begin{split}
      W &= F_g d_1\cos{\left(\frac{\pi}{2}-\theta\right)}
      -\mu_kF_gd_1\cos{\theta}-\mu_kF_gd_2  \\
      &= mg\left( d_1\sin{\theta}-\mu_k(d_1\cos{\theta}-d_2) \right)
    \end{split}
  \end{align}
  중력이 해준 일은 운동 에너지의 변화량과 같으므로,
  \begin{align}
      W = \frac{1}{2}mv^2_2 
      =mg\left(d_1\sin{\theta}-\mu_k(d_1\cos{\theta}-d_2)\right),\,\,\,
      v_2=\sqrt{2g\left( d_1\sin{\theta}-\mu_k(d_1\cos{\theta}-d_2)\right)}.
  \end{align} 
  따라서 스키 점프대를 떠날 때 속력은,
  \begin{align}
    \begin{split}
      v_2&=\sqrt{2g\left( d_1\sin{\theta}-\mu_k(d_1\cos{\theta}-d_2)\right)} \\
      &= \sqrt{2(9.8\,\mathrm{m/s^2})
      \left((100\,\mathrm{m})\sin{32^\circ}
      -(0.1)((100\,\mathrm{m})\cos{32^\circ}-(2\,\mathrm{m}))\right)} \\
      &= 29\,\mathrm{m/s}.
    \end{split}
  \end{align}
  \item[(6)] 식 (\ref{eq:2-2}) 에 의해 구간 $d_1$, $d_2$ 에서 마찰력을
  받은 경우 $d_3$ 는
  \begin{align}
    \begin{split}
      d_3 = \frac{2v^2_2\sin{\theta}}{g\cos^2{\theta}}
      =\frac{2(29\,\mathrm{m/s})^2\sin{32^\circ}}
      {(9.8\,\mathrm{m/s^2})\cos^2{32^\circ}}
      =130\,\mathrm{m},
    \end{split}
  \end{align} 
  이고 식 (\ref{eq:2-3}) 에 의해 착지 위치는 다음과 같다.
  \begin{align}
    \begin{split}
      x_f &= d_3\cos{\theta}
      =\frac{2v^2_2\sin{\theta}}{g\cos{\theta}}
      =\frac{2(29\,\mathrm{m/s})^2\sin{32^\circ}}
      {(9.8\,\mathrm{m/s^2})\cos{32^\circ}}
      =107\,\mathrm{m} \\
      y_f &= -d_3\sin{\theta}
      =\frac{2v^2_2\sin^2{\theta}}{g\cos{\theta}}
      =\frac{2(29\,\mathrm{m/s})^2\sin^2{32^\circ}}
      {(9.8\,\mathrm{m/s^2})\cos^2{32^\circ}}
      =-67\,\mathrm{m}.
    \end{split}
  \end{align}
  \item[(7)] 스키 점프 선수는 두 구간 $d_1$ 과 $d_2$ 에서 마찰력을 받는다. 
  마찰력에 의해 잃는 힘을 $W_f$ 라고 하면,
  \begin{align}
    W_f = f_{k1}d_1+f_{k2}d_2  = \mu_k(N_1d_1+N_2d_2),
  \end{align}
  이고 식 (\ref{eq:5-1}) 에 의해,
  \begin{align}
    \begin{split}
      W_f &= \mu_kmg(d_1\cos{\theta}+d_2)  \\
      &= (0.1)(700\,\mathrm{N})\left((100\,\mathrm{m})\cos{32^\circ}+2\,\mathrm{m}\right) \\
      &= 6076\,\mathrm{J}
    \end{split}
  \end{align}
  마찰에 의해 잃은 일은 $6076\,\mathrm{J}$ 이다.
  \item[(8)] 마찰에 의해 운동 에너지의 일부가 다른 형태의 에너지로 변형되었으므로 운동 에너지 보존은
   지켜지지 않는다. 그 때 변형된 에너지의 양은 $6076\,\mathrm{J}$ 이다.
  \item[(9)] 공기저항이 작용한다 하면 스키 점프 선수가 공중에 머무를 때의
   자유 물체 다이어 그램은 다음과 같다.
  \begin{figure}[h]
    \begin{tikzpicture}
      \draw (-1.5,0) -- (1.5,0) ;
      \draw (0,-1.5) -- (0,1.5) ;
      \draw [blue,very thick,-latex] (-0.1,0) -- (-1.1,0) 
      node [above,black] {$|\vec{F_d}|$};
      \draw [violet,very thick,-latex] (0,-0.1) -- (0,-2) 
      node [right,black] {$F_g$};
    \end{tikzpicture} \caption{자유 물체 다이어 그램}
  \end{figure}
  오른쪽을 $+x$ 방향이라 하자. 운동 방정식을 세워보면 각 방향으로 스키 점프 선수가 받는 힘은,
  \begin{align}
    \begin{split}
      \sum F_x &= ma_x = -|\vec{F_d}| = -700\,\mathrm{N} \\
      \sum F_y &= ma_y = F_g = mg 
    \end{split}
  \end{align}
  이다. 각 방향으로 힘이 일정하게 작용하므로 스키 점프 선수는 
  $x$ 방향과 $y$ 방향으로 등가속도 운동한다. 스키 점프 선수가 
  점프대를 떠날 때 $y$ 방향의 속력은 0 이므로 
  그때의 속력을 $v_{xi}=v_2$ 라고 하면 각 방향으로의 착지 위치 $x_f$, $y_f$ 
  는 다음과 같이 쓸 수 있다.
  \begin{align}
    \begin{split}\label{eq:9-1}
      x_f &=d_3\cos{\theta}= v_{xi}t+\frac{1}{2}a_xt^2 = v_2t+\frac{1}{2}a_xt^2,\,\,\,
      a_x=-\frac{|\vec{F_d}|}{m}  \\
      y_f &=-d_3\sin{\theta}= -\frac{1}{2}a_yt^2 = -\frac{1}{2}gt^2.
    \end{split}
  \end{align}
  두번째 식으로 부터 $t$ 는,
  \begin{align}\label{eq:9-2}
    t = \sqrt{\frac{2d_3\sin{\theta}}{g}},
  \end{align}
  이고 이를 첫번째 식에 넣으면,
  \begin{align}
    d_3\cos{\theta}=v_2\sqrt{\frac{2d_3\sin{\theta}}{g}}+\frac{a_xd_3\sin{\theta}}{g}
    =v_2\sqrt{\frac{2d_3\sin{\theta}}{g}}-\frac{|\vec{F_d}|d_3\sin{\theta}}{mg}.
  \end{align}
  이 식을 $d_3$ 에 대해 정리해보자.
  \begin{align}
    \left(1+\frac{|\vec{F_d}|\sin{\theta}}{mg\cos{\theta}}\right)d_3
    =v_2\sqrt{\frac{2d_3\sin{\theta}}{g\cos^2{\theta}}},\,\,\,
  \end{align}
  양변을 $\sqrt{d_3}$ 로 나누고 좌항에 $d_3$ 만 남기면,
  \begin{align}
    \sqrt{d_3}=\left(\frac{mg\cos{\theta}}{mg\cos{\theta}
    +|\vec{F_d}|\sin{\theta}}\right)
    \sqrt{\frac{2v_2^2\sin{\theta}}{g\cos^2{\theta}}}.
  \end{align}
  따라서 $d_3$ 는,
  \begin{align}
    d_3=\left(\frac{mg\cos{\theta}}{mg\cos{\theta}
    +|\vec{F_d}|\sin{\theta}}\right)^2
   \frac{2v_2^2\sin{\theta}}{g\cos^2{\theta}},
  \end{align}
  이다. 최종적으로 $d_3$ 을 계산하면 다음과 같다.
  \begin{align}
    \begin{split}
      d_3 &=\left(\frac{(700\,\mathrm{N})\cos{32^\circ}}{(700\,\mathrm{N})\cos{32^\circ}
      +(100\,\mathrm{N})\sin{32^\circ}}\right)^2
      \frac{2(29\,\mathrm{m/s})^2\sin{32^\circ}}{(9.8\mathrm{m/s^2})\cos^2{32^\circ}} \\
      &= 107\,\mathrm{m}.
    \end{split}
  \end{align} 
  이 결과를 식 (\ref{eq:9-1}) 에 대입하면,
  \begin{align}
    \begin{split}
      x_f &= d_3\cos{\theta}= (107\,\mathrm{m})\cos{32^\circ} =  91\,\mathrm{m} \\
      y_f &=-d_3\sin{\theta}= (107\,\mathrm{m})\sin{32^\circ} = -57\,\mathrm{m},
    \end{split}
  \end{align}
  스키 점프 선수의 착지 위치를 구할 수 있다.
  \item[(10)] 지점 $d_3$ 에서의 각 방향에 대한 속력을 $v_{xf}$, $v_{yf}$ 라 하면,
  \begin{align}
    \begin{split}
      v_{x f} &= v_{xi}-a_x t = v_2 - \frac{|\vec{F_d}|}{m}t,\,\,\,m=\frac{F_g}{g} \\
      v_{y f} &= -gt
    \end{split}
  \end{align} 
  식 (\ref{eq:9-2}) 으로 부터 $x$ 방향 속력은,
  \begin{align}
    \begin{split}
      v_{x f} &=  v_2 - \frac{|\vec{F_d}|g}{F_g}\sqrt{\frac{2d_3\sin{\theta}}{g}} \\
      &= (29\,\mathrm{m/s}) - \frac{(100\,\mathrm{N})(9.8\,\mathrm{m/s^2})}
      {(700\,\mathrm{N})}\sqrt{\frac{2(107\,\mathrm{m})\sin{32^\circ}}
      {(9.8\,\mathrm{m/s^2})}}  \\
      &= 24\,\mathrm{m/s},
    \end{split}
  \end{align}
  이고, $y$ 방향 속력은,
  \begin{align}
    \begin{split}
      v_{y f} &= -\sqrt{2d_3g\sin{\theta}}  \\
      &=-\sqrt{2(107\,\mathrm{m})(9.8\,\mathrm{m/s^2})\sin{32^\circ}} \\
      &=-33\,\mathrm{m/s}
    \end{split}
  \end{align}
  이다. 따라서 지점 $d_3$ 에서의 속력은 다음과 같다.
  \begin{align}
    \begin{split}
      v &= \sqrt{v_{xf}^2+v_{yf}^2} = \sqrt{(24\,\mathrm{m/s})^2+(33\,\mathrm{m/s})^2} \\
      &= 41\,\mathrm{m/s}
    \end{split}
  \end{align}
\end{itemize} 
\end{document}