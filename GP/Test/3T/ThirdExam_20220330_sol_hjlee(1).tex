\documentclass[floatfix,nofootinbib,superscriptaddress,fleqn]{revtex4-2} 
%\documentclass[aps,epsfig,tightlines,fleqn]{revtex4}
\usepackage[utf]{kotex}
\usepackage[HWP]{dhucs-interword}
\usepackage[dvips]{color}
\usepackage{graphicx}
\usepackage{bm}
%\usepackage{fancyhdr}
%\usepackage{dcolumn}
\usepackage{defcolor}
\usepackage{amsmath}
\usepackage{amsfonts}
\usepackage{amssymb}
\usepackage{amscd}
\usepackage{amsthm}
\usepackage[utf8]{inputenc}
 \usepackage{setspace}
%\pagestyle{fancy}
\usepackage{tikz}

\begin{document}

\title{\Large 2022년 1학기 물리학 I: 제3차 시험}
\author{김현철\footnote{Office: 5S-436D (면담시간 매주
    화요일-16:00$\sim$20:00)}} 
\email{hchkim@inha.ac.kr}
\author{Lee Hui-Jae} 
\email{hjlee6674@inha.edu}
\affiliation{Hadron Theory Group, Department of Physics,
Inha University, Incheon 22212, Republic of Korea }
\date{Spring semester, 2022}


\maketitle
 
\noindent {\bf 1번 풀이 : }        
질량이 $m$인 물체가 각 물체에 대해 가지고 있는 중력 

\vspace{1cm}
 

\vspace{1cm} 
\noindent {\bf 2번 풀이 : } 
줄의 장력 $T$가 도르래에 돌림힘으로 작용하여 도르래를 회전시킨다. 따라서,
\begin{align}
  TR = I\alpha = \frac{1}{2}MRa \Longrightarrow T =\frac{1}{2}Ma
\end{align}    
이고 장력과 중력이 작용하여 중력이 $M/2$인 물체를 움직이도록 한다.
도르래와 물체가 줄로 연결되어 있으므로 가속도의 크기는 같다.
\begin{align}
  \frac{1}{2}Ma = \frac{1}{2}Mg - T.
\end{align}
두 식을 연립하면 장력 $T$는
\begin{align}
  T = \frac{1}{4}Mg
\end{align}
이다.

 

\vspace{1cm} 
\noindent {\bf 3번 풀이 : }
가장 멀리있을 때 거리, 속력을 $r_1, v_1$이라 하고
가장 가까이 있을 때 거리, 속력을 $r_2, v_2$라고 하면,
각운동량 보존법칙에 의해 
\begin{align}
  mv_1r_1 = mv_2r_2
\end{align}
이다. $r_1 = 2r_2$이므로
\begin{align}
  2v_1r_2 = v_2r_2\Longrightarrow 2v_1 = v_2
\end{align}
이다. 따라서 최대 선속력은 최소 선속력의 2배이다.
\vspace{1cm}
 

\vspace{1cm} 

\noindent {\bf 4번 풀이 : }        
팽창 전 별의 회전 운동에너지는
\begin{align}
  E_R = \frac{1}{2}I\omega^2
\end{align}
이다. 팽창 후 별의 회전관성이 3배 늘어났으므로 별의 회전 운동에너지는
\begin{align}
  E'_R=\frac{3}{2}I\omega^2
\end{align}
이다.
\vspace{1cm}
 

\vspace{1cm} 
\noindent {\bf 5번 풀이 : }
액체의 밀도를 $\rho_l$이라 하면 잠긴 부피 만큼의 물의 질량과
물체의 질량이 같으므로
\begin{align}
  L \rho_l = L_0 \rho
\end{align}
이고 $\rho_l$은
\begin{align}
  \rho_l = \frac{L_0}{L}\rho
\end{align}
이다.
\vspace{1cm}
 

\vspace{1cm} 
\noindent {\bf 6번 풀이 : }        

\vspace{1cm}
 

\vspace{1cm} 
\noindent {\bf 7번 풀이 : }        

\vspace{1cm}
 

\vspace{1cm} 
\noindent {\bf 8번 풀이 : }        

\vspace{1cm}
 

\vspace{1cm} 
\noindent {\bf 9번 풀이 : }        
$3\,\mathrm{m}$
\vspace{1cm}
 

\vspace{1cm} 
\noindent {\bf 10번 풀이 : }       

\vspace{1cm}
 

\vspace{1cm} 
\noindent {\bf 11번 풀이 : }       

\vspace{1cm}
 

\vspace{1cm} 
\noindent {\bf 12번 풀이 : }       

\vspace{1cm}
 

\vspace{1cm} 
\noindent {\bf 주관식 1번 풀이 : }

\vspace{1cm}
 

\vspace{1cm} 
\noindent {\bf 주관식 2번 풀이 : }

\vspace{1cm}
 

\vspace{1cm} 
\noindent {\bf 주관식 3번 풀이 : } 

\vspace{1cm}
 

\vspace{1cm} 




\end{document}