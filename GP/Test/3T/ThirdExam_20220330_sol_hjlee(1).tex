\documentclass[floatfix,nofootinbib,superscriptaddress,fleqn]{revtex4-2} 
%\documentclass[aps,epsfig,tightlines,fleqn]{revtex4}
\usepackage[utf]{kotex}
\usepackage[HWP]{dhucs-interword}
\usepackage[dvips]{color}
\usepackage{graphicx}
\usepackage{bm}
%\usepackage{fancyhdr}
%\usepackage{dcolumn}
\usepackage{defcolor}
\usepackage{amsmath}
\usepackage{amsfonts}
\usepackage{amssymb}
\usepackage{amscd}
\usepackage{amsthm}
\usepackage[utf8]{inputenc}
 \usepackage{setspace}
%\pagestyle{fancy}
\usepackage{tikz}

\begin{document}

\title{\Large 2022년 1학기 물리학 I: 제3차 시험}
\author{김현철\footnote{Office: 5S-436D (면담시간 매주
    화요일-16:00$\sim$20:00)}} 
\email{hchkim@inha.ac.kr}
\author{Lee Hui-Jae} 
\email{hjlee6674@inha.edu}
\affiliation{Hadron Theory Group, Department of Physics,
Inha University, Incheon 22212, Republic of Korea }
\date{Spring semester, 2022}


\maketitle
 
\noindent {\bf 1번 풀이 : }        
질량이 $m$인 물체가 정삼각형의 각 꼭짓점에 놓여 있을 때 계의 총 중력 퍼텐셜에너지는
\begin{align}
  E_p =-3G\frac{m^2}{a}
\end{align} 
이고 한 물체를 무한히 먼 곳에 이동시켜 계에 두 물체만 남았다면 계의 총 중력 퍼텐셜에너지는
\begin{align}
  E'_p = -G\frac{m^2}{a}
\end{align}
이다. 외부에서 해준 일은 계의 총 에너지 변화량과 같으므로
\begin{align}
  W = E'_p - E_p = 2G\frac{m^2}{a}
\end{align}
이다.
\vspace{0.5cm}
 

\vspace{0.5cm} 
\noindent {\bf 2번 풀이 : } 
줄의 장력 $T$가 도르래에 돌림힘으로 작용하여 도르래를 회전시킨다. 따라서,
\begin{align}
  TR = I\alpha = \frac{1}{2}MRa \Longrightarrow T =\frac{1}{2}Ma
\end{align}    
이고 장력과 중력이 작용하여 중력이 $M/2$인 물체를 움직이도록 한다.
도르래와 물체가 줄로 연결되어 있으므로 가속도의 크기는 같다.
\begin{align}
  \frac{1}{2}Ma = \frac{1}{2}Mg - T.
\end{align}
두 식을 연립하면 장력 $T$는
\begin{align}
  T = \frac{1}{4}Mg
\end{align}
이다.

 

\vspace{0.5cm} 
\noindent {\bf 3번 풀이 : }
가장 멀리있을 때 거리, 속력을 $r_1, v_1$이라 하고
가장 가까이 있을 때 거리, 속력을 $r_2, v_2$라고 하면,
각운동량 보존법칙에 의해 
\begin{align}
  mv_1r_1 = mv_2r_2
\end{align}
이다. $r_1 = 2r_2$이므로
\begin{align}
  2v_1r_2 = v_2r_2\Longrightarrow 2v_1 = v_2
\end{align}
이다. 따라서 최대 선속력은 최소 선속력의 2배이다.
\vspace{0.5cm}
 

\vspace{0.5cm} 

\noindent {\bf 4번 풀이 : }        
팽창 전 별의 회전 운동에너지는
\begin{align}
  E_R = \frac{1}{2}I\omega^2
\end{align}
이다. 팽창 후 별의 회전관성이 3배 늘어났으므로 별의 회전 운동에너지는
\begin{align}
  E'_R=\frac{3}{2}I\omega^2
\end{align}
이다.
\vspace{0.5cm}
 

\vspace{0.5cm} 
\noindent {\bf 5번 풀이 : }
액체의 밀도를 $\rho_l$이라 하면 잠긴 부피 만큼의 물의 질량과
물체의 질량이 같으므로
\begin{align}
  L \rho_l = L_0 \rho
\end{align}
이고 $\rho_l$은
\begin{align}
  \rho_l = \frac{L_0}{L}\rho
\end{align}
이다.
\vspace{0.5cm}
 

\vspace{0.5cm} 
\noindent {\bf 6번 풀이 : }        
베르누이 방정식에 의해
\begin{align}
  P_1 + \frac{1}{2}\rho v_1^2
  =  P_2 + \frac{1}{2}\rho v_2^2
\end{align}
이므로 $P_2$는
\begin{align}
  P_2 = P_1 + \frac{1}{2}\rho\left(v_1^2-v_2^2\right)
\end{align}
이다.
\vspace{0.5cm}
 

\vspace{0.5cm} 
\noindent {\bf 7번 풀이 : }        
막대의 회전에 대한 운동방정식은
\begin{align}
  I\alpha = -MgL\sin\theta \approx -MgL\theta
\end{align}
이다. 각 $\theta$가 매우 작을 때 근사를 취하였다. 따라서
\begin{align}
  \frac{1}{3}ML^2 \frac{d^2\theta}{dt^2} = -MgL\theta \Longrightarrow
  \frac{d^2\theta}{dt^2} = -\frac{3g}{L}\theta = -\omega^2\theta
\end{align}
이므로 각진동수 $\omega$는
\begin{align}
  \omega = \sqrt{\frac{3g}{L}}
\end{align}
임을 알 수 있다. 주기 $T$는
\begin{align}
  T = \frac{2\pi}{\omega} =2\pi\sqrt{\frac{L}{3g}}
\end{align}
이다. 이로부터 주기는 질량이 2배 늘어나고 길이도 2배 늘어나면 늘어나기 전 주기의
$\sqrt{2}$배가 된다.
\vspace{0.5cm}
 

\vspace{0.5cm} 
\noindent {\bf 8번 풀이 : }  

\vspace{0.5cm}
 

\vspace{0.5cm} 
\noindent {\bf 9번 풀이 : }
정상파의 기본진동수의 파형이라면 줄의 길이 $L$은 정상파의 파장 $\lambda$의 절반이다. 즉,
\begin{align}
  L = \frac{1}{2}\lambda
\end{align}
이다. 정상파의 파수를 $k$라 하면 정상파의 파장 $\lambda$은
\begin{align}
  \lambda = \frac{2\pi}{k}
\end{align}
이므로 줄의 길이 $L$은
\begin{align}
  L = \frac{\pi}{\lambda} = 3\,\mathrm{m}
\end{align}
이다.
\vspace{0.5cm}
 

\vspace{0.5cm} 
\noindent {\bf 10번 풀이 : }       
원래 음원의 진동수보다 관측자가 듣는 음원의 진동수가 감소한느 경우는 관측자와 음원이 서로 멀어지는
경우이다. 따라서 답은 (2), (4)번이다.
\vspace{0.5cm}
 

\vspace{0.5cm} 
\noindent {\bf 11번 풀이 : }
\begin{itemize}
  \item[(a)]
  물리진자에 작용하는 돌림힘 $\tau$는
  \begin{align}
    \tau = -mgh\sin\theta
  \end{align}
  이다. $\theta$가 매우 작다면 $\sin\theta\approx\theta$로 근사할 수 있으므로
  \begin{align}
    \tau \approx -mgh\theta
  \end{align}
  이다.
  \item[(b)] 
  문제 7번과 같이 각진동수 $\omega$를 구하면
  \begin{align}
    \omega = \sqrt{\frac{mgh}{I}}
  \end{align}
  이고 주기 $T$는
  \begin{align}
    T = 2\pi\sqrt{\frac{I}{mgh}}
  \end{align}
  이다.
\end{itemize}
\vspace{0.5cm}
 

\vspace{0.5cm} 
\noindent {\bf 12번 풀이 : }       
중첩된 파동의 방정식은
\begin{align}
  y(x,t) = A\sin(kx-\omega t)+A\cos(kx-\omega t)
\end{align}
이다. 삼각함수 덧셈공식 $\sin(a+b) = \sin a\cos b+\sin b\cos a$를 생각하자.
$a = kx-\omega t$, $b= \pi/4$라고 생각하면
\begin{align}
  y(x,t) &= \frac{A}{\cos\frac{\pi}{4}}\sin(kx-\omega t)\cos\frac{\pi}{4}
  +\frac{A}{\sin\frac{\pi}{4}}\cos(kx-\omega t)\sin\frac{\pi}{4}  \\
  &=\sqrt{2}A\sin(kx-\omega t+\frac{\pi}{4})
\end{align}
이므로 진폭은 $\sqrt{2}A$이다.
\vspace{0.5cm}
 

\vspace{0.5cm} 
\noindent {\bf 주관식 1번 풀이 : }

\vspace{0.5cm}
 

\vspace{0.5cm} 
\noindent {\bf 주관식 2번 풀이 : }

\vspace{0.5cm}
 

\vspace{0.5cm} 
\noindent {\bf 주관식 3번 풀이 : } 
Quiz 13 참조.
\vspace{0.5cm}
 

\vspace{0.5cm} 




\end{document}