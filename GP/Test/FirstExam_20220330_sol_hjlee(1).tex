\documentclass[floatfix,nofootinbib,superscriptaddress,fleqn]{revtex4-2} 
%\documentclass[aps,epsfig,tightlines,fleqn]{revtex4}
\usepackage[utf]{kotex}
\usepackage[HWP]{dhucs-interword}
\usepackage[dvips]{color}
\usepackage{graphicx}
\usepackage{bm}
%\usepackage{fancyhdr}
%\usepackage{dcolumn}
\usepackage{defcolor}
\usepackage{amsmath}
\usepackage{amsfonts}
\usepackage{amssymb}
\usepackage{amscd}
\usepackage{amsthm}
\usepackage[utf8]{inputenc}
 \usepackage{setspace}
%\pagestyle{fancy}
\usepackage{tikz}

\begin{document}

\title{\Large 2022년 1학기 물리학 I: 제3차 시험}
\author{김현철\footnote{Office: 5S-436D (면담시간 매주
    화요일-16:00$\sim$20:00)}} 
\email{hchkim@inha.ac.kr}
\author{Lee Hui-Jae} 
\email{hjlee6674@inha.edu}
\affiliation{Hadron Theory Group, Department of Physics,
Inha University, Incheon 22212, Republic of Korea }
\date{Spring semester, 2022}


\maketitle
 
\noindent {\bf 1. }         
한 변의 길이가 $a$인 정삼각형의 세 꼭짓점에 질량이 $m$인 물체가 하나씩 놓여 있다. 
물체 사이에는 만유인력이 작용하고 있다. 이 중 한 물체를 무한히 먼 곳으로 이동시키는 데 
외부에서 해 주어야 하는 일을 구하시오. (만유인력 상수 $G$를 사용하시오.) 

\vspace{1cm}
\noindent {\bf 풀이. } 

\vspace{1cm} 
\noindent {\bf 2. }         

\vspace{1cm}
\noindent {\bf 풀이. } 

\vspace{1cm} 
\noindent {\bf 3. }         

\vspace{1cm}
\noindent {\bf 풀이. } 

\vspace{1cm} 
\noindent {\bf 4. }         

\vspace{1cm}
\noindent {\bf 풀이. } 

\vspace{1cm} 
\noindent {\bf 5. }         

\vspace{1cm}
\noindent {\bf 풀이. } 

\vspace{1cm} 
\noindent {\bf 6. }         

\vspace{1cm}
\noindent {\bf 풀이. } 

\vspace{1cm} 
\noindent {\bf 7. }         

\vspace{1cm}
\noindent {\bf 풀이. } 

\vspace{1cm} 
\noindent {\bf 8. }         

\vspace{1cm}
\noindent {\bf 풀이. } 

\vspace{1cm} 
\noindent {\bf 9. }         

\vspace{1cm}
\noindent {\bf 풀이. } 

\vspace{1cm} 
\noindent {\bf 10. }        

\vspace{1cm}
\noindent {\bf 풀이. } 

\vspace{1cm} 
\noindent {\bf 11. }        

\vspace{1cm}
\noindent {\bf 풀이. } 

\vspace{1cm} 
\noindent {\bf 12. }        

\vspace{1cm}
\noindent {\bf 풀이. } 

\vspace{1cm} 
\noindent {\bf 주관식 1. } 

\vspace{1cm}
\noindent {\bf 풀이. } 

\vspace{1cm} 
\noindent {\bf 주관식 2. }

\vspace{1cm}
\noindent {\bf 풀이. } 

\vspace{1cm} 
\noindent {\bf 주관식 3. }  

\vspace{1cm}
\noindent {\bf 풀이. } 

\vspace{1cm} 




\end{document}