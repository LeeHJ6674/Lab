%\documentclass[preprint,tightenlines,showpacs,showkeys,floatfix,
%nofootinbib,superscriptaddress,fleqn]{revtex4} 
\documentclass[floatfix,nofootinbib,superscriptaddress,fleqn]{revtex4-2} 
%\documentclass[aps,epsfig,tightlines,fleqn]{revtex4}
\usepackage{kotex}
\usepackage[HWP]{dhucs-interword}
\usepackage[dvips]{color}
\usepackage{graphicx}
\usepackage{bm}
%\usepackage{fancyhdr}
%\usepackage{dcolumn}
\usepackage{defcolor}
\usepackage{amsmath}
\usepackage{amsfonts}
\usepackage{amssymb}
\usepackage{amscd}
\usepackage{amsthm}
\usepackage[utf8]{inputenc}
\usepackage{mathtools}
\usepackage{bbm}
%\pagestyle{fancy}

\begin{document}

\title{\Large Quantum Mechanics}
\author{김현철}
\email{hchkim@inha.ac.kr}
\affiliation{Hadron Theory Group, Department of Physics, Inha University,
Incheon 22212, Republic of Korea }
\date{2021}

\maketitle

\noindent {\bf Due date:} \textbf{\color{red} April 9, 2022} \\ 
\vspace{2cm}

\section*{\large Problem Set 2}
\noindent \textbf{Problem 1.}
A constant electric field $\mathcal{E}$ is exerted on a charged linear
harmonic oscillator. 
\begin{itemize}
\item[(1)] Write down the corresponding Schr\"odinger equation. 
\item[(2)] Derive the eigenvalues and eigenvectors of the charged
  linear oscillators under a uniform electric field. 
\item[(3)] Discuss the change in energy levels and physics. 
  eigenstates. 
\end{itemize}
Hint: Use the operator method.

\noindent \textbf{Answer : }
\begin{itemize}
  \item[(1)] A charged particle away from the equilibrium position 
  has the potential energy when it is in the electric field. 
  Let a distance from equilibrium position to a particle is $x$.
  In the constant electric field, the electric potential energy $E_p$ is,
  \begin{align}
    E_p = q\mathcal{E}x.
  \end{align} 
  Then, the Hamiltonain of the charged linear
  harmonic oscillator is,
  \begin{align}\label{eq:1-1}
    H = \frac{p^2}{2m}+\frac{1}{2}m\omega^2x^2-q\mathcal{E}x.
  \end{align}
  So, the Schr\"odinger equation is,
  \begin{align}\label{eq:1-2}
    -\frac{\hbar}{2m}\frac{\partial^2\psi}{\partial x^2}
    +\frac{1}{2}m\omega^2x^2\psi
    -q\mathcal{E}x\psi = E\psi.
  \end{align}
  \item[(2)] First, Suppose that there is no electric field.
  That is, $\mathcal{E}$ is zero. Then the Schr\"odinger equation and 
  the energy is,
  \begin{align}
    -\frac{\hbar}{2m}\frac{\partial^2\psi}{\partial x^2}
    +\frac{1}{2}m\omega^2x^2\psi = E\psi,\,\,\, 
    E_n = \left(\frac{1}{2}+n\right)\hbar\omega.
  \end{align}
  It is the Schr\"odinger equation of the simple harmonic oscillator. 
  In the algebric method to solve the equation, 
  we defined new operators,
  \begin{align}
    \begin{split}
      a = \sqrt{\frac{m\omega}{2\hbar}}\left(x+i\frac{p}{m\omega}\right),\,\,\,
      a^\dagger = \sqrt{\frac{m\omega}{2\hbar}}\left(x-i\frac{p}{m\omega}\right),\,\,\,
      \left[a,a^\dagger\right] = \mathbbm{I}.
    \end{split}
  \end{align}
  $\mathbbm{I}$ is the identity operator. And,
  \begin{align}
    x = \sqrt{\frac{2\hbar}{m\omega}}\left(\frac{a+a^\dagger}{2}\right),\,\,\,
    p = \sqrt{2\hbar m\omega}\left(\frac{a-a^\dagger}{2i}\right)
  \end{align}
  It is said to be ladder operators. Operators are from the hamiltonian of 
  the simple harmonic oscillator,
  \begin{align}\label{eq:1-3}
    H = \frac{p^2}{2m}+\frac{1}{2}m\omega^2x^2 
    = \hbar\omega\left(a^\dagger a + \frac{1}{2}\right)
    = \hbar\omega\left(a a^\dagger - \frac{1}{2}\right).
  \end{align}
  Now, recall that there is a constant electric field $\mathcal{E}$. From Eq.~\eqref{eq:1-1}
   and \eqref{eq:1-3}, hamiltonian with a constant electric field is,
   \begin{align}
    H^\prime = \hbar\omega\left(a^\dagger a + \frac{1}{2}-q\mathcal{E}x\right)
    = \hbar\omega\left(a^\dagger a + \frac{1}{2}-\frac{q\mathcal{E}}{2}
    \sqrt{\frac{2\hbar}{m\omega}}\left(a+a^\dagger\right)\right),
   \end{align}  
   To eliminate the terms of $a$ and $a^\dagger$, define the new operator $b$.
   \begin{align}
    b \coloneqq a + \kappa,\,\,\,
    \left[a,b\right] = a(a+\kappa)-(a+\kappa)a = aa-aa+\kappa a-\kappa a=0.
   \end{align}
   With this definition, the new operator $b$ commutes with the old one $a$. 
   And the adjoint of $b$ is automatically $b^\dagger = a^\dagger+\kappa$.
   \begin{align}
     \left[b,b^\dagger\right] = \left[(a+\kappa),(a+\kappa)^\dagger\right] = \mathbbm{I}.
   \end{align}
   Then the hamiltonian with a constant electric field $H^\prime$ is,
   \begin{align}
    \begin{split}
      H^\prime &= \hbar\omega\left((a^\dagger+\kappa)(a+\kappa)-\kappa(a+a^\dagger)-\kappa^2
      + \frac{1}{2}-\frac{q\mathcal{E}}{2}
      \sqrt{\frac{2\hbar}{m\omega}}\left(a+a^\dagger\right)\right)  \\
      &=\hbar\omega\left(b^\dagger b-\kappa^2
      + \frac{1}{2}-\left(\kappa+\frac{q\mathcal{E}}{2}
      \sqrt{\frac{2\hbar}{m\omega}}\right)\left(a+a^\dagger\right)\right)
    \end{split}
   \end{align}.
   Setting $\kappa$ as,
   \begin{align}
    \kappa = -\frac{q\mathcal{E}}{2}
    \sqrt{\frac{2\hbar}{m\omega}},\,\,\,
    \kappa+\frac{q\mathcal{E}}{2}
    \sqrt{\frac{2\hbar}{m\omega}} = 0.
   \end{align}
   Then $H^\prime$ is,
   \begin{align}
    H^\prime = \hbar\omega\left(b^\dagger b-\kappa^2
    + \frac{1}{2}\right).
   \end{align}
  %  If $H^\prime $ and $H$ are commute, they are same eigenfunction simultaneously.
  % \begin{align}
  %   \begin{split}
  %     \left[H,H^\prime\right] 
  %     &= \left[H,H-\frac{q\mathcal{E}}{2}\sqrt{\frac{2\hbar}{m\omega}}
  %     \left(a+a^\dagger\right)\right] 
  %     =-\frac{q\mathcal{E}}{2}\sqrt{\frac{2\hbar}{m\omega}}\left(\left[H,
  %     a\right]+\left[H,a^\dagger\right] \right).
  %   \end{split}
  % \end{align}
  % The commtutors between $H$ and $a$,$a^\dagger$ are,
  % \begin{align}
  %   \begin{split}
  %     \left[H,a\right] &= \left[a^\dagger a,a\right] = \left[a^\dagger,a\right] a = -a  \\
  %     \left[H,a^\dagger\right] &= \left[a^\dagger a,a^\dagger\right] = a^\dagger\left[a,a^\dagger\right]  = a^\dagger.
  %   \end{split}
  % \end{align}
  %  \begin{align}
  %   \begin{split}
  %     \left[H,H^\prime\right] 
  %     &= \hbar^2\omega^2\left[\left(aa^\dagger-\frac{1}{2}\right),
  %     \left(b^\dagger b-\kappa^2
  %     + \frac{1}{2}\right)\right] = aa^\dagger b^\dagger b-b^\dagger baa^\dagger.
  %   \end{split}
  %  \end{align}
  %  Since $a$ and $b$ commute,
  %  \begin{align}
  %    \left[a,b^\dagger\right]=\left[a,a^\dagger\right]=\mathbbm{I}.
  %  \end{align}
  %  Therefore,
  %  \begin{align}
  %   \left[H,H^\prime\right] = aa^\dagger b^\dagger b-b^\dagger baa^\dagger.
  %  \end{align}
    We can write the eigenvalue eqaution with the new operator.
    \begin{align}
      H^\prime \psi^\prime_n = E^\prime_n\psi^\prime_n.
    \end{align}
    Or,
    \begin{align}
      \hbar\omega\left(b^\dagger b
    + \frac{1}{2}\right)\psi^\prime_n = \left(E^\prime_n+\kappa^2\right)\psi^\prime_n.
    \end{align}
  %----------------------------------------------
  %  or,
  %  \begin{align}\label{eq:1-4}
  %   H = \hbar\omega\left(a a^\dagger - \frac{1}{2}\right)
  %   -q\mathcal{E}\sqrt{\frac{2\hbar}{m\omega}}
  %   \left(\frac{a+a^\dagger}{2}\right).
  %  \end{align}
  %  If $\psi_n$ is the eigenvector and $E_n$ is the eigenvalue of $\psi_n$, 
  %  the Schr\"odinger equation is,
  %  \begin{align}
  %   -\frac{\hbar}{2m}\frac{\partial^2\psi_n}{\partial x^2}
  %   +\frac{1}{2}m\omega^2x^2\psi_n
  %   -q\mathcal{E}x\psi_n = E_n\psi_n,\,\,\, 
  %   E_n = \left(\frac{1}{2}+n\right)\hbar\omega+E_m.
  %  \end{align}
  %  $E_m$ is a energy due to a constant electric field. Then we write the 
  %  reduced equation,
  %  \begin{align}\label{eq:1-5}
  %   -q\mathcal{E}x\psi_n = -q\mathcal{E}\sqrt{\frac{2\hbar}{m\omega}}
  %   \left(\frac{a+a^\dagger}{2}\right)\psi_n = E_m\psi_n.
  %  \end{align}
  %  Define $\kappa$ as,
  %  \begin{align}
  %    \kappa = -\frac{q\mathcal{E}}{2\hbar\omega}\sqrt{\frac{2\hbar}{m\omega}}
  %    = -\frac{1}{\omega}\frac{q\mathcal{E}}{\sqrt{2\hbar m\omega}}.
  %  \end{align}
  %  Then Eq.~\eqref{eq:1-4} is,
  %  \begin{align}
  %    H = \hbar\omega\left[a^\dagger a-\kappa\left(a+a^\dagger\right)
  %    +\frac{1}{2}\right]
  %    = \hbar\omega\left[(a^\dagger - \kappa)(a - \kappa) -\kappa^2.
  %    +\frac{1}{2}\right]
  %  \end{align}
  %  Now we define new operators from ladder operators,
  %  \begin{align}
  %    b = a - \kappa,\,\,\,
  %    b^\dagger = a^\dagger - \kappa.
  %  \end{align}
  %  The hamiltonian can be repersented by new operators.
  %  \begin{align}
  %    H = \hbar\omega\left( b^\dagger b -\left(\kappa^2-\frac{1}{2}\right)\right).
  %  \end{align}
  %  The Schr\"odinger equation and reduced equation from Eq.~\eqref{eq:1-2} 
  %  and \eqref{eq:1-5} are,
  %  \begin{align}
  %    \begin{split}
  %     &\hbar\omega\left( b^\dagger b 
  %     -\left(\kappa^2-\frac{1}{2}\right)\right)\psi_n = E_n\psi_n \\
  %     &\hbar\omega\kappa\left(b+b^\dagger+2\kappa\right)\psi_n = E_m\psi_n
  %     ,\,\,\,E_n = \left(\frac{1}{2}+n\right)\hbar\omega+E_m.
  %    \end{split}
  %  \end{align}
  \item[(3)] 
  \end{itemize}

\vspace{0.5cm}

\noindent \textbf{Problem 2.} 
The generating function $S(x,t)$ for the Hermite polynomial $H_n(x)$
is defined as 
\begin{align}
S(x,t) = e^{x^2-(t-x)^2} = e^{-t^2 + 2 t x} = \sum_{n=0}^\infty
  \frac{H_n(x)}{n!} t^n.  
\label{eq:1}
\end{align}
\begin{itemize}
\item[(1)] Using this generating function, derive the Hermite
  differential equation. 
\item[(2)] Derive the following formula from Eq.~\eqref{eq:1}:
  \begin{align}
H_n(x) = (-1)^n e^{x^2} \frac{d^n}{dx^n} e^{-x^2}    ,
  \end{align}
which is called the Rodrigues representation of the Hermite
polynomial. 
\item[(3)] Using Eq.~\eqref{eq:1}, derive the orthogonal relation of
  the Hermite polynomials
  \begin{align}
    \int_{-\infty}^\infty e^{-x^2} H_n(x) H_m(x) dx = 2^n \sqrt{\pi}
    n! \delta_{nm}.
  \end{align}
\item[(4)] Prove that
  \begin{align}
    \left(2x-\frac{d}{dx}\right)^n 1 = H_n(x),
  \end{align}
\item[(5)] Prove
  \begin{align}
    \int_{-\infty}^\infty x e^{-x^2} H_n(x) H_m(x) dx = \sqrt{\pi}
    2^{n-1} n!\delta_{m,n-1} + \sqrt{\pi} 2^n (n+1)! \delta_{m,n+1}.
  \end{align}
\item[(6)] Prove
  \begin{align}\label{eq:2}
    \int_{-\infty}^\infty x^2 e^{-x^2} H_n(x) H_n(x) dx = 
\sqrt{\pi} 2^n n! \left(n+\frac12\right).
  \end{align}
\end{itemize}

\noindent \textbf{Answer : }
\begin{itemize}
\item[(1)] The Hermite differential eqaution is,
\begin{align}\label{eq:2-1}
  \frac{d^2y}{dx^2}-2x\frac{dy}{dx}+\lambda y = 0,
\end{align}
$\lambda$ is a any constant. Derivatives for $x$ of generating function $S$ are,
\begin{align}\label{eq:2-2}
  \begin{split}
    &\frac{dS}{dx} = 2tS = \sum^\infty_{n=0}\frac{H^\prime_n(x)}{n!}t^n  \\
    &\frac{d^2S}{dx^2} = 4t^2S=\sum^\infty_{n=0}\frac{H^{\prime\prime}_n(x)}{n!}t^n.
  \end{split}
\end{align}
And,
\begin{align}\label{eq:2-3}
  \frac{dS}{dt} = 2(-t+x)S = \sum^\infty_{n=0}\frac{H_n(x)}{n!}nt^{n-1}
  =-\frac{dS}{dx} +2xS.
\end{align}
From Eq.~\eqref{eq:2-2},
\begin{align*}
  \begin{split}
    \frac{dS}{dx} &= \frac{1}{2t}\frac{d^2S}{dx^2}
    =\frac{1}{2}\sum^\infty_{n=0}\frac{H^{\prime\prime}_n(x)}{n!}t^{n-1}  \\
    2xS &= 2x\frac{1}{2t}\frac{dS}{dt} 
    =x\sum^\infty_{n=0}\frac{H^{\prime}_n(x)}{n!}t^{n-1}
  \end{split}
\end{align*}
Then Eq.~\eqref{eq:2-3} is,
\begin{align*}
  \sum^\infty_{n=0}\frac{H_n(x)}{n!}nt^{n-1}
  =-\frac{1}{2}\sum^\infty_{n=0}\frac{H^{\prime\prime}_n(x)}{n!}t^{n-1}
  +x\sum^\infty_{n=0}\frac{H^{\prime}_n(x)}{n!}t^{n-1}.
\end{align*}
Finally we obtain,
\begin{align*}
  \sum^\infty_{n=0}\left(\frac{H^{\prime\prime}_n(x)
  -2xH^{\prime}_n(x)+2nH_n(x)}{n!}t^{n-1}\right)=0.
\end{align*}
It is true for any $t$ when all coefficient is zero. So,
\begin{align}\label{eq:2-4}
  H^{\prime\prime}_n(x)
  -2xH^{\prime}_n(x)+2nH_n(x)=0.
\end{align}
\item[(2)] From Eq.~\eqref{eq:1}, 
\begin{align*}
  e^{-(t-x)^2+x^2}=e^{x^2}e^{-(t-x)^2}.
\end{align*}
And, 
\begin{align*}
  e^{x^2}e^{-(t-x)^2} = \left.e^{x^2}\sum^\infty_{n=0}\frac{t^n}{n!}
  \frac{d^n}{dt^n}e^{-(t-x)^2}\right|_{t=0} 
  = \sum^\infty_{n=0}\frac{H_n(x)}{n!}t^n.
\end{align*}
Since the series representation is unique,
\begin{align}\label{eq:2-5}
  H_n(x)=e^{x^2}\left.\frac{d^n}{dt^n}e^{-(t-x)^2}\right|_{t=0}.
\end{align}
If we regard $t$ as just the parameter, Eq.~\eqref{eq:2-5} is true for 
any $t$. 
A differential part of a LHS is,
\begin{align*}
  \left.\frac{d^n}{dt^n}e^{-(t-x)^2}\right|_{t=0}
  =\left.(-1)^n\frac{d^n}{dx^n}e^{-(t-x)^2}\right|_{t=0}
  =(-1)^n\frac{d^n}{dx^n}e^{-x^2}
\end{align*}
Finally we obtain,
\begin{align}\label{eq:2-5-1}
  H_n(x) = (-1)^ne^{x^2}\frac{d^n}{dx^n}e^{-x^2}.
\end{align}
\item[(3)]First, when $t=1$ and $t=-1$, Eq.~\eqref{eq:1} is,
\begin{align*}
  \begin{split}
    e^{2x-1}&=\sum^\infty_{n=0}\frac{H_n(x)}{n!} \\
    e^{-2x-1}&=\sum^\infty_{n=0}(-1)^n\frac{H_n(x)}{n!}.
  \end{split}
\end{align*} 
For checking the value $2^n \sqrt{\pi}n!$, consider a integration as,
\begin{align}\label{eq:2-6}
  \int_{-\infty}^{\infty}e^{-x^2}H_m(x)e^{2x-1}\,dx
  =\sum_{n=0}^\infty\frac{1}{n!}\int_{-\infty}^{\infty}e^{-x^2}H_m(x)H_n(x)\,dx
\end{align}
The RHS is,
\begin{align*}
  \int_{-\infty}^{\infty}e^{-x^2}H_m(x)e^{2x-1}\,dx 
  =(-1)^m\int_{-\infty}^{\infty}e^{2x-1}\frac{d^m}{dx^m}e^{-x^2}\,dx
\end{align*}
Using the integration by part to RHS,
\begin{align*}
  \begin{split}
    \int_{-\infty}^{\infty}e^{2x-1}\frac{d^m}{dx^m}e^{-x^2}\,dx  
    &=\left.e^{2x-1}\frac{d^{m-1}}{dx^{m-1}}e^{-x^2}\right|^\infty_{-\infty}
    -\int_{-\infty}^{\infty}2e^{2x-1}\frac{d^{m-1}}{dx^{m-1}}e^{-x^2}\,dx \\
    &=-2\int_{-\infty}^{\infty}e^{2x-1}\frac{d^{m-1}}{dx^{m-1}}e^{-x^2}\,dx 
  \end{split}
\end{align*}
Repetition of the integration by part for $m$ times conserves the 
form in the integration multiplying $(-2)^m$. 
\begin{align*}
  \int_{-\infty}^{\infty}e^{2x-1}\frac{d^m}{dx^m}e^{-x^2}\,dx  
  =(-2)^m\int_{-\infty}^{\infty}e^{2x-1}e^{-x^2}\,dx=(-2)^m\sqrt{\pi}. 
\end{align*}
Therefore, Eq.~\eqref{eq:2-6} is,
\begin{align*}
  \sum_{n=0}^\infty\frac{1}{n!}\int_{-\infty}^{\infty}e^{-x^2}H_m(x)H_n(x)\,dx
  =(-1)^m(-2)^m\sqrt{\pi}=2^m\sqrt{\pi}.
\end{align*}
Now, check the orthogonality. From Eq.~\eqref{eq:2-4},
\begin{align*}
  e^{x^2}\frac{d}{dx}\left(e^{-x^2}H^\prime_n(x)\right)+2nH_n(x)=0.
\end{align*}
Multiplying $e^{-x^2}$,
\begin{align*}
  \int_{-\infty}^\infty\frac{d}{dx}\left(e^{-x^2}H^\prime_n(x)\right)H_m(x)\,dx
  +2n\int_{-\infty}^\infty e^{-x^2}H_n(x)H_m(x)\,dx=0.
\end{align*}
Change $m$ and $n$ each other, subtract the previous one,
\begin{align*}
  \int_{-\infty}^\infty\frac{d}{dx}\left(e^{-x^2}H^\prime_n\right)H_m
  -\frac{d}{dx}\left(e^{-x^2}H^\prime_m\right)H_n\,dx
  +2(n-m)\int_{-\infty}^\infty e^{-x^2}H_n(x)H_m(x)\,dx=0.
\end{align*}
Integrations by part of first two terms are, 
\begin{align*}
  \begin{split}
    &\int_{-\infty}^\infty\frac{d}{dx}\left(e^{-x^2}H^\prime_n\right)H_m
    -\frac{d}{dx}\left(e^{-x^2}H^\prime_m\right)H_n\,dx \\
    &=\left.e^{-x^2}H^\prime_nH_m\right|_{-\infty}^\infty
    -\int_{-\infty}^\infty e^{-x^2}H^\prime_nH^\prime_m\,dx
    -\left.e^{-x^2}H_nH^\prime_m\right|_{-\infty}^\infty
    +\int_{-\infty}^\infty e^{-x^2}H^\prime_nH^\prime_m\,dx \\
    &=0.
  \end{split}
\end{align*}
It means that,
\begin{align*}
  2(n-m)\int_{-\infty}^\infty e^{-x^2}H_n(x)H_m(x)\,dx=0.
\end{align*}
If $n\neq m$, the integration is a zero. For this reason,
\begin{align*}
  \sum_{n=0}^\infty\frac{1}{n!}\int_{-\infty}^{\infty}e^{-x^2}H_m(x)H_n(x)\,dx
  =\frac{1}{m!}\int_{-\infty}^{\infty}e^{-x^2}H_m(x)H_m(x)\,dx
  =2^m\sqrt{\pi}.
\end{align*}
Finally we obtain,
\begin{align}\label{eq:2-7}
  \int_{-\infty}^\infty e^{-x^2} H_n(x) H_m(x) dx = 2^nn! \sqrt{\pi}
   \delta_{nm}.
\end{align}
\item[(4)] {\it Proof}. We use the mathematical induction. 
If $n=0$ and $n=1$, then,
\begin{align*}
  H_0(x) = 1,\,\,\,H_1(x)=2x. 
\end{align*}
The statement is true. Suppose it is true:
\begin{align*}
  H_k(x) = \left(2x-\frac{d}{dx}\right)^k 1.
\end{align*}
Then,
\begin{align*}
  \begin{split}
    H_{k+1}(x) &= \left(2x-\frac{d}{dx}\right)\left(2x-\frac{d}{dx}\right)^k 1
    =\left(2x-\frac{d}{dx}\right)H_k(x)
  \end{split}
\end{align*}
From Eq.~\eqref{eq:2-5-1},
\begin{align*}
  \begin{split}
    \left(2x-\frac{d}{dx}\right)H_k(x)
    &=\left(2x-\frac{d}{dx}\right)(-1)^ke^{x^2}\frac{d^k}{dx^k}e^{-x^2}  \\
    &=(-1)^k 2x e^{x^2}\frac{d^k}{dx^k}e^{-x^2}
    -(-1)^k 2x e^{x^2}\frac{d^k}{dx^k}e^{-x^2}
    -(-1)^k e^{x^2}\frac{d^{k+1}}{dx^{k+1}}e^{-x^2}  \\
    &=(-1)^{k+1}= e^{x^2}\frac{d^{k+1}}{dx^{k+1}}e^{-x^2}
    =H_{k+1}(x).
  \end{split}
\end{align*}
Hence this statement is true for $n=k+1$. \\
By mathematical induction, 
this statement is true for any $n$. ~\hfill $\square$
\item[(5)] {\it Proof}. Set $I_{nm}$,
\begin{align*}
  \begin{split}
    I_{nm} &= \int_{-\infty}^\infty xe^{-x^2}H_n(x)H_m(x) \,dx  \\
    &=-\left.\frac{1}{2}e^{-x^2}H_n(x)H_m(x)\right|_{-\infty}^\infty
    +\frac{1}{2}\int_{-\infty}^\infty 
    e^{-x^2}\left(H^\prime_n(x)H_m(x)+H_n(x)H^\prime_m(x)\right) \,dx \\
    &=\frac{1}{2}\int_{-\infty}^\infty 
    e^{-x^2}H^\prime_n(x)H_m(x) \,dx+
    \frac{1}{2}\int_{-\infty}^\infty 
    e^{-x^2}H_n(x)H^\prime_m(x) \,dx.
  \end{split}
\end{align*}
From Eq.~\eqref{eq:2-5-1},
\begin{align}\label{eq:2-8}
  \begin{split}
    H^\prime_n(x) &= \frac{d}{dx}\left( (-1)^ne^{x^2}
    \frac{d^n}{dx^n}e^{-x^2} \right)
    =(-1)^n\left( 2xe^{x^2}\frac{d^n}{dx^n}e^{-x^2}
    +e^{x^2}\frac{d^{n+1}}{dx^{n+1}}e^{-x^2} \right)  \\
    &=2xH_n(x)-H_{n+1}(x).
  \end{split}
\end{align}
Then $I_{nm}$ is,
\begin{align*}
  \begin{split}
    I_{nm} &= \frac{1}{2}\int_{-\infty}^\infty 
    e^{-x^2}(2xH_n(x)-H_{n+1}(x))H_m(x) \,dx+
    \frac{1}{2}\int_{-\infty}^\infty 
    e^{-x^2}H_n(x)(2xH_m(x)-H_{m+1}(x)) \,dx  \\
    &= 2I_{nm}
    -\frac{1}{2}\int_{-\infty}^\infty 
    e^{-x^2}H_{n+1}(x)H_m(x) \,dx
    -\frac{1}{2}\int_{-\infty}^\infty 
    e^{-x^2}H_n(x)H_{m+1}(x) \,dx.
  \end{split}
\end{align*}
Hence,
\begin{align*}
  I_{nm}=\frac{1}{2}\left(\int_{-\infty}^\infty 
  e^{-x^2}H_{n+1}(x)H_m(x) \,dx
  +\int_{-\infty}^\infty 
  e^{-x^2}H_n(x)H_{m+1}(x) \,dx\right).
\end{align*}
From Eq.~\eqref{eq:2-7}, we obtain that,
\begin{align}\label{eq:2-9}
  \begin{split}
    I_{nm}&=\frac{1}{2}\left(2^{n+1}\sqrt{\pi}(n+1)!\delta_{n+1,m}
    +2^n\sqrt{\pi}n!\delta_{n,m+1} \right)  \\
    &=2^n\sqrt{\pi}(n+1)!\delta_{n+1,m}
    +2^{n-1}\sqrt{\pi}n!\delta_{n,m+1}.
  \end{split}
\end{align}
Therefore the statement is true. ~\hfill $\square$
\item[(6)] {\it Proof}. Eq.~\eqref{eq:2} is,
\begin{align*}
  \int_{-\infty}^\infty x^2e^{-x^2}H_nH_n \,dx  
  =-\left.\frac{1}{2}xe^{-x^2}H_nH_n\right|_{-\infty}^\infty
  +\int_{-\infty}^\infty xe^{-x^2}H_n^\prime H_n \,dx  
  -\frac{1}{2}\int_{-\infty}^\infty e^{-x^2}H_nH_n \,dx  .
\end{align*}
From Eq.~\eqref{eq:2-8}, the second term of the RHS is,
\begin{align*}
  \int_{-\infty}^\infty xe^{-x^2}H_n^\prime H_n \,dx  
  =\int_{-\infty}^\infty 2x^2e^{-x^2}H_n H_n \,dx 
  -\int_{-\infty}^\infty xe^{-x^2}H_{n+1}H_n \,dx  .
\end{align*}
Hence, 
\begin{align*}
  \begin{split}
    \int_{-\infty}^\infty x^2e^{-x^2}H_nH_n \,dx  
    &=\int_{-\infty}^\infty 2x^2e^{-x^2}H_n H_n \,dx 
    -\int_{-\infty}^\infty xe^{-x^2}H_{n+1}H_n \,dx 
    -\frac{1}{2}\int_{-\infty}^\infty e^{-x^2}H_nH_n \,dx \\
    &=\int_{-\infty}^\infty xe^{-x^2}H_{n+1}H_n \,dx 
    +\frac{1}{2}\int_{-\infty}^\infty e^{-x^2}H_nH_n \,dx.
  \end{split}
\end{align*}
From Eq.~\eqref{eq:2-9} and \eqref{eq:2-7},
\begin{align}\label{eq:2-10}
  \begin{split}
    \int_{-\infty}^\infty x^2e^{-x^2}H_nH_n \,dx  
    &=\sqrt{\pi}2^{n-1}n!\delta_{n+1,n-1}
    +\sqrt{\pi}2^{n}(n+1)!\delta_{n+1,n+1}
    -\sqrt{\pi}2^{n-1}n!  \\
    &=\sqrt{\pi}2^nn!\left(n+\frac{1}{2}\right).
  \end{split}
\end{align}
Therefore the statement is true. ~\hfill $\square$
\end{itemize}

\vspace{0.5cm}

\noindent \textbf{Problem 3.} 
Given the eigenfunctions and eigenenergies of the SHO, 
\begin{itemize}
\item[(1)] Compute the kinetic and potential energies at the $n^{th}$
  level.  Show that the results satisfy the virial theorem. 
\item[(2)] Show that the $n^{th}$ state of the SHO satisfies 
  \begin{align}
    \Delta x \Delta p = \left(n+\frac12\right)\hbar.
  \end{align}
\end{itemize}

\noindent \textbf{Answer : }
\begin{itemize}
  \item[(1)] The eigenvector and eigenfunction of the SHO are,
  \begin{align}\label{eq:3}
    \psi_n(x)=\psi^*_n(x)= (n!2^n)^{-\frac{1}{2}}
    \left(\frac{m\omega}{\hbar\pi}\right)^{\frac{1}{4}}
    \exp\left(-\frac{m\omega}{2\hbar}x^2\right)
    H_n\left(\sqrt{\frac{m\omega}{\hbar}}x\right),\,\,\,
    E_n=\left( n+\frac{1}{2} \right)\hbar\omega.
  \end{align}
  The expectation value of the kinetic energy is,
  \begin{align}
   \langle T_n\rangle = \frac{1}{2m}\int \psi^*_n p^2 \psi_n\,dx
   =\frac{\langle p^2\rangle}{2m}.
  \end{align}
  Since the expectation value of the kinetic energy is an integer multiple 
  of the square of momentum, we just calculate the expectation value of the 
  square of momentum. Using the integration by part,
  \begin{align}\label{eq:3-1}
    \langle p^2\rangle = -\hbar^2\int \psi^*_n 
    \frac{\partial^2 \psi_n}{\partial x^2} \,dx
    =\hbar^2\int\frac{\partial \psi^*_n}{\partial x} 
    \frac{\partial \psi_n}{\partial x} \,dx
  \end{align}
  Changing the variable,
  \begin{align}\label{eq:3-2}
    \sqrt{\frac{m\omega}{\hbar}}x = \xi,\,\,\,
    \frac{\partial \psi_n}{\partial x}
    =\frac{\partial \psi_n}{\partial \xi}
    \frac{\partial \xi}{\partial x}
    =\sqrt{\frac{m\omega}{\hbar}}
    \frac{\partial \psi_n}{\partial \xi}
  \end{align}  
  Then,
  \begin{align}
    \frac{\partial \psi_n}{\partial \xi}
    =(n!2^n)^{-\frac{1}{2}}
    \left(\frac{m\omega}{\hbar\pi}\right)^{\frac{1}{4}}
    (-\xi H_n\left(\xi\right) 
    +H^\prime_n\left(\xi\right))
    e^{-\frac{\xi^2}{2}}.
  \end{align}
  The integration of Eq.~\eqref{eq:3-1} is,
  \begin{align}
    \begin{split}
      \int\frac{\partial \psi^*_n}{\partial x} 
      \frac{\partial \psi_n}{\partial x} \,dx
      &= (n!2^n)^{-1}\sqrt{\frac{m\omega}{\hbar\pi}}
      \frac{m\omega}{\hbar}\int(-\xi H_n\left(\xi\right) 
      +H^\prime_n\left(\xi\right))^2 e^{-\xi^2}\,
      \sqrt{\frac{\hbar}{m\omega}}d\xi  \\
      &= (n!2^n)^{-1}\frac{m\omega}{\hbar\sqrt{\pi}}
      \int(-\xi H_n\left(\xi\right) 
      +H^\prime_n\left(\xi\right))^2 e^{-\xi^2}\,d\xi  \\
    \end{split}
  \end{align}
  From Eq.~\eqref{eq:2-8}
  \begin{align}
    \begin{split}
      \int(-\xi H_n\left(\xi\right) 
      +H^\prime_n\left(\xi\right))^2 e^{-\xi^2}\,d\xi
      &=\int(-\xi H_n\left(\xi\right) 
      + 2\xi H_n(\xi)-H_{n+1}(\xi))^2 e^{-\xi^2}\,d\xi  \\
      &=\int(\xi H_n\left(\xi\right) 
      -H_{n+1}(\xi))^2 e^{-\xi^2}\,d\xi \\
      &=\int(
      (\xi^2H_nH_n-2\xi H_nH_{n+1}+H_{n+1}H_{n+1})  
      e^{-\xi^2}\,d\xi.
    \end{split}
  \end{align}
  We can use Eq.~\eqref{eq:2-7}, \eqref{eq:2-9} and \eqref{eq:2-10} to calculate 
  this integration.
  \begin{align}
    \begin{split}
      \int \xi^2 H_n H_n e^{-\xi^2}\,dx 
      &= 2^nn!\sqrt{\pi}\left( n+\frac{1}{2} \right) \\  
      \int \xi H_n H_{n+1} e^{-\xi^2}\,dx 
      &= 2^n(n+1)!\sqrt{\pi}\\  
      \int  H_{n+1} H_{n+1} e^{-\xi^2}\,dx 
      &= 2^{n+1}(n+1)!\sqrt{\pi}
    \end{split}
  \end{align}
  Then,
  \begin{align}
    \begin{split}
      \int(-\xi H_n\left(\xi\right) 
      +H^\prime_n\left(\xi\right))^2 e^{-\xi^2}\,d\xi
      &= \sqrt{\pi}2^nn!
      \left(n+\frac{1}{2}-2(n+1)+2(n+1) \right) \\
      &= \sqrt{\pi}2^nn!\left( n+\frac{1}{2} \right).
    \end{split}
  \end{align}
  Therefore the expectation value of the square of the momentum is,
  \begin{align}\label{eq:3-ps}
    \langle p^2\rangle = \hbar^2\,(n!2^n)^{-1}\frac{m\omega}{\hbar\sqrt{\pi}}\,
    \sqrt{\pi}2^nn!\left( n+\frac{1}{2} \right)
    =\hbar m\omega\left( n+\frac{1}{2} \right).
  \end{align}
  We obtain the expectation value of the kinetic energy.
  \begin{align}\label{eq:3-3}
    \langle T_n\rangle=\frac{\langle p^2\rangle}{2m}
    =\frac{1}{2}\hbar \omega\left( n+\frac{1}{2} \right).
  \end{align}
  The expectation value of the potential energy is,
  \begin{align}
    \langle V_n\rangle = \int \psi^*_n \frac{1}{2}m\omega^2x^2 \psi_n \,dx
    =\frac{1}{2}m\omega^2\int \psi^*_n x^2 \psi_n \,dx
    =\frac{1}{2}m\omega^2\langle x^2\rangle.
  \end{align}
  From Eq.~\eqref{eq:3-2}, the expectation value of the square of $x$ is,
  \begin{align}
    \begin{split}
      \langle x^2\rangle &= \int \psi^*_n x^2 \psi_n \,dx
      =\langle x^2\rangle ={\left(\frac{\hbar}{m\omega}\right)}^{\frac{3}{2}} 
      \int \psi^*_n(\xi) \xi^2 \psi_n(\xi) \,d\xi \\
      &= (n!2^n)^{-1}\sqrt{\frac{m\omega}{\hbar\pi}}
      {\left(\frac{\hbar}{m\omega}\right)}^{\frac{3}{2}} 
      \int \xi^2 H_n(\xi) H_n(\xi) e^{-\xi^2}\,d\xi
    \end{split}    
  \end{align}
  The integration part can be calculated by Eq.~\eqref{eq:2-10}.
  \begin{align}
    \int \xi^2 H_n(\xi) H_n(\xi) e^{-\xi^2}\,d\xi
    =\sqrt{\pi}2^nn!\left( n+\frac{1}{2} \right).
  \end{align}
  So, $\langle x^2\rangle$ is,
  \begin{align}\label{eq:3-xs}
    \langle x^2\rangle = 
    \frac{\hbar}{m\omega}
    \left( n+\frac{1}{2}\right).
  \end{align}
  Finally we obtain the expectation value of the potential energy.
  \begin{align}\label{eq:3-4}
    \langle V_n\rangle=\frac{1}{2}m\omega^2(n!2^n)^{-1}
    \sqrt{\frac{m\omega}{\hbar\pi}}
    {\left(\frac{\hbar}{m\omega}\right)}^{\frac{3}{2}} 
    \sqrt{\pi}2^nn!\left( n+\frac{1}{2} \right)
    =\frac{1}{2}\hbar\omega
    \left( n+\frac{1}{2} \right).
  \end{align}
  Let us confirm that the results satisfy the virial theorem. 
  In this condition the virial theorem is,
  \begin{align}
    \left\langle x\frac{\partial V}{\partial x}\right\rangle 
    = 2\left\langle T\right\rangle .
  \end{align}
  Substituting Eq.~\eqref{eq:3-3} and \eqref{eq:3-4},
  \begin{align}
  \left\langle x\frac{\partial V}{\partial x}\right\rangle 
  =m\omega^2\int\psi^*_nx^2\psi_n\,dx = 2\left\langle V_n\right\rangle 
  =2\left\langle T_n\right\rangle.
  \end{align}
  The results satisfy the virial theorem.
  \item[(2)] Let us calculate $\Delta x$ and $\Delta p$. 
  From the definition, $\Delta x$ and $\Delta p$ are,
  \begin{align}
      \Delta x = \sqrt{\left\langle x^2\right\rangle - {\langle x\rangle}^2  },\,\,\,
      \Delta p = \sqrt{\left\langle p^2\right\rangle - {\langle p\rangle}^2  }.
  \end{align}
  $\langle x\rangle$ is,
  \begin{align}
    \begin{split}
      \langle x\rangle = \int x\psi^*\psi\,dx 
      = (n!2^n)^{-1}\sqrt{\frac{m\omega}{\hbar\pi}}\left(\frac{\hbar}{m\omega}\right)
      \int \xi H_nH_n e^{-\xi^2}\,d\xi.
    \end{split}
  \end{align}
  Since the integrated term is a even function and integration interval is symmetric,
  the integration is a zero. Therefore,
  \begin{align}
    \langle x\rangle = 0.
  \end{align}
  With Eq.~\eqref{eq:3-xs} $\Delta x$ is,
  \begin{align}
    \Delta x = \sqrt{\left\langle x^2\right\rangle}
    = \sqrt{\frac{\hbar}{m\omega}
    \left( n+\frac{1}{2}\right)}.
  \end{align}
  To calculate $\Delta p$,
  $\langle p\rangle$ is,
  \begin{align}
    \langle p\rangle = \int \psi^*p\psi\,dx
    = -i\hbar\int\psi^*_n
    \frac{\partial \psi_n}{\partial x} \,dx.
  \end{align}
  The integration by part is,
  \begin{align}
    \begin{split}
      \langle p\rangle=-i\hbar\int\psi^*_n
      \frac{\partial \psi_n}{\partial x} \,dx
      =i\hbar\int\frac{\partial \psi^*_n}{\partial x}
      \psi_n \,dx.
    \end{split}
  \end{align}
  From Eq.~\eqref{eq:3}, $\psi = \psi^*$. So,
  \begin{align}
    \langle p\rangle=i\hbar\int\frac{\partial \psi^*_n}{\partial x}
    \psi_n \,dx=i\hbar\int\psi^*_n
    \frac{\partial \psi_n}{\partial x}\,dx = -\langle p\rangle.
  \end{align}
  Therefore,
  \begin{align}
    \langle p\rangle = 0.
  \end{align}
  With Eq.~\eqref{eq:3-ps} $\Delta p$ is,
  \begin{align}
    \Delta p = \sqrt{\left\langle p^2\right\rangle}
    = \sqrt{\hbar m\omega\left( n+\frac{1}{2} \right)}.
  \end{align}
  Finally $\Delta x\Delta p$ si,
  \begin{align}
    \Delta x\Delta p = \sqrt{\frac{\hbar}{m\omega}
    \left(n+\frac{1}{2}\right)}\sqrt{\hbar m\omega
    \left(n+\frac{1}{2}\right)}
    =\hbar\left(n+\frac{1}{2}\right).
  \end{align}
\end{itemize}

\vspace{1cm}

\noindent \textbf{Problem 4.}
If a wavefunction desribes a mixed state of the eigenstates of the SHO
given as 
\begin{align}
  \psi(x,t) = \frac1{\sqrt{2}}[\psi_0(x,t) + \psi_1(x,t)] ,
\end{align}
\begin{itemize}
\item[(1)] Investigate how the probability density changes in time. 
\item[(2)] Prove the following relations 
  \begin{align}
\langle E\rangle &= \langle H\rangle =\hbar \omega    ,\cr
\langle x \rangle &= \frac1{\sqrt{2\alpha}}\cos\omega t,\cr
\langle p \rangle &= -\sqrt{\frac{\alpha}{2}}\hbar \sin\omega t,
 \end{align}
where $\alpha = \sqrt{m\omega/\hbar}$.
 \item[(3)] If 
   \begin{align}
 \psi(x,t) = \frac1{\sqrt{2}}[e^{i\delta_0} \psi_0(x,t) + e^{i\delta}
     \psi_1(x,t)],     
   \end{align}
discuss the effects of the phase factors $\delta_0$ and $\delta$ on
$\langle x\rangle$ and $\langle p\rangle$.
\end{itemize}
 \vspace{1cm}

\noindent \textbf{Problem 5.}
Derive the wavefunction in momentum space, which corresponds to the
eigenfunctions for the SHO in coordinates, $\psi_n(x)$. 

\noindent \textbf{Answer : }
We already know that the solution of the Schr\"odinger equation in the 
coordinate space is,
\begin{align}\label{eq:6-1}
  \psi_n(x)=(n!2^n)^{-\frac{1}{2}}
  \left(\frac{m\omega}{\hbar\pi}\right)^{\frac{1}{4}}
  \exp\left(-\frac{m\omega}{2\hbar}x^2\right)
  H_n\left(\sqrt{\frac{m\omega}{\hbar}}x\right).
\end{align}
It satisfies the equation,
\begin{align}\label{eq:6-2}
  -\frac{\hbar^2}{2m}\frac{\partial^2\psi_n}{\partial x^2}
  +\frac{1}{2}m\omega^2x^2\psi_n = E_n\psi_n,\,\,\,
  E_n = \left(n+\frac{1}{2}\right)\hbar\omega.
\end{align}
And the wave function in the momentum space safisfies that,
\begin{align}\label{eq:6-3}
  \frac{p^2}{2m}\phi_n-\frac{1}{2}m\omega^2\hbar^2
  \frac{\partial^2\phi_n}{\partial p^2} = \mathcal{E}_n\phi_n. 
\end{align}
Multiplying $ 1/{m^2\omega^2}$ to Eq.~\eqref{eq:7-3},
\begin{align}
  \frac{p^2}{2m^3\omega^2}\phi_n-\frac{\hbar^2}{2m}
  \frac{\partial^2\phi_n}{\partial p^2} 
  = \frac{\mathcal{E}_n}{m^2\omega^2}\phi_n. 
\end{align}
Set $1/m^4\omega^2=\omega^2_p$. Then,
\begin{align}
  -\frac{\hbar^2}{2m}\frac{\partial^2\phi_n}{\partial p^2} 
  +\frac{1}{2}m\omega^2_pp^2\phi_n
  = m^2\omega^2_p\mathcal{E}_n\phi_n. 
\end{align}
Since the form of the equation corresponds to Eq.~\eqref{eq:6-2},
the form of the solution will be the same.
Comparing the two equations,
\begin{align}
  \omega^2x^2 \longrightarrow \omega^2_pp^2 ,\,\,\,
  E_n \longrightarrow m^2\omega^2_p\mathcal{E}_n.
\end{align}
Then $\phi_n(p)$ and $\mathcal{E}$ are,
\begin{align}
  \phi_n(p)=(n!2^n)^{-\frac{1}{2}}
  \left(\frac{m\omega_p}{\hbar\pi}\right)^{\frac{1}{4}}
  \exp\left(-\frac{m\omega_p}{2\hbar}p^2\right)
  H_n\left(\sqrt{\frac{m\omega^2_p}{\hbar\omega}}p\right),\,\,\,
  \mathcal{E}_n = \left(n+\frac{1}{2}\right)m^2\hbar\omega^3.
\end{align}
\vspace{1cm}

\noindent \textbf{Problem 6.}
At $t=0$, the wavefunction for a state is described by
\begin{align}
\psi(x,0) = \sum_n A_n u_n(x) =
  \left(\frac{\alpha^2}{\pi}\right)^{1/4} e^{-\alpha^2(x-a)^2/2}  .
\end{align}
show that after some time $t$, the probability density changes in time
as 
\begin{align}
|\psi(x,t)|^2  =  \left(\frac{\alpha^2}{\pi}\right)^{1/4}e^{-\alpha^2
  (x-a\cos\omega t)^2}
\end{align}
and discuss the result. 


\noindent \textbf{Answer : }
The wave function in the momentum space $\phi(p,0)$ at $t=0$ is,
\begin{align}
  \phi(p,0) = \frac{1}{\sqrt{2\pi\hbar}}
  \int e^{-\frac{i}{\hbar}px}\psi(x,0)\,dx
  = \frac{1}{\sqrt{2\pi\hbar}}\left(\frac{\alpha^2}{\pi}\right)^{1/4} 
  \int \exp{\left(-\frac{i}{\hbar}px-\frac{\alpha^2}{2}(x-a)^2\right)} \,dx.
\end{align}
The exponential of the integrated term can be represented into the 
complete square form about $x$.
\begin{align}
  \begin{split}
    -\frac{i}{\hbar}px-\frac{\alpha^2}{2}(x-a)^2
    &=-\frac{\alpha^2}{2}x^2+\left(\alpha^2 a
    -\frac{i}{\hbar}p\right)x-\frac{\alpha^2a^2}{2}\\
    &=-\frac{\alpha^2}{2}\left(x-\left(a-\frac{i}
    {\alpha^2\hbar}p\right)\right)^2
    +\frac{\alpha^2}{2}\left(a-\frac{i}{\alpha^2\hbar}p\right)^2
    -\frac{\alpha^2a^2}{2}.
  \end{split}
\end{align}
Since the last two terms in the RHS are independent on $x$,
we just integrate the first term in the RHS and 
it is the gaussian integration.
\begin{align}
  \begin{split}
    \phi(p,0) &= \frac{1}{\sqrt{2\pi\hbar}}\exp{\left(\frac{\alpha^2}{2}
    \left(a-\frac{i}{\alpha^2\hbar}p\right)^2
    -\frac{\alpha^2a^2}{2}\right)}
    \int\exp{\left(-\frac{\alpha^2}{2}\left[x-\left(a-\frac{i}
    {\alpha^2\hbar}p\right)\right]^2\right)}\,dx  \\
    &=\frac{1}{\sqrt{2\pi\hbar}}\sqrt{\frac{2\pi}{\alpha^2}}
    \exp{\left(\frac{\alpha^2}{2}
    \left(a-\frac{i}{\alpha^2\hbar}p\right)^2
    -\frac{\alpha^2a^2}{2}\right)}  \\
    &= \frac{1}{\alpha\sqrt{\hbar}}
    \exp{\left(-\frac{p^2}{2\alpha^2\hbar^2}-i\frac{a}{\hbar}p\right)}.
  \end{split}
\end{align}
The wave function in the momentum space $\phi(p,t)$ at any time $t$ that is not a zero is,
\begin{align}
  \begin{split}
    \phi(p,t) &= \exp{\left({-i\omega t}\right)}\phi(p,0)
    = \frac{1}{\alpha\sqrt{\hbar}}
    \exp{\left(-i\omega t-\frac{p^2}{2\alpha^2\hbar^2}
    -i\frac{a}{\hbar}p\right)} \\
    &=\frac{1}{\alpha\sqrt{\hbar}}
    \exp{\left(  \right)}.
  \end{split}
\end{align}
To obtain the wave function in the coordinate space at time $t$ that is not a zero,
we operate Fourier transform into the $\phi(p,t)$.

%The wave function in the momentum space $\phi(p,0)$ at $t=0$ is,
%\begin{align}
%  \phi(p,0) = \frac{1}{\sqrt{2\pi\hbar}}
%  \int e^{-\frac{i}{\hbar}px}\psi(x,0)\,dx
%  = \frac{1}{\sqrt{2\pi\hbar}}\left(\frac{\alpha^2}{\pi}\right)^{1/4} 
%  \int \exp{\left(-\frac{i}{\hbar}px-\frac{\alpha^2}{2}(x-a)^2\right)} \,dx.
%\end{align}
%The exponential of the integrated term can be represented into the 
%complete square form about $x$.
%\begin{align}
%  \begin{split}
%    -\frac{i}{\hbar}px-\frac{\alpha^2}{2}(x-a)^2
%    &=-\frac{\alpha^2}{2}x^2+\left(\alpha^2 a
%    -\frac{i}{\hbar}p\right)x-\frac{\alpha^2a^2}{2}\\
%    &=-\frac{\alpha^2}{2}\left(x-\left(a-\frac{i}
%    {\alpha^2\hbar}p\right)\right)^2
%    +\frac{\alpha^2}{2}\left(a-\frac{i}{\alpha^2\hbar}p\right)^2
%    -\frac{\alpha^2a^2}{2}.
%  \end{split}
%\end{align}
%Since the last two terms in the RHS are independent on $x$,
%we just integrate the first term in the RHS and 
%it is the gaussian integration.
%\begin{align}
%  \begin{split}
%    \phi(p,0) &= \frac{1}{\sqrt{2\pi\hbar}}\exp{\left(\frac{\alpha^2}{2}
%    \left(a-\frac{i}{\alpha^2\hbar}p\right)^2
%    -\frac{\alpha^2a^2}{2}\right)}
%    \int\exp{\left(-\frac{\alpha^2}{2}\left[x-\left(a-\frac{i}
%    {\alpha^2\hbar}p\right)\right]^2\right)}\,dx  \\
%    &=\frac{1}{\sqrt{2\pi\hbar}}\sqrt{\frac{2\pi}{\alpha^2}}
%    \exp{\left(\frac{\alpha^2}{2}
%    \left(a-\frac{i}{\alpha^2\hbar}p\right)^2
%    -\frac{\alpha^2a^2}{2}\right)}  \\
%    &= \frac{1}{\alpha\sqrt{\hbar}}
%    \exp{\left(-\frac{p^2}{2\alpha^2\hbar^2}-i\frac{a}{\hbar}p\right)}.
%  \end{split}
%\end{align}
%The wave function in the momentum space $\phi(p,t)$ at any time $t$ that is not a zero is,
%\begin{align}
%  \begin{split}
%    \phi(p,t) &= \exp{\left({-i\frac{p^2}{2m\hbar}t}\right)}\phi(p,0)
%    = \frac{1}{\alpha\sqrt{\hbar}}
%    \exp{\left({-i\frac{p^2}{2m\hbar}t}
%    -\frac{p^2}{2\alpha^2\hbar^2}-i\frac{a}{\hbar}p\right)} \\
%    &=\frac{1}{\alpha\sqrt{\hbar}}
%    \exp{\left(\left({-\frac{i\alpha^2\hbar t+m}{2m\alpha^2\hbar^2}}
%    \right)p^2-i\frac{a}{\hbar}p\right)}.
%  \end{split}
%\end{align}
%To obtain the wave function in the coordinate space at time $t$ that is not a zero,
%we operate Fourier transform into the $\phi(p,t)$.
%\begin{align}
%  \psi(x,t)=\frac{1}{\sqrt{2\pi\hbar}}\frac{1}{\alpha\sqrt{\hbar}}
%  \int\exp{\left(\frac{i}{\hbar}px\right)}
%  \exp{\left(\left({-\frac{i\alpha^2\hbar t+m}{2m\alpha^2\hbar^2}}
%  \right)p^2-i\frac{a}{\hbar}p\right)}\,dp
%\end{align}

\vspace{1cm}

\noindent \textbf{Problem 7.} THe Einstein model for a solid
assumes that it consists of many SHOs. If the $N$ atoms are similar
each other and oscillate similarly in average, the solid can be
explained in terms of $N$ SHOs. At a given temperature $T$, $N$ atoms
are in thermal equilibrium. Then, the Boltzmann distribution is given
by 
\begin{align}\label{eq:7-1}
P_n = \frac1{Z} e^{-E_n/kT}   
\end{align}
with
\begin{align}\label{eq:7-2}
  Z = \sum_n e^{-E_n/kT}, 
\end{align}
where
\begin{align}\label{eq:7-3}
E_n = \left(n+\frac12\right)\hbar \omega .
\end{align}
\begin{itemize}
\item[(1)] Derive the mean energy per an SHO 
  \begin{align}
    \langle E\rangle = \frac{\hbar\omega}{e^{\hbar\omega/kT}-1} +
    \frac12 \hbar \omega.
  \end{align}
\item[(2)] If $U$ is the internal energy of the solid, derive the
  specific heat with constant volume 
  \begin{align}
    C_V = \frac{\partial U}{\partial T}.
  \end{align}
Show that when $T$ is large, $C_V=3R$.
\item[(3)] Discuss the physics related to this problem as far as you
  can. 
\end{itemize}

\noindent \textbf{Answer : }
\begin{itemize}
  \item[(1)] By the definition, the expectation value of the energy is,
  \begin{align*}
    \langle E\rangle = \sum_n E_nP_n 
    = \frac{1}{Z}\sum_n E_n e^{-E_n/kT},   \,\,\,
    Z = \sum_n P_n.
  \end{align*}
  Define $\beta$ as,
  \begin{align}\label{eq:7-4}
    \beta =\frac{1}{kT}.
  \end{align}
  Then,
  \begin{align*}
    \langle E\rangle = \frac{1}{Z}\sum_n E_n e^{-\beta E_n},   \,\,\,
    Z = \sum_n e^{-\beta E_n}.
  \end{align*}
  The summation term is regraded as the deriavtive for $\beta$.
  \begin{align}\label{eq:7-5}
    \langle E\rangle = -\frac{1}{Z}\frac{\partial Z}{\partial \beta}
    = -\frac{\partial (\ln{Z})}{\partial \beta}.
  \end{align}
  From Eq.~\eqref{eq:7-3}, $Z$ is,
  \begin{align*}
    Z = \sum_n e^{-\beta\hbar\omega(\frac{1}{2}+n)} 
    = e^{-\frac{1}{2}\beta\hbar\omega}\sum_n e^{-n\beta\hbar\omega}.
  \end{align*}
  It is power series with a common ratio $e^{-\beta\hbar\omega}$ 
  and first term $e^{-\frac{1}{2}\beta\hbar\omega}$.
  \begin{align}
    Z = \frac{e^{-\frac{1}{2}\beta\hbar\omega}}{1-e^{-\beta\hbar\omega}}.
  \end{align}
  Then $\ln{Z}$ is,
  \begin{align*}
    \ln{Z} = \ln{\left(e^{-\frac{1}{2}\beta\hbar\omega}\right)}
    -\ln{\left(1-e^{-\beta\hbar\omega}\right)}
    =-\frac{1}{2}\beta\hbar\omega
    -\ln{\left(1-e^{-\beta\hbar\omega}\right)}.
  \end{align*}
  And Eq.~\eqref{eq:7-5} is,
  \begin{align*}
      \langle E\rangle = -\frac{\partial (\ln{Z})}{\partial \beta}
      = \frac{1}{2}\hbar\omega
      +\frac{\hbar\omega e^{-\beta\hbar\omega}}{1-e^{-\beta\hbar\omega}}  
  \end{align*}
  Multiplying $e^{\beta\hbar\omega}$ to the second term of the RHS,
  \begin{align}
    \begin{split}\label{eq:7-6}
      \langle E\rangle &=\frac{1}{2}\hbar\omega
      +\frac{\hbar\omega}{e^{\beta\hbar\omega}-1} \\
      &=\frac{\hbar\omega}{e^{\hbar\omega/kT}-1} 
      + \frac{1}{2}\hbar\omega.
    \end{split}
  \end{align}
  \item[(2)] The first law of the thermodynamics is,
  \begin{align}
    dU = dQ - dW.
  \end{align} 
  $Q$ and $W$ are the heat supplied to the system and 
  the work done on the system respectively. In the case of
  the solid, the volume is the constant and the work is a zero.
  \begin{align}
    dW = PdV = 0,\,\,\, dU = dQ.
  \end{align}
  By the definition of the specific heat with constant volume,
  \begin{align}
    C_V = \left(\frac{\partial Q}{\partial T}\right)_V 
    =\frac{\partial U}{\partial T}.
  \end{align}
  Since $U$ is the total energy of the solid and there are the $N$
  atoms,
  \begin{align}
    U = N\langle E\rangle=\frac{N\hbar\omega}{e^{\hbar\omega/kT}-1} 
    + \frac{1}{2}N\hbar\omega.
  \end{align}
  Hence the specific heat $C_V$ is,
  \begin{align}\label{eq:7-8}
    C_V = \frac{\partial U}{\partial T}
    = \frac{\partial }{\partial T}
    \left(\frac{N\hbar\omega}{e^{\hbar\omega/kT}-1}\right) 
    =\frac{N\hbar^2\omega^2e^{\hbar\omega/kT}}
    {kT^2\left(e^{\hbar\omega/kT}-1\right)^2}.
  \end{align}
  Consdiering that the degree of freedom is 3,
  \begin{align}
    C_V = \frac{3N\hbar^2\omega^2e^{\hbar\omega/kT}}
    {kT^2\left(e^{\hbar\omega/kT}-1\right)^2}.
  \end{align}
  From Eq.~\eqref{eq:7-4}, Eq.~\eqref{eq:7-8} can be rewritten as,
  \begin{align}
    C_V=\frac{3Nk\beta^2\hbar^2\omega^2e^{\beta\hbar\omega}}
    {\left(e^{\beta\hbar\omega}-1\right)^2}.
  \end{align}
  When $T$ is large, $\beta$ is converged to a zero and $e^{-\beta\hbar\omega}$
  is converged to 1. Then,
  \begin{align}
    \lim_{\beta\rightarrow 0}\frac{\beta}
    {e^{\beta\hbar\omega}-1} = \frac{1}{\hbar\omega}.
  \end{align}
  Finally $C_V$ is,
  \begin{align}
    C_V = \frac{3Nk\hbar^2\omega^2}
    {\hbar^2\omega^2}=3Nk = 3nR.
  \end{align}
  \item[(3)] From Eq.~\eqref{eq:7-6}, when the temperature is large,
  the mean energy per an SHO behaves approximately as the linear function.
   \begin{align}
    \begin{split}
      \langle E\rangle &= \frac{1}{2}\hbar\omega 
      + \hbar\omega\left[e^{\beta\hbar\omega}-1\right]^{-1}
      =\frac{1}{2}\hbar\omega 
      + \hbar\omega\left[(1+\hbar\omega\beta+\cdots)-1\right]^{-1}  \\
      &= 
    \end{split}
   \end{align} 

\end{itemize}
\end{document}