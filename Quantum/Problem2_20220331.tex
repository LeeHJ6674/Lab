%\documentclass[preprint,tightenlines,showpacs,showkeys,floatfix,
%nofootinbib,superscriptaddress,fleqn]{revtex4} 
\documentclass[floatfix,nofootinbib,superscriptaddress,fleqn]{revtex4-2} 
%\documentclass[aps,epsfig,tightlines,fleqn]{revtex4}
\usepackage{kotex}
\usepackage[HWP]{dhucs-interword}
\usepackage[dvips]{color}
\usepackage{graphicx}
\usepackage{bm}
%\usepackage{fancyhdr}
%\usepackage{dcolumn}
\usepackage{defcolor}
\usepackage{amsmath}
\usepackage{amsfonts}
\usepackage{amssymb}
\usepackage{amscd}
\usepackage{amsthm}
\usepackage[utf8]{inputenc}
%\pagestyle{fancy}

\begin{document}

\title{\Large Quantum Mechanics}
\author{김현철}
\email{hchkim@inha.ac.kr}
\affiliation{Hadron Theory Group, Department of Physics, Inha University,
Incheon 22212, Republic of Korea }
\date{2021}

\maketitle

\noindent {\bf Due date:} \textbf{\color{red} April 9, 2022} \\ 
\vspace{2cm}

\section*{\large Problem Set 2}
\noindent \textbf{Problem 1.}
A constant electric field $\mathcal{E}$ is exerted on a charged linear
harmonic oscillator. 
\begin{itemize}
\item[(1)] Write down the corresponding Schr\"odinger equation. 
\item[(2)] Derive the eigenvalues and eigenvectors of the charged
  linear oscillators under a uniform electric field. 
\item[(3)] Discuss the change in energy levels and physics. 
  eigenstates. 
\end{itemize}
Hint: Use the operator method.

\noindent \textbf{Answer : }
\begin{itemize}
  \item[(1)] A charged particle away from the equilibrium position 
  has the potential energy when it is in the electric field. 
  Let a distance from equilibrium position to a particle is $x$.
  In the constant electric field, the electric potential energy $E_p$ is,
  \begin{align}
    E_p = q\mathcal{E}x.
  \end{align} 
  Then, the Hamiltonain of the charged linear
  harmonic oscillator is,
  \begin{align}\label{eq:1-1}
    H = \frac{p^2}{2m}+\frac{1}{2}m\omega^2x^2-q\mathcal{E}x.
  \end{align}
  So, the Schr\"odinger equation is,
  \begin{align}\label{eq:1-2}
    -\frac{\hbar}{2m}\frac{\partial^2\psi}{\partial x^2}
    +\frac{1}{2}m\omega^2x^2\psi
    -q\mathcal{E}x\psi = E\psi.
  \end{align}
  \item[(2)] First, Suppose that there is no electric field,
  that is, $\mathcal{E}$ is zero. Then the Schr\"odinger equation and 
  the energy is,
  \begin{align}
    -\frac{\hbar}{2m}\frac{\partial^2\psi}{\partial x^2}
    +\frac{1}{2}m\omega^2x^2\psi = E\psi,\,\,\, 
    E_n = \left(\frac{1}{2}+n\right)\hbar\omega.
  \end{align}
  It is the Schr\"odinger equation of the simple harmonic oscillator. 
  In the algebric method to solve the equation, 
  we defined new operators,
  \begin{align}
    \begin{split}
      a = \sqrt{\frac{m\omega}{2\hbar}}\left(x+i\frac{p}{m\omega}\right),\,\,\,
      a^\dagger = \sqrt{\frac{m\omega}{2\hbar}}\left(x-i\frac{p}{m\omega}\right). 
    \end{split}
  \end{align}
  And,
  \begin{align}
    x = \sqrt{\frac{2\hbar}{m\omega}}\left(\frac{a+a^\dagger}{2}\right),\,\,\,
    p = \sqrt{2\hbar m\omega}\left(\frac{a-a^\dagger}{2i}\right)
  \end{align}
  It is said to ladder operators. Operators is from the hamiltonian of 
  simple harmonic oscillator,
  \begin{align}\label{eq:1-3}
    H = \frac{p^2}{2m}+\frac{1}{2}m\omega^2x^2 
    = \hbar\omega\left(a^\dagger a + \frac{1}{2}\right)
    = \hbar\omega\left(a a^\dagger - \frac{1}{2}\right).
  \end{align}
  Now, recall a constant electric field $\mathcal{E}$. From Eq.~\eqref{eq:1-1}
   and \eqref{eq:1-3}, hamiltonian is,
   \begin{align}
    H = \hbar\omega\left(a^\dagger a + \frac{1}{2}-q\mathcal{E}x\right)
    = \hbar\omega\left(a^\dagger a + \frac{1}{2}-q\mathcal{E}
    \sqrt{\frac{2\hbar}{m\omega}}\left(\frac{a+a^\dagger}{2}\right)\right),
   \end{align}  
   or,
   \begin{align}\label{eq:1-4}
    H = \hbar\omega\left(a a^\dagger - \frac{1}{2}\right)
    -q\mathcal{E}\sqrt{\frac{2\hbar}{m\omega}}
    \left(\frac{a+a^\dagger}{2}\right).
   \end{align}
   If $\psi_n$ is the eigenvector and $E_n$ is the eigenvalue of $\psi_n$, 
   the Schr\"odinger equation is,
   \begin{align}
    -\frac{\hbar}{2m}\frac{\partial^2\psi_n}{\partial x^2}
    +\frac{1}{2}m\omega^2x^2\psi_n
    -q\mathcal{E}x\psi_n = E_n\psi_n,\,\,\, 
    E_n = \left(\frac{1}{2}+n\right)\hbar\omega+E_m.
   \end{align}
   $E_m$ is a energy due to a constant electric field. Then we write the 
   reduced equation,
   \begin{align}\label{eq:1-5}
    -q\mathcal{E}x\psi_n = -q\mathcal{E}\sqrt{\frac{2\hbar}{m\omega}}
    \left(\frac{a+a^\dagger}{2}\right)\psi_n = E_m\psi_n.
   \end{align}
   Define $\kappa$ as,
   \begin{align}
     \kappa = -\frac{q\mathcal{E}}{2\hbar\omega}\sqrt{\frac{2\hbar}{m\omega}}
     = -\frac{1}{\omega}\frac{q\mathcal{E}}{\sqrt{2\hbar m\omega}}.
   \end{align}
   Then Eq.~\eqref{eq:1-4} is,
   \begin{align}
     H = \hbar\omega\left[a^\dagger a-\kappa\left(a+a^\dagger\right)
     +\frac{1}{2}\right]
     = \hbar\omega\left[(a^\dagger - \kappa)(a - \kappa) -\kappa^2.
     +\frac{1}{2}\right]
   \end{align}
   Now we define new operators from ladder operators,
   \begin{align}
     b = a - \kappa,\,\,\,
     b^\dagger = a^\dagger - \kappa.
   \end{align}
   The hamiltonian can be repersented by new operators.
   \begin{align}
     H = \hbar\omega\left( b^\dagger b -\left(\kappa^2-\frac{1}{2}\right)\right).
   \end{align}
   The Schr\"odinger equation and reduced equation from Eq.~\eqref{eq:1-2} 
   and \eqref{eq:1-5} are,
   \begin{align}
     \begin{split}
      &\hbar\omega\left( b^\dagger b 
      -\left(\kappa^2-\frac{1}{2}\right)\right)\psi_n = E_n\psi_n \\
      &\hbar\omega\kappa\left(b+b^\dagger+2\kappa\right)\psi_n = E_m\psi_n
      ,\,\,\,E_n = \left(\frac{1}{2}+n\right)\hbar\omega+E_m.
     \end{split}
   \end{align}
  \item[(3)] 
  \end{itemize}

\vspace{0.5cm}

\noindent \textbf{Problem 2.} 
The generating function $S(x,t)$ for the Hermite polynomial $H_n(x)$
is defined as 
\begin{align}
S(x,t) = e^{x^2-(t-x)^2} = e^{-t^2 + 2 t x} = \sum_{n=0}^\infty
  \frac{H_n(x)}{n!} t^n.  
\label{eq:1}
\end{align}
\begin{itemize}
\item[(1)] Using this generating function, derive the Hermite
  differential equation. 
\item[(2)] Derive the following formula from Eq.~\eqref{eq:1}:
  \begin{align}
H_n(x) = (-1)^n e^{x^2} \frac{d^n}{dx^n} e^{-x^2}    ,
  \end{align}
which is called the Rodrigues representation of the Hermite
polynomial. 
\item[(3)] Using Eq.~\eqref{eq:1}, derive the orthogonal relation of
  the Hermite polynomials
  \begin{align}
    \int_{-\infty}^\infty e^{-x^2} H_n(x) H_m(x) dx = 2^n \sqrt{\pi}
    n! \delta_{nm}.
  \end{align}
\item[(4)] Prove that
  \begin{align}
    \left(2x-\frac{d}{dx}\right)^n 1 = H_n(x),
  \end{align}
\item[(5)] Prove
  \begin{align}
    \int_{-\infty}^\infty x e^{-x^2} H_n(x) H_m(x) dx = \sqrt{\pi}
    2^{n-1} n!\delta_{m,n-1} + \sqrt{\pi} 2^n (n+1)! \delta_{m,n+1}.
  \end{align}
\item[(6)] Prove
  \begin{align}\label{eq:2}
    \int_{-\infty}^\infty x^2 e^{-x^2} H_n(x) H_n(x) dx = 
\sqrt{\pi} 2^n n! \left(n+\frac12\right).
  \end{align}
\end{itemize}

\noindent \textbf{Answer : }
\begin{itemize}
\item[(1)] The Hermite differential eqaution is,
\begin{align}\label{eq:2-1}
  \frac{d^2y}{dx^2}-2x\frac{dy}{dx}+\lambda y = 0,
\end{align}
$\lambda$ is a any constant. Derivatives for $x$ of generating function $S$ are,
\begin{align}\label{eq:2-2}
  \begin{split}
    &\frac{dS}{dx} = 2tS = \sum^\infty_{n=0}\frac{H^\prime_n(x)}{n!}t^n  \\
    &\frac{d^2S}{dx^2} = 4t^2S=\sum^\infty_{n=0}\frac{H^{\prime\prime}_n(x)}{n!}t^n.
  \end{split}
\end{align}
And,
\begin{align}\label{eq:2-3}
  \frac{dS}{dt} = 2(-t+x)S = \sum^\infty_{n=0}\frac{H_n(x)}{n!}nt^{n-1}
  =-\frac{dS}{dx} +2xS.
\end{align}
From Eq.~\eqref{eq:2-2},
\begin{align*}
  \begin{split}
    \frac{dS}{dx} &= \frac{1}{2t}\frac{d^2S}{dx^2}
    =\frac{1}{2}\sum^\infty_{n=0}\frac{H^{\prime\prime}_n(x)}{n!}t^{n-1}  \\
    2xS &= 2x\frac{1}{2t}\frac{dS}{dt} 
    =x\sum^\infty_{n=0}\frac{H^{\prime}_n(x)}{n!}t^{n-1}
  \end{split}
\end{align*}
Then Eq.~\eqref{eq:2-3} is,
\begin{align*}
  \sum^\infty_{n=0}\frac{H_n(x)}{n!}nt^{n-1}
  =-\frac{1}{2}\sum^\infty_{n=0}\frac{H^{\prime\prime}_n(x)}{n!}t^{n-1}
  +x\sum^\infty_{n=0}\frac{H^{\prime}_n(x)}{n!}t^{n-1}.
\end{align*}
Finally we obtain,
\begin{align*}
  \sum^\infty_{n=0}\left(\frac{H^{\prime\prime}_n(x)
  -2xH^{\prime}_n(x)+2nH_n(x)}{n!}t^{n-1}\right)=0.
\end{align*}
It is true for any $t$ when all coefficient is zero. So,
\begin{align}\label{eq:2-4}
  H^{\prime\prime}_n(x)
  -2xH^{\prime}_n(x)+2nH_n(x)=0.
\end{align}
\item[(2)] From Eq.~\eqref{eq:1}, 
\begin{align*}
  e^{-(t-x)^2+x^2}=e^{x^2}e^{-(t-x)^2}.
\end{align*}
And, 
\begin{align*}
  e^{x^2}e^{-(t-x)^2} = \left.e^{x^2}\sum^\infty_{n=0}\frac{t^n}{n!}
  \frac{d^n}{dt^n}e^{-(t-x)^2}\right|_{t=0} 
  = \sum^\infty_{n=0}\frac{H_n(x)}{n!}t^n.
\end{align*}
Since the series representation is unique,
\begin{align}\label{eq:2-5}
  H_n(x)=e^{x^2}\left.\frac{d^n}{dt^n}e^{-(t-x)^2}\right|_{t=0}.
\end{align}
If we regard $t$ as just the parameter, Eq.~\eqref{eq:2-5} is true for 
any $t$. 
A differential part of a LHS is,
\begin{align*}
  \left.\frac{d^n}{dt^n}e^{-(t-x)^2}\right|_{t=0}
  =\left.(-1)^n\frac{d^n}{dx^n}e^{-(t-x)^2}\right|_{t=0}
  =(-1)^n\frac{d^n}{dx^n}e^{-x^2}
\end{align*}
Finally we obtain,
\begin{align}\label{eq:2-5-1}
  H_n(x) = (-1)^ne^{x^2}\frac{d^n}{dx^n}e^{-x^2}.
\end{align}
\item[(3)]First, when $t=1$ and $t=-1$, Eq.~\eqref{eq:1} is,
\begin{align*}
  \begin{split}
    e^{2x-1}&=\sum^\infty_{n=0}\frac{H_n(x)}{n!} \\
    e^{-2x-1}&=\sum^\infty_{n=0}(-1)^n\frac{H_n(x)}{n!}.
  \end{split}
\end{align*} 
For checking the value $2^n \sqrt{\pi}n!$, consider a integration as,
\begin{align}\label{eq:2-6}
  \int_{-\infty}^{\infty}e^{-x^2}H_m(x)e^{2x-1}\,dx
  =\sum_{n=0}^\infty\frac{1}{n!}\int_{-\infty}^{\infty}e^{-x^2}H_m(x)H_n(x)\,dx
\end{align}
The RHS is,
\begin{align*}
  \int_{-\infty}^{\infty}e^{-x^2}H_m(x)e^{2x-1}\,dx 
  =(-1)^m\int_{-\infty}^{\infty}e^{2x-1}\frac{d^m}{dx^m}e^{-x^2}\,dx
\end{align*}
Using the integration by part to RHS,
\begin{align*}
  \begin{split}
    \int_{-\infty}^{\infty}e^{2x-1}\frac{d^m}{dx^m}e^{-x^2}\,dx  
    &=\left.e^{2x-1}\frac{d^{m-1}}{dx^{m-1}}e^{-x^2}\right|^\infty_{-\infty}
    -\int_{-\infty}^{\infty}2e^{2x-1}\frac{d^{m-1}}{dx^{m-1}}e^{-x^2}\,dx \\
    &=-2\int_{-\infty}^{\infty}e^{2x-1}\frac{d^{m-1}}{dx^{m-1}}e^{-x^2}\,dx 
  \end{split}
\end{align*}
Repetition of the integration by part for $m$ times conserves the 
form in the integration multiplying $(-2)^m$. 
\begin{align*}
  \int_{-\infty}^{\infty}e^{2x-1}\frac{d^m}{dx^m}e^{-x^2}\,dx  
  =(-2)^m\int_{-\infty}^{\infty}e^{2x-1}e^{-x^2}\,dx=(-2)^m\sqrt{\pi}. 
\end{align*}
Therefore, Eq.~\eqref{eq:2-6} is,
\begin{align*}
  \sum_{n=0}^\infty\frac{1}{n!}\int_{-\infty}^{\infty}e^{-x^2}H_m(x)H_n(x)\,dx
  =(-1)^m(-2)^m\sqrt{\pi}=2^m\sqrt{\pi}.
\end{align*}
Now, check the orthogonality. From Eq.~\eqref{eq:2-4},
\begin{align*}
  e^{x^2}\frac{d}{dx}\left(e^{-x^2}H^\prime_n(x)\right)+2nH_n(x)=0.
\end{align*}
Multiplying $e^{-x^2}$,
\begin{align*}
  \int_{-\infty}^\infty\frac{d}{dx}\left(e^{-x^2}H^\prime_n(x)\right)H_m(x)\,dx
  +2n\int_{-\infty}^\infty e^{-x^2}H_n(x)H_m(x)\,dx=0.
\end{align*}
Change $m$ and $n$ each other, subtract the previous one,
\begin{align*}
  \int_{-\infty}^\infty\frac{d}{dx}\left(e^{-x^2}H^\prime_n\right)H_m
  -\frac{d}{dx}\left(e^{-x^2}H^\prime_m\right)H_n\,dx
  +2(n-m)\int_{-\infty}^\infty e^{-x^2}H_n(x)H_m(x)\,dx=0.
\end{align*}
Integrations by part of first two terms are, 
\begin{align*}
  \begin{split}
    &\int_{-\infty}^\infty\frac{d}{dx}\left(e^{-x^2}H^\prime_n\right)H_m
    -\frac{d}{dx}\left(e^{-x^2}H^\prime_m\right)H_n\,dx \\
    &=\left.e^{-x^2}H^\prime_nH_m\right|_{-\infty}^\infty
    -\int_{-\infty}^\infty e^{-x^2}H^\prime_nH^\prime_m\,dx
    -\left.e^{-x^2}H_nH^\prime_m\right|_{-\infty}^\infty
    +\int_{-\infty}^\infty e^{-x^2}H^\prime_nH^\prime_m\,dx \\
    &=0.
  \end{split}
\end{align*}
It means that,
\begin{align*}
  2(n-m)\int_{-\infty}^\infty e^{-x^2}H_n(x)H_m(x)\,dx=0.
\end{align*}
If $n\neq m$, the integration is a zero. For this reason,
\begin{align*}
  \sum_{n=0}^\infty\frac{1}{n!}\int_{-\infty}^{\infty}e^{-x^2}H_m(x)H_n(x)\,dx
  =\frac{1}{m!}\int_{-\infty}^{\infty}e^{-x^2}H_m(x)H_m(x)\,dx
  =2^m\sqrt{\pi}.
\end{align*}
Finally we obtain,
\begin{align}\label{eq:2-7}
  \int_{-\infty}^\infty e^{-x^2} H_n(x) H_m(x) dx = 2^nn! \sqrt{\pi}
   \delta_{nm}.
\end{align}
\item[(4)] {\it Proof}. We use the mathematical induction. 
If $n=0$ and $n=1$, then,
\begin{align*}
  H_0(x) = 1,\,\,\,H_1(x)=2x. 
\end{align*}
The statement is true. Suppose it is true:
\begin{align*}
  H_k(x) = \left(2x-\frac{d}{dx}\right)^k 1.
\end{align*}
Then,
\begin{align*}
  \begin{split}
    H_{k+1}(x) &= \left(2x-\frac{d}{dx}\right)\left(2x-\frac{d}{dx}\right)^k 1
    =\left(2x-\frac{d}{dx}\right)H_k(x)
  \end{split}
\end{align*}
From Eq.~\eqref{eq:2-5-1},
\begin{align*}
  \begin{split}
    \left(2x-\frac{d}{dx}\right)H_k(x)
    &=\left(2x-\frac{d}{dx}\right)(-1)^ke^{x^2}\frac{d^k}{dx^k}e^{-x^2}  \\
    &=(-1)^k 2x e^{x^2}\frac{d^k}{dx^k}e^{-x^2}
    -(-1)^k 2x e^{x^2}\frac{d^k}{dx^k}e^{-x^2}
    -(-1)^k e^{x^2}\frac{d^{k+1}}{dx^{k+1}}e^{-x^2}  \\
    &=(-1)^{k+1}= e^{x^2}\frac{d^{k+1}}{dx^{k+1}}e^{-x^2}
    =H_{k+1}(x).
  \end{split}
\end{align*}
Hence this statement is true for $n=k+1$. \\
By mathematical induction, 
this statement is true for any $n$. ~\hfill $\square$
\item[(5)] {\it Proof}. Set $I_{nm}$,
\begin{align*}
  \begin{split}
    I_{nm} &= \int_{-\infty}^\infty xe^{-x^2}H_n(x)H_m(x) \,dx  \\
    &=-\left.\frac{1}{2}e^{-x^2}H_n(x)H_m(x)\right|_{-\infty}^\infty
    +\frac{1}{2}\int_{-\infty}^\infty 
    e^{-x^2}\left(H^\prime_n(x)H_m(x)+H_n(x)H^\prime_m(x)\right) \,dx \\
    &=\frac{1}{2}\int_{-\infty}^\infty 
    e^{-x^2}H^\prime_n(x)H_m(x) \,dx+
    \frac{1}{2}\int_{-\infty}^\infty 
    e^{-x^2}H_n(x)H^\prime_m(x) \,dx.
  \end{split}
\end{align*}
From Eq.~\eqref{eq:2-5-1},
\begin{align}\label{eq:2-8}
  \begin{split}
    H^\prime_n(x) &= \frac{d}{dx}\left( (-1)^ne^{x^2}
    \frac{d^n}{dx^n}e^{-x^2} \right)
    =(-1)^n\left( 2xe^{x^2}\frac{d^n}{dx^n}e^{-x^2}
    +e^{x^2}\frac{d^{n+1}}{dx^{n+1}}e^{-x^2} \right)  \\
    &=2xH_n(x)-H_{n+1}(x).
  \end{split}
\end{align}
Then $I_{nm}$ is,
\begin{align*}
  \begin{split}
    I_{nm} &= \frac{1}{2}\int_{-\infty}^\infty 
    e^{-x^2}(2xH_n(x)-H_{n+1}(x))H_m(x) \,dx+
    \frac{1}{2}\int_{-\infty}^\infty 
    e^{-x^2}H_n(x)(2xH_m(x)-H_{m+1}(x)) \,dx  \\
    &= 2I_{nm}
    -\frac{1}{2}\int_{-\infty}^\infty 
    e^{-x^2}H_{n+1}(x)H_m(x) \,dx
    -\frac{1}{2}\int_{-\infty}^\infty 
    e^{-x^2}H_n(x)H_{m+1}(x) \,dx.
  \end{split}
\end{align*}
Hence,
\begin{align*}
  I_{nm}=\frac{1}{2}\left(\int_{-\infty}^\infty 
  e^{-x^2}H_{n+1}(x)H_m(x) \,dx
  +\int_{-\infty}^\infty 
  e^{-x^2}H_n(x)H_{m+1}(x) \,dx\right).
\end{align*}
From Eq.~\eqref{eq:2-7}, we obtain that,
\begin{align}\label{eq:2-9}
  \begin{split}
    I_{nm}&=\frac{1}{2}\left(2^{n+1}\sqrt{\pi}(n+1)!\delta_{n+1,m}
    +2^n\sqrt{\pi}n!\delta_{n,m+1} \right)  \\
    &=2^n\sqrt{\pi}(n+1)!\delta_{n+1,m}
    +2^{n-1}\sqrt{\pi}n!\delta_{n,m+1}.
  \end{split}
\end{align}
Therefore the statement is true. ~\hfill $\square$
\item[(6)] {\it Proof}. Eq.~\eqref{eq:2} is,
\begin{align*}
  \int_{-\infty}^\infty x^2e^{-x^2}H_nH_n \,dx  
  =-\left.\frac{1}{2}xe^{-x^2}H_nH_n\right|_{-\infty}^\infty
  +\int_{-\infty}^\infty xe^{-x^2}H_n^\prime H_n \,dx  
  -\frac{1}{2}\int_{-\infty}^\infty e^{-x^2}H_nH_n \,dx  .
\end{align*}
From Eq.~\eqref{eq:2-8}, the second term of the RHS is,
\begin{align*}
  \int_{-\infty}^\infty xe^{-x^2}H_n^\prime H_n \,dx  
  =\int_{-\infty}^\infty 2x^2e^{-x^2}H_n H_n \,dx 
  -\int_{-\infty}^\infty xe^{-x^2}H_{n+1}H_n \,dx  .
\end{align*}
Hence, 
\begin{align*}
  \begin{split}
    \int_{-\infty}^\infty x^2e^{-x^2}H_nH_n \,dx  
    &=\int_{-\infty}^\infty 2x^2e^{-x^2}H_n H_n \,dx 
    -\int_{-\infty}^\infty xe^{-x^2}H_{n+1}H_n \,dx 
    -\frac{1}{2}\int_{-\infty}^\infty e^{-x^2}H_nH_n \,dx \\
    &=\int_{-\infty}^\infty xe^{-x^2}H_{n+1}H_n \,dx 
    +\frac{1}{2}\int_{-\infty}^\infty e^{-x^2}H_nH_n \,dx.
  \end{split}
\end{align*}
From Eq.~\eqref{eq:2-9} and \eqref{eq:2-7},
\begin{align}\label{eq:2-10}
  \begin{split}
    \int_{-\infty}^\infty x^2e^{-x^2}H_nH_n \,dx  
    &=\sqrt{\pi}2^{n-1}n!\delta_{n+1,n-1}
    +\sqrt{\pi}2^{n}(n+1)!\delta_{n+1,n+1}
    -\sqrt{\pi}2^{n-1}n!  \\
    &=\sqrt{\pi}2^nn!\left(n+\frac{1}{2}\right).
  \end{split}
\end{align}
Therefore the statement is true. ~\hfill $\square$
\end{itemize}

\vspace{0.5cm}

\noindent \textbf{Problem 3.} 
Given the eigenfunctions and eigenenergies of the SHO, 
\begin{itemize}
\item[(1)] Compute the kinetic and potential energies at the $n^{th}$
  level.  Show that the results satisfy the virial theorem. 
\item[(2)] Show that the $n^{th}$ state of the SHO satisfies 
  \begin{align}
    \Delta x \Delta p = \left(n+\frac12\right)\hbar.
  \end{align}
\end{itemize}

\noindent \textbf{Answer : }
\begin{itemize}
  \item[(1)] The eigenvector and eigenfunction of the SHO are,
  \begin{align}
    \psi_n(x)=\psi^*_n(x)= (n!2^n)^{-\frac{1}{2}}
    \left(\frac{m\omega}{\hbar\pi}\right)^{\frac{1}{4}}
    \exp\left(-\frac{m\omega}{2\hbar}x^2\right)
    H_n\left(\sqrt{\frac{m\omega}{\hbar}}x\right),\,\,\,
    E_n=\left( n+\frac{1}{2} \right)\hbar\omega.
  \end{align}
  The expectation value of the kinetic energy is,
  \begin{align}
   \langle T_n\rangle = \frac{1}{2m}\int \psi^*_n p^2 \psi_n\,dx
   =\frac{\langle p^2\rangle}{2m}.
  \end{align}
  Since the expectation value of the kinetic energy is an integer multiple 
  of the square of momentum, we just calculate the expectation value of the 
  square of momentum. Using the integration by part,
  \begin{align}\label{eq:3-1}
    \langle p^2\rangle = -\hbar^2\int \psi^*_n 
    \frac{\partial^2 \psi_n}{\partial x^2} \,dx
    =\hbar^2\int\frac{\partial \psi^*_n}{\partial x} 
    \frac{\partial \psi_n}{\partial x} \,dx
  \end{align}
  Changing the variable,
  \begin{align}\label{eq:3-2}
    \sqrt{\frac{m\omega}{\hbar}}x = \xi,\,\,\,
    \frac{\partial \psi_n}{\partial x}
    =\frac{\partial \psi_n}{\partial \xi}
    \frac{\partial \xi}{\partial x}
    =\sqrt{\frac{m\omega}{\hbar}}
    \frac{\partial \psi_n}{\partial \xi}
  \end{align}  
  Then,
  \begin{align}
    \frac{\partial \psi_n}{\partial \xi}
    =(n!2^n)^{-\frac{1}{2}}
    \left(\frac{m\omega}{\hbar\pi}\right)^{\frac{1}{4}}
    (-\xi H_n\left(\xi\right) 
    +H^\prime_n\left(\xi\right))
    e^{-\frac{\xi^2}{2}}.
  \end{align}
  The integration of Eq.~\eqref{eq:3-1} is,
  \begin{align}
    \begin{split}
      \int\frac{\partial \psi^*_n}{\partial x} 
      \frac{\partial \psi_n}{\partial x} \,dx
      &= (n!2^n)^{-1}\sqrt{\frac{m\omega}{\hbar\pi}}
      \frac{m\omega}{\hbar}\int(-\xi H_n\left(\xi\right) 
      +H^\prime_n\left(\xi\right))^2 e^{-\xi^2}\,
      \sqrt{\frac{\hbar}{m\omega}}d\xi  \\
      &= (n!2^n)^{-1}\frac{m\omega}{\hbar\sqrt{\pi}}
      \int(-\xi H_n\left(\xi\right) 
      +H^\prime_n\left(\xi\right))^2 e^{-\xi^2}\,d\xi  \\
    \end{split}
  \end{align}
  From Eq.~\eqref{eq:2-8}
  \begin{align}
    \begin{split}
      \int(-\xi H_n\left(\xi\right) 
      +H^\prime_n\left(\xi\right))^2 e^{-\xi^2}\,d\xi
      &=\int(-\xi H_n\left(\xi\right) 
      + 2\xi H_n(\xi)-H_{n+1}(\xi))^2 e^{-\xi^2}\,d\xi  \\
      &=\int(\xi H_n\left(\xi\right) 
      -H_{n+1}(\xi))^2 e^{-\xi^2}\,d\xi \\
      &=\int(
      (\xi^2H_nH_n-2\xi H_nH_{n+1}+H_{n+1}H_{n+1})  
      e^{-\xi^2}\,d\xi.
    \end{split}
  \end{align}
  We can use Eq.~\eqref{eq:2-7}, \eqref{eq:2-9} and \eqref{eq:2-10} to calculate 
  this integration.
  \begin{align}
    \begin{split}
      \int \xi^2 H_n H_n e^{-\xi^2}\,dx 
      &= 2^nn!\sqrt{\pi}\left( n+\frac{1}{2} \right) \\  
      \int \xi H_n H_{n+1} e^{-\xi^2}\,dx 
      &= 2^n(n+1)!\sqrt{\pi}\\  
      \int  H_{n+1} H_{n+1} e^{-\xi^2}\,dx 
      &= 2^{n+1}(n+1)!\sqrt{\pi}
    \end{split}
  \end{align}
  Then,
  \begin{align}
    \begin{split}
      \int(-\xi H_n\left(\xi\right) 
      +H^\prime_n\left(\xi\right))^2 e^{-\xi^2}\,d\xi
      &= \sqrt{\pi}2^nn!
      \left(n+\frac{1}{2}-2(n+1)+2(n+1) \right) \\
      &= \sqrt{\pi}2^nn!\left( n+\frac{1}{2} \right).
    \end{split}
  \end{align}
  Therefore the expectation value of the square of the momentum is,
  \begin{align}
    \langle p^2\rangle = \hbar^2\,(n!2^n)^{-1}\frac{m\omega}{\hbar\sqrt{\pi}}\,
    \sqrt{\pi}2^nn!\left( n+\frac{1}{2} \right)
    =\hbar m\omega\left( n+\frac{1}{2} \right).
  \end{align}
  We obtain the expectation value of the kinetic energy.
  \begin{align}\label{eq:3-3}
    \langle T_n\rangle=\frac{\langle p^2\rangle}{2m}
    =\frac{1}{2}\hbar \omega\left( n+\frac{1}{2} \right).
  \end{align}
  The expectation value of the potential energy is,
  \begin{align}
    \langle V_n\rangle = \int \psi^*_n \frac{1}{2}m\omega^2x^2 \psi_n \,dx
    =\frac{1}{2}m\omega^2\int \psi^*_n x^2 \psi_n \,dx
    =\frac{1}{2}m\omega^2\langle x^2\rangle.
  \end{align}
  From Eq.~\eqref{eq:3-2}, the expectation value of the square of $x$ is,
  \begin{align}
    \begin{split}
      \langle x^2\rangle &= \int \psi^*_n x^2 \psi_n \,dx
      =\langle x^2\rangle ={\left(\frac{\hbar}{m\omega}\right)}^{\frac{3}{2}} 
      \int \psi^*_n(\xi) \xi^2 \psi_n(\xi) \,d\xi \\
      &= (n!2^n)^{-1}\sqrt{\frac{m\omega}{\hbar\pi}}
      {\left(\frac{\hbar}{m\omega}\right)}^{\frac{3}{2}} 
      \int \xi^2 H_n(\xi) H_n(\xi) e^{-\xi^2}\,d\xi
    \end{split}    
  \end{align}
  The integration part can be calculated by Eq.~\eqref{eq:2-10}.
  \begin{align}
    \int \xi^2 H_n(\xi) H_n(\xi) e^{-\xi^2}\,d\xi
    =\sqrt{\pi}2^nn!\left( n+\frac{1}{2} \right).
  \end{align}
  Finally we obtain the expectation value of the potential energy.
  \begin{align}\label{eq:3-4}
    \langle V_n\rangle=\frac{1}{2}m\omega^2(n!2^n)^{-1}
    \sqrt{\frac{m\omega}{\hbar\pi}}
    {\left(\frac{\hbar}{m\omega}\right)}^{\frac{3}{2}} 
    \sqrt{\pi}2^nn!\left( n+\frac{1}{2} \right)
    =\frac{1}{2}\hbar\omega
    \left( n+\frac{1}{2} \right).
  \end{align}
  Let us confirm that the results satisfy the virial theorem. 
  In this condition the virial theorem is,
  \begin{align}
    \left\langle x\frac{\partial V}{\partial x}\right\rangle 
    = 2\left\langle T\right\rangle .
  \end{align}
  Substituting Eq.~\eqref{eq:3-3} and \eqref{eq:3-4},
  \begin{align}
  \left\langle x\frac{\partial V}{\partial x}\right\rangle 
  =m\omega^2\int\psi^*_nx^2\psi_n\,dx = 2\left\langle V_n\right\rangle 
  =2\left\langle T_n\right\rangle.
  \end{align}
  The results satisfy the virial theorem.
  \item[(2)] 
\end{itemize}

\vspace{1cm}

\noindent \textbf{Problem 4.}
If a wavefunction desribes a mixed state of the eigenstates of the SHO
given as 
\begin{align}
  \psi(x,t) = \frac1{\sqrt{2}}[\psi_0(x,t) + \psi_1(x,t)] ,
\end{align}
\begin{itemize}
\item[(1)] Investigate how the probability density changes in time. 
\item[(2)] Prove the following relations 
  \begin{align}
\langle E\rangle &= \langle H\rangle =\hbar \omega    ,\cr
\langle x \rangle &= \frac1{\sqrt{2\alpha}}\cos\omega t,\cr
\langle p \rangle &= -\sqrt{\frac{\alpha}{2}}\hbar \sin\omega t,
 \end{align}
where $\alpha = \sqrt{m\omega/\hbar}$.
 \item[(3)] If 
   \begin{align}
 \psi(x,t) = \frac1{\sqrt{2}}[e^{i\delta_0} \psi_0(x,t) + e^{i\delta}
     \psi_1(x,t)],     
   \end{align}
discuss the effects of the phase factors $\delta_0$ and $\delta$ on
$\langle x\rangle$ and $\langle p\rangle$.
\end{itemize}
 \vspace{1cm}

\noindent \textbf{Problem 5.}
Derive the wavefunction in momentum space, which corresponds to the
eigenfunctions for the SHO in coordinates, $\psi_n(x)$. 
\vspace{1cm}

\noindent \textbf{Problem 6.}
At $t=0$, the wavefunction for a state is described by
\begin{align}
\psi(x,0) = \sum_n A_n u_n(x) =
  \left(\frac{\alpha^2}{\pi}\right)^{1/4} e^{-\alpha^2(x-a)^2/2}  .
\end{align}
show that after some time $t$, the probability density changes in time
as 
\begin{align}
|\pi(x,t)|^2  =  \left(\frac{\alpha^2}{\pi}\right)^{1/4}e^{-\alpha^2
  (x-a\cos\omega t)^2}
\end{align}
and discuss the result. 
\vspace{1cm}

\noindent \textbf{Problem 7.} THe Einstein model for a solid
assumes that it consists of many SHOs. If the $N$ atoms are similar
each other and oscillate similarly in average, the solid can be
explained in terms of $N$ SHOs. At a given temperature $T$, $N$ atoms
are in thermal equilibrium. Then, the Boltzmann distribution is given
by 
\begin{align}
P_n = \frac1{Z} e^{-E_n/kT}   
\end{align}
with
\begin{align}
  Z = \sum_n e^{-E_n/kT}, 
\end{align}
where
\begin{align}
E_n = \left(n+\frac12\right)\hbar \omega .
\end{align}
\begin{itemize}
\item[(1)] Derive the mean energy per an SHO 
  \begin{align}
    \langle E\rangle = \frac{\hbar\omega}{e^{\hbar\omega/kT}-1} +
    \frac12 \hbar \omega.
  \end{align}
\item[(2)] If $U$ is the internal energy of the solid, derive the
  specific heat with constant volume 
  \begin{align}
    C_V = \frac{\partial U}{\partial T}.
  \end{align}
Show that when $T$ is large, $C_V=3R$.
\item[(3)] Discuss the physics related to this problem as far as you
  can. 
\end{itemize}

\end{document}