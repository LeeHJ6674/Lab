%\documentclass[preprint,tightenlines,showpacs,showkeys,floatfix,
%nofootinbib,superscriptaddress,fleqn]{revtex4} 
\documentclass[floatfix,nofootinbib,superscriptaddress,fleqn]{revtex4-2} 
%\documentclass[aps,epsfig,tightlines,fleqn]{revtex4}
\usepackage{kotex}
\usepackage[HWP]{dhucs-interword}
\usepackage[dvips]{color}
\usepackage{graphicx}
\usepackage{bm}
%\usepackage{fancyhdr}
%\usepackage{dcolumn}
\usepackage{defcolor}
\usepackage{amsmath}
\usepackage{amsfonts}
\usepackage{amssymb}
\usepackage{amscd}
\usepackage{amsthm}
\usepackage[utf8]{inputenc}
%\pagestyle{fancy}

\begin{document}

\title{\Large Quantum Mechanics}
\author{김현철}
\email{hchkim@inha.ac.kr}
\affiliation{Hadron Theory Group, Department of Physics, Inha University,
Incheon 22212, Republic of Korea }
\date{2021}

\maketitle

\noindent {\bf Due date:} \textbf{\color{red} April 9, 2022} \\ 
\vspace{2cm}

\section*{\large Problem Set 2}
\noindent \textbf{Problem 1.}
A constant electric field $\mathcal{E}$ is exerted on a charged linear
harmonic oscillator. 
\begin{itemize}
\item[(1)] Write down the corresponding Schr\"odinger equation. 
\item[(2)] Derive the eigenvalues and eigenvectors of the charged
  linear oscillators under a uniform electric field. 
\item[(3)] Discuss the change in energy levels and physics. 
  eigenstates. 
\end{itemize}
Hint: Use the operator method.

\noindent \textbf{Answer : }
\begin{itemize}
  \item[(1)] A charged particle away from the equilibrium position 
  has the potential energy when it is in the electric field. 
  Let a distance from equilibrium position to a particle is $x$.
  In the constant electric field, the electric potential energy $E_p$ is,
  \begin{align}
    E_p = q\mathcal{E}x.
  \end{align} 
  Then, the Hamiltonain of the charged linear
  harmonic oscillator is,
  \begin{align}\label{eq:1-1}
    H = \frac{p^2}{2m}+\frac{1}{2}m\omega^2x^2-q\mathcal{E}x.
  \end{align}
  So, the Schr\"odinger equation is,
  \begin{align}\label{eq:1-2}
    -\frac{\hbar}{2m}\frac{\partial^2\psi}{\partial x^2}
    +\frac{1}{2}m\omega^2x^2\psi
    -q\mathcal{E}x\psi = E\psi.
  \end{align}
  \item[(2)] First, Suppose that there is no electric field,
  that is, $\mathcal{E}$ is zero. Then the Schr\"odinger equation and 
  the energy is,
  \begin{align}
    -\frac{\hbar}{2m}\frac{\partial^2\psi}{\partial x^2}
    +\frac{1}{2}m\omega^2x^2\psi = E\psi,\,\,\, 
    E_n = \left(\frac{1}{2}+n\right)\hbar\omega.
  \end{align}
  It is the Schr\"odinger equation of the simple harmonic oscillator. 
  In the algebric method to solve the equation, 
  we defined new operators,
  \begin{align}
    \begin{split}
      a = \sqrt{\frac{m\omega}{2\hbar}}\left(x+i\frac{p}{m\omega}\right),\,\,\,
      a^\dagger = \sqrt{\frac{m\omega}{2\hbar}}\left(x-i\frac{p}{m\omega}\right). 
    \end{split}
  \end{align}
  And,
  \begin{align}
    x = \sqrt{\frac{2\hbar}{m\omega}}\left(\frac{a+a^\dagger}{2}\right),\,\,\,
    p = \sqrt{2\hbar m\omega}\left(\frac{a-a^\dagger}{2i}\right)
  \end{align}
  It is said to ladder operators. Operators is from the hamiltonian of 
  simple harmonic oscillator,
  \begin{align}\label{eq:1-3}
    H = \frac{p^2}{2m}+\frac{1}{2}m\omega^2x^2 
    = \hbar\omega\left(a^\dagger a + \frac{1}{2}\right)
    = \hbar\omega\left(a a^\dagger - \frac{1}{2}\right).
  \end{align}
  Now, recall a constant electric field $\mathcal{E}$. From Eq. \eqref{eq:1-1}
   and \eqref{eq:1-3}, hamiltonian is,
   \begin{align}
    H = \hbar\omega\left(a^\dagger a + \frac{1}{2}-q\mathcal{E}x\right)
    = \hbar\omega\left(a^\dagger a + \frac{1}{2}-q\mathcal{E}
    \sqrt{\frac{2\hbar}{m\omega}}\left(\frac{a+a^\dagger}{2}\right)\right),
   \end{align}  
   or,
   \begin{align}\label{eq:1-4}
    H = \hbar\omega\left(a a^\dagger - \frac{1}{2}\right)
    -q\mathcal{E}\sqrt{\frac{2\hbar}{m\omega}}
    \left(\frac{a+a^\dagger}{2}\right).
   \end{align}
   If $\psi_n$ is the eigenvector and $E_n$ is the eigenvalue of $\psi_n$, 
   the Schr\"odinger equation is,
   \begin{align}
    -\frac{\hbar}{2m}\frac{\partial^2\psi_n}{\partial x^2}
    +\frac{1}{2}m\omega^2x^2\psi_n
    -q\mathcal{E}x\psi_n = E_n\psi_n,\,\,\, 
    E_n = \left(\frac{1}{2}+n\right)\hbar\omega+E_m.
   \end{align}
   $E_m$ is a energy due to a constant electric field. Then we write the 
   reduced equation,
   \begin{align}\label{eq:1-5}
    -q\mathcal{E}x\psi_n = -q\mathcal{E}\sqrt{\frac{2\hbar}{m\omega}}
    \left(\frac{a+a^\dagger}{2}\right)\psi_n = E_m\psi_n.
   \end{align}
   Define $\kappa$ as,
   \begin{align}
     \kappa = -\frac{q\mathcal{E}}{2\hbar\omega}\sqrt{\frac{2\hbar}{m\omega}}
     = -\frac{1}{\omega}\frac{q\mathcal{E}}{\sqrt{2\hbar m\omega}}.
   \end{align}
   Then Eq. \eqref{eq:1-4} is,
   \begin{align}
     H = \hbar\omega\left[a^\dagger a-\kappa\left(a+a^\dagger\right)
     +\frac{1}{2}\right]
     = \hbar\omega\left[(a^\dagger - \kappa)(a - \kappa) -\kappa^2.
     +\frac{1}{2}\right]
   \end{align}
   Now we define new operators from ladder operators,
   \begin{align}
     b = a - \kappa,\,\,\,
     b^\dagger = a^\dagger - \kappa.
   \end{align}
   The hamiltonian can be repersented by new operators.
   \begin{align}
     H = \hbar\omega\left( b^\dagger b -\left(\kappa^2-\frac{1}{2}\right)\right).
   \end{align}
   The Schr\"odinger equation and reduced equation from Eq. \eqref{eq:1-2} 
   and \eqref{eq:1-5} are,
   \begin{align}
     \begin{split}
      &\hbar\omega\left( b^\dagger b 
      -\left(\kappa^2-\frac{1}{2}\right)\right)\psi_n = E_n\psi_n \\
      &\hbar\omega\kappa\left(b+b^\dagger+2\kappa\right)\psi_n = E_m\psi_n
      ,\,\,\,E_n = \left(\frac{1}{2}+n\right)\hbar\omega+E_m.
     \end{split}
   \end{align}
  \item[(3)] 
  \end{itemize}

\vspace{0.5cm}

\noindent \textbf{Problem 2.} 
The generating function $S(x,t)$ for the Hermite polynomial $H_n(x)$
is defined as 
\begin{align}
S(x,t) = e^{x^2-(t-x)^2} = e^{-t^2 + 2 t x} = \sum_{n=0}^\infty
  \frac{H_n(x)}{n!} t^n.  
\label{eq:1}
\end{align}
\begin{itemize}
\item[(1)] Using this generating function, derive the Hermite
  differential equation. 
\item[(2)] Derive the following formula from Eq.~\eqref{eq:1}:
  \begin{align}
H_n(x) = (-1)^n e^{x^2} \frac{d^n}{dx^n} e^{-x^2}    ,
  \end{align}
which is called the Rodrigues representation of the Hermite
polynomial. 
\item[(3)] Using Eq.~\eqref{eq:1}, derive the orthogonal relation of
  the Hermite polynomials
  \begin{align}
    \int_{-\infty}^\infty e^{-x^2} H_n(x) H_m(x) dx = 2^n \sqrt{\pi}
    n! \delta_{nm}.
  \end{align}
\item[(4)] Prove that
  \begin{align}
    \left(2x-\frac{d}{dx}\right)^n 1 = H_n(x),
  \end{align}
\item[(5)] Prove
  \begin{align}
    \int_{-\infty}^\infty x e^{-x^2} H_n(x) H_m(x) dx = \sqrt{\pi}
    2^{n-1} n!\delta_{m,n-1} + \sqrt{\pi} 2^n (n+1)! \delta_{m,n+1}.
  \end{align}
\item[(6)] Prove
  \begin{align}
    \int_{-\infty}^\infty x^2 e^{-x^2} H_n(x) H_n(x) dx = 
\sqrt{\pi} 2^n n! \left(n+\frac12\right).
  \end{align}
\end{itemize}

\noindent \textbf{Answer : }
\begin{itemize}
\item[(1)] The Hermite differential eqaution is,
\begin{align}
  \frac{d^2y}{dx^2}-2x\frac{dy}{dx}+\lambda y = 0,
\end{align}
$\lambda$ is a any constant. Then,
\begin{align}\label{eq:2-2}
  \begin{split}
    &\frac{dS}{dx} = 2tS = \sum^\infty_{n=0}\frac{H^\prime_n(x)}{n!}t^n  \\
    &\frac{d^2S}{dx^2} = 4t^2S=\sum^\infty_{n=0}\frac{H^{\prime\prime}_n(x)}{n!}t^n.
  \end{split}
\end{align}
And,
\begin{align}\label{eq:2-3}
  \frac{dS}{dt} = 2(-t+x)S = \sum^\infty_{n=0}\frac{H_n(x)}{(n-1)!}t^{n-1}.
\end{align}
Substituting Eq. \eqref{eq:2-2} and \eqref{eq:2-3} into Eq. \eqref{eq:1},
\begin{align}
  asd
\end{align} 
\item[(2)]
\item[(3)]
\item[(4)]
\item[(5)]
\item[(6)]
\end{itemize}


\noindent \textbf{Problem 3.} 
Given the eigenfunctions and eigenenergies of the SHO, 
\begin{itemize}
\item[(1)] Compute the kinetic and potential energies at the $n^{th}$
  level.  Show that the results satisfy the virial theorem. 
\item[(2)] Show that the $n^{th}$ state of the SHO satisfies 
  \begin{align}
    \Delta x \Delta p = \left(n+\frac12\right)\hbar.
  \end{align}
\end{itemize}

\vspace{1cm}

\noindent \textbf{Problem 4.}
If a wavefunction desribes a mixed state of the eigenstates of the SHO
given as 
\begin{align}
  \psi(x,t) = \frac1{\sqrt{2}}[\psi_0(x,t) + \psi_1(x,t)] ,
\end{align}
\begin{itemize}
\item[(1)] Investigate how the probability density changes in time. 
\item[(2)] Prove the following relations 
  \begin{align}
\langle E\rangle &= \langle H\rangle =\hbar \omega    ,\cr
\langle x \rangle &= \frac1{\sqrt{2\alpha}}\cos\omega t,\cr
\langle p \rangle &= -\sqrt{\frac{\alpha}{2}}\hbar \sin\omega t,
 \end{align}
where $\alpha = \sqrt{m\omega/\hbar}$.
 \item[(3)] If 
   \begin{align}
 \psi(x,t) = \frac1{\sqrt{2}}[e^{i\delta_0} \psi_0(x,t) + e^{i\delta}
     \psi_1(x,t)],     
   \end{align}
discuss the effects of the phase factors $\delta_0$ and $\delta$ on
$\langle x\rangle$ and $\langle p\rangle$.
\end{itemize}
 \vspace{1cm}

\noindent \textbf{Problem 5.}
Derive the wavefunction in momentum space, which corresponds to the
eigenfunctions for the SHO in coordinates, $\psi_n(x)$. 
\vspace{1cm}

\noindent \textbf{Problem 6.}
At $t=0$, the wavefunction for a state is described by
\begin{align}
\psi(x,0) = \sum_n A_n u_n(x) =
  \left(\frac{\alpha^2}{\pi}\right)^{1/4} e^{-\alpha^2(x-a)^2/2}  .
\end{align}
show that after some time $t$, the probability density changes in time
as 
\begin{align}
|\pi(x,t)|^2  =  \left(\frac{\alpha^2}{\pi}\right)^{1/4}e^{-\alpha^2
  (x-a\cos\omega t)^2}
\end{align}
and discuss the result. 
\vspace{1cm}

\noindent \textbf{Problem 7.} THe Einstein model for a solid
assumes that it consists of many SHOs. If the $N$ atoms are similar
each other and oscillate similarly in average, the solid can be
explained in terms of $N$ SHOs. At a given temperature $T$, $N$ atoms
are in thermal equilibrium. Then, the Boltzmann distribution is given
by 
\begin{align}
P_n = \frac1{Z} e^{-E_n/kT}   
\end{align}
with
\begin{align}
  Z = \sum_n e^{-E_n/kT}, 
\end{align}
where
\begin{align}
E_n = \left(n+\frac12\right)\hbar \omega .
\end{align}
\begin{itemize}
\item[(1)] Derive the mean energy per an SHO 
  \begin{align}
    \langle E\rangle = \frac{\hbar\omega}{e^{\hbar\omega/kT}-1} +
    \frac12 \hbar \omega.
  \end{align}
\item[(2)] If $U$ is the internal energy of the solid, derive the
  specific heat with constant volume 
  \begin{align}
    C_V = \frac{\partial U}{\partial T}.
  \end{align}
Show that when $T$ is large, $C_V=3R$.
\item[(3)] Discuss the physics related to this problem as far as you
  can. 
\end{itemize}

\end{document}