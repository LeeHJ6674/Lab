%\documentclass[preprint,tightenlines,showpacs,showkeys,floatfix,
%nofootinbib,superscriptaddress,fleqn]{revtex4} 
\documentclass[floatfix,nofootinbib,superscriptaddress,fleqn]{revtex4} 
%\documentclass[aps,epsfig,tightlines,fleqn]{revtex4}
\usepackage{kotex}
\usepackage[HWP]{dhucs-interword}
\usepackage[dvips]{color}
\usepackage{graphicx}
\usepackage{bm}
%\usepackage{fancyhdr}
%\usepackage{dcolumn}
\usepackage{defcolor}
\usepackage{amsmath}
\usepackage{amsfonts}
\usepackage{amssymb}
\usepackage{amscd}
\usepackage{amsthm}
\usepackage[utf8]{inputenc}
%\pagestyle{fancy}

\begin{document}

\title{\Large Quantum Mechanics}
\author{김현철}
\email{hchkim@inha.ac.kr}
\affiliation{Hadron Theory Group, Department of Physics, Inha University,
Incheon 22212, Republic of Korea }
\date{2021}

\maketitle

\noindent {\bf Due date:} \textbf{\color{red} March 2. 2022} \\ 
\vspace{2cm}

\section*{\large Problem Set 1}
\noindent \textbf{Problem 1.}
The wave function for a free particle is given by
\begin{align*}
\psi(x,\,0) = N \exp\left( i\frac{p_0 x}{\hbar}  -\frac{(x-x_0)^2}{4\sigma^2} \right),   
\end{align*}
where $\sigma \in \mathbb{R}$ is a contant and $N$ is a normalization
constant. 
\begin{itemize}
\item[(1)] Derive the normalization constant $N$.  
\item[(2)] Derive the wave function $\phi(0,\,0)$ in momentum space. 
\item[(3)] Find $\phi(p,\,t)$.
\item[(4)] Find $\psi(x\,t)$.
\item[(5)] Show that the the spread in the spatial probability
  distribution increases with time $t$. Note that the spread is
  defined as 
  \begin{align*}
\mathcal{S}(t) = \frac{|\psi(x,\,t)|^2}{|\psi(0,t)|^2}.    
  \end{align*}
\end{itemize}
\noindent \textbf{Solution}
\begin{itemize}
  \item[(1)] From the normalization of the wave function,
  \begin{align}
    \int_{-\infty}^{\infty} |\psi(x,0)|^2\,dx=N^2\int_{-\infty}^{\infty}\exp\left(-2{\left(\frac{x-x_0}{2\sigma}\right)}^2\right) \,dx =1.
  \end{align}
Since a range of integration is all space, the translation about $x$ can be ignored.
%\begin{align}
%         N^2\int_{-\infty}^{\infty}\exp\left(-2{\left(\frac{x-x_0}{2\sigma}\right)}^2\right) \,dx =N^2\int_{-\infty}^{\infty}\exp\left({-2\left(\frac{x}{2\sigma}\right)}^2\right) \,dx
%\end{align} 
To make a compact form, it needs to change an integral variable.
  \begin{align}
    t \equiv \left(\frac{x-x_0}{\sqrt{2}\sigma}\right),\;\;\; dt = \frac{1}{\sqrt{2}\sigma} dx 
  \end{align}
Then the wave function changes into more comfort form to integrate.
  \begin{align}
    N^2\int_{-\infty}^{\infty}\exp\left(-2{\left(\frac{x-x_0}{2\sigma}\right)}^2\right) \,dx
    %N^2\int_{-\infty}^{\infty}\exp\left(-2{\left(\frac{x}{2\sigma}\right)}^2\right) \,dx =N^2\int_{-\infty}^{\infty}\exp\left({-2\left(\frac{x}{2\sigma}\right)}^2\right) \,dx 
    =\sqrt{2}\sigma N^2\int_{-\infty}^{\infty} e^{-t^2} \,dt
  \end{align}
To calcualte this integration, we use a idea of double integration,
  \begin{align}
    \int_{-\infty}^{\infty} e^{-(x^2+y^2)}\,dx\,dy &= \int_{0}^{2\pi}\int_{0}^{\infty} e^{-r^2}r\,dr\,d\theta
    \\   &= \int_{0}^{2\pi}\frac{1}{2}\,d\theta = \pi.
  \end{align}
First double integration about coordinate space can be decomposed.
  \begin{align}
    \int_{-\infty}^{\infty} e^{-(x^2+y^2)}\,dx\,dy 
    = \int_{-\infty}^{\infty} e^{-x^2}\,dx\int_{-\infty}^{\infty}e^{-y^2}\,dy
    = \left(\int_{-\infty}^{\infty} e^{-x^2}\,dx\right)^2
  \end{align}
From this result,
  \begin{align}
    \int_{-\infty}^{\infty} e^{-x^2} \,dx = \sqrt{\pi},\quad \sqrt{2\pi}\sigma N^2 =1.
  \end{align}
Finally we obtain the normalization constant,
  \begin{align}
    \,N = {\left(\frac{1}{2\pi\sigma^2}\right)}^{\frac{1}{4}}.
  \end{align}
  \item[(2)]  We will find $\phi(p,0)$ first. $\phi(p,0)$ is the Fourier transform of $\psi(x,0)$.
  \begin{align}
    \phi(p,0)=\frac{1}{\sqrt{2\pi\hbar}}
    \int\psi(x,0) e^{-\frac{i}{\hbar}px}\,dx
      =\frac{N}{\sqrt{2\pi\hbar}}
      \int\exp\left(i\frac{p_0 x}{\hbar}-{\left(\frac{x-x_0}{2\sigma}\right)}^2\right) 
      e^{-\frac{i}{\hbar}px}\,dx  \\
      =\frac{N}{\sqrt{2\pi\hbar}}
      \int\exp\left(-{\left(\frac{x-x_0}{2\sigma}\right)}^2
      -\frac{i}{\hbar}(p-p_0)x\right)\,dx
  \end{align}
  To make it compact form, let us erase the translation term and change the variable.
  \begin{align}
    u \equiv \frac{x-x_0}{2\sigma},\;\;\; du = \frac{1}{2\sigma} dx 
  \end{align}
  Then, a $\phi(p,0)$ is,
  \begin{align}
    \phi(p,0)&=\frac{2\sigma N}{\sqrt{2\pi\hbar}}
    \int\exp\left(-{u}^2-\frac{i}{\hbar}(p-p_0)(2\sigma u+x_0)\right)\,du \\
             &={\left(\frac{2\sigma^2}{\pi^3\hbar^2}\right)}^{\frac{1}{4}}
    e^{-\frac{i}{\hbar}(p-p_0)x_0}
    \int\exp\left(-u^2-2\frac{i}{\hbar}\sigma (p-p_0)u\right)\,du.
  \end{align}
  And, a exponential of integrated function can be expressed in terms of complete square form about u.
  \begin{align}
    -u^2-2\frac{i}{\hbar}\sigma (p-p_0)u 
    = -{\left( u+\frac{i}{\hbar}\sigma (p-p_0)\right)}^2-\frac{\sigma^2}{\hbar^2}{(p-p_0)}^2
  \end{align}
  $\frac{i}{\hbar}\sigma p$ is the translation term that can be ignored since the integration range is from $-\infty$ to $\infty$,
  \begin{align}
    \phi(p,0) &= {\left(\frac{2\sigma^2}{\pi^3\hbar^2}\right)}^{\frac{1}{4}}
    e^{-\frac{i}{\hbar}(p-p_0)x_0}
    \int\exp\left(-u^2-2\frac{i}{\hbar}\sigma (p-p_0)u\right)\,du \\
    &={\left(\frac{2\sigma^2}{\pi^3\hbar^2}\right)}^{\frac{1}{4}}
    e^{-\frac{i}{\hbar}(p-p_0)x_0}  
    \int\exp\left(-{\left( u+\frac{i}{\hbar}\sigma (p-p_0)\right)}^2-\frac{\sigma^2}{\hbar^2}(p-p_0)^2\right)\,du \\
    &= {\left(\frac{2\sigma^2}{\pi^3\hbar^2}\right)}^{\frac{1}{4}}
    \exp\left( -\frac{i}{\hbar}(p-p_0)x_0-\frac{\sigma^2}{\hbar^2}(p-p_0)^2 \right)
    \int e^{-u^2}\,du
  \end{align}
  So, we obtain a $\phi(p.0)$.
  \begin{align}
    \phi(p,0) &= {\left(\frac{2\sigma^2}{\pi^3\hbar^2}\right)}^{\frac{1}{4}}
    \exp\left( -\frac{i}{\hbar}(p-p_0)x_0-\frac{\sigma^2}{\hbar^2}(p-p_0)^2 \right)
    \int e^{-u^2}\,du \\
    &= {\left(\frac{2\sigma^2}{\pi\hbar^2}\right)}^{\frac{1}{4}}
    \exp\left( -\frac{i}{\hbar}(p-p_0)x_0-\frac{\sigma^2}{\hbar^2}(p-p_0)^2 \right)
  \end{align}
  Finally, $\phi(0,0)$ is,
  \begin{align}
    \phi(0,0) = {\left(\frac{2\sigma^2}{\pi\hbar^2}\right)}^{\frac{1}{4}}
    \exp\left(-\frac{\sigma^2}{\hbar^2}{p_0}^2+\frac{i}{\hbar}p_0x_0\right).
  \end{align}
  \item[(3)]Because it is a free particle, the time evolution of $\phi(p,0)$ is $\phi(p,t)=e^{-i\omega t}\phi(p,0)$ and $\omega = \frac{p^2}{2m\hbar}$.
  \begin{align}
    \phi(p,t) &=  {\left(\frac{2\sigma^2}{\pi\hbar^2}\right)}^{\frac{1}{4}}
    \exp\left( -\frac{\sigma^2}{\hbar^2}(p-p_0)^2-i\frac{p^2}{2m\hbar} t -\frac{i}{\hbar}(p-p_0)x_0 \right) \\
    &= {\left(\frac{2\sigma^2}{\pi\hbar^2}\right)}^{\frac{1}{4}}
    \exp\left( -\left( \frac{\sigma^2}{\hbar^2} +\frac{it}{2m\hbar}  \right)p^2 + \left( \frac{2\sigma^2}{\hbar^2}p_0-\frac{i}{\hbar}x_0  \right)p - \frac{\sigma^2}{\hbar^2}p_0^2-\frac{i}{\hbar}p_0x_0 \right) \\ 
    &= {\left(\frac{2\sigma^2}{\pi\hbar^2}\right)}^{\frac{1}{4}}
    \exp\left( -\frac{2m\sigma^2+i\hbar t}{2m\hbar^2} p^2 -\frac{2\sigma^2p_0-i\hbar x_0}{\hbar^2}  p-\frac{\left(\sigma^2p_0+i\hbar x_0\right)p_0}{\hbar^2} \right)
  \end{align}
  The complete square form of $\phi(p,t)$ is,
  \begin{align}
    \phi(p,t) &= {\left(\frac{2\sigma^2}{\pi\hbar^2}\right)}^{\frac{1}{4}}
    \exp\left( -\frac{2m\sigma^2+i\hbar t}{2m\hbar^2} p^2 -\frac{2\sigma^2p_0-i\hbar x_0}{\hbar^2}  p-\frac{\left(\sigma^2p_0+i\hbar x_0\right)p_0}{\hbar^2} \right) \\
    &= {\left(\frac{2\sigma^2}{\pi\hbar^2}\right)}^{\frac{1}{4}}
    \exp\left(
      \alpha(t)\left(p+\beta(t)\right)^2+\gamma(t)\right)  \\
        \alpha(t) &= -\frac{2m\sigma^2+i\hbar t}{2m\hbar^2}, \,\,\,
        \beta(t) = \frac{2m\sigma^2p_0-im\hbar x_0}{2m\sigma^2+i\hbar t}, \\
        \gamma(t) &= \frac{-m x_0 \left(\frac{1}{2}\hbar x_0+4i\sigma^2p_0\right)
                    -(\hbar x_0-i\sigma^2 p_0)p_0 t}
                   {2m\hbar\sigma^2+i\hbar^2t}.
  \end{align}
  \item[(4)] $\psi(x,t)$ is the Fourier transform of $\phi(p,t)$.
  \begin{align}
    \psi(x,t) = \frac{1}{\sqrt{2\pi\hbar}}\int\phi(p,t)e^{\frac{i}{\hbar}px}\,dp = 
  \end{align}
  \item[(5)] 
  \end{itemize}
\vspace{0.5cm}

\noindent \textbf{Problem 2.} 
The Hamiltonian for a free particle is given by
\begin{align*}
H  = \frac{p^2}{2m}.
\end{align*}
\begin{itemize}
\item[(1)] Show 
  \begin{align*}
    \langle p_x \rangle = \langle p_x\rangle_{t=0}.
  \end{align*}
\item[(2)] Show 
  \begin{align*}
    \langle x \rangle = \frac{\langle p_x\rangle_{t=0}}{m} t + \langle
    x \rangle_{t=0}.
  \end{align*}
\item[(3)] Show 
  \begin{align*}
(\Delta p_x)^2 =  (\Delta p_x)_{t=0}^2  .
  \end{align*}
\item[(4)] Find $d(\Delta x)^2/dt$ as a function of time and initial 
  conditions. 
\end{itemize}
\noindent \textbf{Solution}
\begin{itemize}
  \item[(1)]The expectation value of physical quantity can be 
  expressed in coordinate space and momentum space each other.
  For free particle, the $\phi(p,t)$ is,
  \begin{align}
    \phi(p,t)=e^{-i\frac{p^2}{2m\hbar}t}\phi(p,0)  .
  \end{align}
  And the expectation value of $p_x$ in the momentum space is,
  \begin{align}
    \langle p_x \rangle &= \int \phi^*(p,t)\,p_x\,\phi(p,t)\,d^3p
    =\int e^{i\frac{p^2}{2m\hbar}t}\phi^*(p,0)\,p_x\,e^{-i\frac{p^2}{2m\hbar}t}\phi(p,0)\,d^3p  \\
    &=\int \phi^*(p,0)\,p_x\,\phi(p,0)\,d^3p = \langle p_x \rangle_{t=0}.
  \end{align}
 
  \item[(2)]
  \item[(3)]
  \item[(4)]   
\end{itemize} 
\newpage

\noindent \textbf{Problem 3.} 
The state of a particle is described by the following wavefunction:
\begin{align*}
\psi(x) = C\exp\left[
i\frac{p_0 x}{\hbar} - \frac{(x-x_0)^2}{2\sigma^2} 
\right]
\end{align*}
where $p_0$, $x_0$, and $a$ are real parameters. 
\begin{itemize}
\item[(1)] Find the normalization constant $C$.
\item[(2)] Find the mean values of $x$ and $p$.
\item[(3)] Find the standard deviations $\Delta x$ and $\Delta p$.
\end{itemize}
\vspace{1cm}

\noindent \textbf{Problem $4^*$.}
Consider a particle and two normalized energy eigenfunctions
$\psi_1(\bm{x})$ and $\psi_2(\bm{x})$ corresponding to the eigenvalues
$E_1\neq E_2$. Assume that the eigenfunctions vanish outside the two 
non-overlapping regions $\Omega_1$ and $\Omega_2$, respectively. 
\begin{itemize}
\item[(1)] (a) Show that, if the particle is initially in region
  $\Omega_1$ then it will stay there forever. 
\item[(b)] If, initially, the particle is in the state with wave function
\begin{align*}
  \psi(\bm{x},\,0) = \frac1{\sqrt{2}} [\psi_1(\bm{x}) +
  \psi_2(\bm{x})] 
\end{align*}
show that the probability density $|\psi(\bm{x},t)|^2$ is independent
of time. 
\item[(c)] Now assume that the two regions $\Omega_1$ and $\Omega_2$
  overlap partially. Starting with the initial wave function of case
  (b), show that the probability density is a periodic function of 
time. ($E_2-E_1=\hbar \omega$).
\item[(d)] Starting with the same initial wave function and assuming
  that the two eigenfunctions are real and isotropic, take the two
  partially overlapping regions $\Omega_1$ and $\Omega_2$ to be 
two concentric spheres of radii $R_1>R_2$. Compute the probability
current that flows through $\Omega_1$.
\end{itemize}
(Problem 4 is a bit difficult. To solve (3), introduce phase
factors for $\psi_1(\bm{x})$ and $\psi_2(\bm{x})$ and consider the
interference term when one computes the probability density. To solve
(4), consider the current density and continuity equation.  
\end{document}